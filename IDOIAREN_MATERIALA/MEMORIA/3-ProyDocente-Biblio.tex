\newpage
\section {Bibliograf\'{\i}a}
\label{PD-biblio}

La bibliograf\'{\i}a de F\'{\i}sica General es muy extensa y el contenido de 
casi todos los libros es bastante similar. Por otra parte, es dif\'{\i}cil
 encontrar un \'{u}nico texto que se adapte perfectamente a los prop\'{o}sitos
 del profesor, aunque cualquier texto es mejor para el alumno que 
los apuntes tomados en las clases. Un aspecto importante  es
 la escasez de libros de texto en lengua vasca. Este hecho implica que 
los alumnos deben realizar un esfuerzo adicional para adaptarse a
 la terminolog\'{\i}a en otras lenguas lo que, por otra parte,
es enriquecedor para el alumno.

La bibliograf\'{\i}a presentada contiene un n\'{u}mero suficiente de 
libros para que los alumnos tengan la posibilidad de elegir uno o varios 
libros de acuerdo con sus gustos, pero el n\'{u}mero de libros propuestos
 tampoco es excesivo para no desanimarlos ni desorientarlos.
 La primera parte de esta secci\'{o}n  incluye libros 
 para desarrollar el contenido te\'{o}rico de la 
asignatura, junto con un breve comentario sobre cada uno de ellos. 
Seguidamente se da una breve bibliograf\'{\i}a de profundizaci\'{o}n
para aquellos alumnos que deseen ampliar su conocimiento a un nivel superior
del impartido en clase.
A continuaci\'{o}n se propone un peque\~{n}o conjunto de libros de problemas. 
Hay que tener en cuenta que 
los libros de contenido te\'{o}rico  poseen todos ellos colecciones 
 de problemas del nivel adecuado; sin embargo, los
 alumnos pueden echar de menos el comprobar c\'{o}mo se resuelve alguno de
 los ejercicios que intentan solucionar. 
Es importante recomendar al alumno que utilice de forma adecuada los libros
 de problemas y no se limite a estudiar la forma en que se resuelven
 los ejercicios, sino que solamente recurra a la soluci\'{o}n una vez que lo haya
 resuelto \'{e}l mismo o haya reflexionado un tiempo suficiente sobre
 el problema.

Finalmente se incluyen varias direcciones de Internet donde pueden encontrarse 
 art\'{\i}culos divulgativos y materiales de inter\'{e}s.


\subsection{Libros de teor\'{\i}a recomendados a los estudiantes}
\begin{itemize}
\item
 TIPLER P. A.,  MOSCA G.,
{\it F\'{\i}sica para la ciencia y la tecnolog\'{\i}a 6$^a$ Ed.},
Editorial Revert\'{e}, Barcelona, 2010.

Es muy apropiado para desarrollar el programa presentado. Su dise\~{n}o es
 muy agradable e incluye una gran variedad de esquemas, ejemplos resueltos, 
cuestiones y problemas.
 Los desarrollos matem\'{a}ticos son tratados 
de forma sencilla, pero es muy rico en conceptos y elegante en los
 razonamientos. Posee adem\'{a}s  material adicional muy \'{u}til para el profesor. Las dos \'{u}ltimas ediciones en castellano, en 6 tomos,
 son muy pr\'{a}cticas.

\item 
FISHBANE, P. M., GASIOROWICZ, S., THORNTON, S. T.,
{\it Physics for Scientists and Engineers, 3$^a$ Ed.},
Addison-Wesley, 2003.

Al igual que el anterior es muy apropiado para desarrollar el programa 
propuesto. Se recomienda
la edici\'{o}n en ingl\'{e}s ya que la traducci\'{o}n al espa\~{n}ol es bastante pobre y 
no est\'{a} muy cuidada.

\item   
 FISHBANE, P. M., GASIOROWICZ, S., THORNTON, S. T. ,
{\it Fisika zientzialari eta ingeniarientzat}. Servicio editorial de la 
UPV/EHU, 2008.

Es la traducci\'{o}n al Euskera de la 2$^a$ edici\'{o}n en ingl\'{e}s. 
Se public\'{o} en 2008 y es el 
segundo libro publicado de F\'{\i}sica General a nivel universitario en lengua vasca.

\item 
YOUNG H. D.,  FREEDMAN R. A., 
{\it Sears  Zemansky F\'{\i}sica Universitaria,  12$^a$ Ed.},
Addison-Wesley, 2009.

Se trata de un texto cl\'{a}sico de F\'{\i}sica General que se ha modernizado al estilo 
de las anteriores referencias y ha mejorado mucho desde el punto de vista pedag\'{o}gico.
Las explicaciones son claras, con numerosos ejemplos y  contiene una 
gran variedad de ejercicios muy interesantes y adecuados para el nivel del curso que se propone.

\item
AGIRREGABIRIA J.M., DUOANDIKOETXEA A., ENSUNZA M.,ETXEBARRIA J.R., EZENARRO O., PITARKE J.M., TRANCHO A. Y UGALDE P., 
{\it  Fisika Orokorra 2$^a$ Ed.} UEU, Bilbao, 2003.

Es un libro basado en  las clases de F\'{\i}sica impartidas en la Facultad de Ciencias y su contenido ha sido elaborado a partir de libros de texto cl\'{a}sicos. De entre
todos los libros propuestos es el \'{u}nico que trata con rigor el tema del 
movimiento relativo tal y como se presenta en este proyecto docente.
Est\'a disponible en internet para su uso libre en 
 {\tt  www.buruxkak.org}

\end{itemize}

\subsection{Bibliograf\'{\i}a de profundizaci\'{o}n}
\begin{itemize}

\item FEYNMAN  R. P., LEIGHTON R. B. y SANDS M. L.,
 {\it The Feynman Lectures on Physics},
 Pearson-Addison-Wesley Iberoamericana 2006. \\
Sus tres vol\'{u}menes est\'{a}n basados en las clases impartidas por
 R. P. Feynman en Caltech en el periodo 1961-1963.
Tienen un enfoque totalmente distinto a los libros actuales de F\'{\i}sica 
universitaria. Est\'{a} estructurado en cap\'{\i}tulos cortos que pueden resultar 
de lectura interesante para aquellos estudiantes m\'{a}s avanzados y motivados.
Muchos de los temas tratados no son objeto del programa de esta asignatura 
y se estudiar\'{a}n en asignaturas de cursos superiores.



\item ALONSO M. y FINN E. J.,
 {\it F\'{\i}sica},
 Addison-Wesley 1995

Se trata de un esfuerzo de s\'{\i}ntesis y de puesta al d\'{\i}a de la edici\'{o}n en 
3 vol\'{u}menes publicada en 1976. Aquella edici\'{o}n era m\'{a}s formal desde el punto de
 vista matem\'{a}tico pero pedag\'{o}gicamente no era tan \'{u}til.
La edici\'{o}n resumida de 1995 cuida m\'{a}s el aspecto pedag\'{o}gico.

\end{itemize}

\subsection{Libros de problemas}
\begin{itemize}


\item HERN\'{A}NDEZ J., TOVAR J.,
 {\it Problemas de F\'{\i}sica: mec\'{a}nica}, 
Universidad de Ja\'{e}n, 2009.

Libro elaborado  a partir de problemas propuestos
en ex\'{a}menes y resueltos en clase durante los \'{u}ltimos a\~{n}os en la 
Escuela Polit\'{e}cnica Superior de la Universidad de Ja\'{e}n por los responsables de la 
asignatura en dicha escuela. Proporciona a los
alumnos un texto de referencia para establecer estrategias adecuadas en la resoluci\'{o}n de 
problemas. Los problemas resueltos est\'{a}n comentados y contienen una gran cantidad 
de figuras y diagramas. Trata todos los temas de mec\'{a}nica salvo la din\'{a}mica de 
fluidos.


\item BURBANO DE ERCILLA S., BURBANO GARC\'{I}A E.,
 {\it Problemas de F\'{\i}sica, 32$^a$ Ed.},
 Editorial T\'ebar, Madrid 2006.

Se trata de un libro muy cl\'{a}sico en la bibliograf\'{\i}a de 
F\'{\i}sica General de primer curso universitario. Contiene ejercicios a todos 
los niveles, desde bachillerato hasta nivel universitario. Es recomendable que los 
estudiantes lo utilicen con precauci\'{o}n, y no
debe utilizarse como una lectura. Al contrario, se animar\'{a} a los estudiantes
a realizar los 
ejercicios con su propia estrategia y utilizar el resultado del libro solo para 
comprobar que han llegado a la soluci\'{o}n correcta.

\item ENSUNZA M., ETXEBARRIA J.R., EZENARRO O., PITARKE J.M., UGALDE P. Y ZABALA N., 
{\it Fisika Orokorra, Ariketak},
 UEU, Iru\~{n}ea 1989.

Es un libro m\'{a}s modesto que los anteriores. Sin embargo tiene un nivel adecuado
para el programa que presento. Adem\'{a}s de contener problemas resueltos
con explicaciones claras ayudadas de diagramas y figuras, propone una serie de 
problemas al final de cada tema e indica la soluci\'{o}n a la que debe llegar el estudiante.

\item HALPERN, A.,
{\it 3000 Solved Problems in Physics. Schaum's Solved Problems Series},
Mc Graw Hill.

Se trata de una edici\'{o}n moderna y actualizada de uno de los libros cl\'{a}sicos de problemas
resueltos de la serie Schaum. Es muy completo. Contiene problemas de todos los temas propuestos en
este proyecto docente adem\'{a}s de temas de F\'{\i}sica General que por limitaciones de tiempo no est\'{a}n 
incluidos en el programa y que se ver\'{a}n en cursos superiores. 
Los problemas est\'{a}n resueltos apoyados por explicaciones claras y muchos diagramas de gran utilidad para 
el alumno. As\'{\i} mismo, y pese a ser un libro estadounidense, todas las unidades utilizadas son del sistema 
internacional.


\end{itemize}

\subsection{Revistas y direcciones de internet}
\begin{itemize}
\item La revista American Journal of Physics, editada por ``American Association of 
Physics Teachers'' presenta a menudo art\'{\i}culos de diferente dificultad 
destinados a profesores y estudiantes de F\'{\i}sica:
 \href{http://scitation.aip.org/ajp/}{http://scitation.aip.org/ajp/} 

\item La Real Sociedad Espa\~{n}ola de F\'{\i}sica (RSEF) en su p\'{a}gina WEB, 
zona de ``links'' da acceso a su revista, en la cual a menudo aparecen
 art\'{\i}culos divulgativos: \href{http://rsef.org}{http://rsef.org}


\item ``Open Courseware'' del Massachusetts Institute of Technology alberga materiales \'{u}tiles de sus cursos de F\'{\i}sica. 
\href{http://ocw.mit.edu/courses/physics/}{http://ocw.mit.edu/courses/physics/} 


\item Curso interactivo de F\'{\i}sica en Internet de Angel Franco, 
del Departamento de F\'{\i}sica Aplicada I de la UPV/EHU.
\href{http://www.sc.ehu.es/sbweb/fisica/}{http://www.sc.ehu.es/sbweb/fisica/}

\item  Repositorio de material educativo del consorcio ``Conceptual Learning of
 Science'': \href{http://www.colos.org/}{http://www.colos.org/}

\item Repositorio de materiales de Open Source Physics.
\href{http://www.compadre.org/osp/}{http://www.compadre.org/osp/}


\item The Python tutorial: 
\href{http://docs.python.org/py3k/tutorial/index.html}{http://docs.python.org/py3k/tutorial/index.html}


\end{itemize}

%\end{chapter}
