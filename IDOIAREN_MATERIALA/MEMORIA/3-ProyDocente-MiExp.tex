%\begin{chapter}{Mi experiencia con la docencia  adaptada al EEES}
\chapter{Mi experiencia con la docencia  adaptada al EEES}
\label{PD-miexp}

En  este apartado me propongo  hacer una reflexi\'{o}n cr\'{\i}tica
basada en mi propia experiencia,
 valorando las ventajas y deventajas del nuevo plan, siempre bajo el punto
de vista restrictivo que implica no abordar la innovaci\'{o}n en
su conjunto, sino s\'{o}lo de una manera parcial en lo referente
a metodolog\'{\i}a y evaluaci\'{o}n.

\section{Cursos piloto en la Licenciatura en Qu\'{\i}mica: 2007-2010}

Durante los cursos acad\'{e}micos 2007/2008,  2008/2009 y
2009/2010 
particip\'{e} en el Programa de Impulso para la Innovaci\'{o}n 
Docente  (IBP) de la UPV/EHU para la implantaci\'{o}n del sistema
de cr\'{e}dito europeo (ECTS), como miembro del equipo docente
encargado de experimentar la nueva metodolog\'{\i}a en un grupo 
piloto correspondiente al primer curso de la Licenciatura en Qu\'{\i}mica.


La asignatura {\it F\'{\i}sica I} para la Licenciatura  en Qu\'{\i}mica
era anual y ten\'{\i}a asignados 9~ECTS, lo que, para un alumno supon\'{\i}a 
 3 clases presenciales por semana, durante 2 cuatrimestres
de 15 semanas cada uno.
De estas clases, dos eran  de teor\'{\i}a
y una de pr\'{a}cticas de aula (PA). 
Debido al n\'{u}mero de 
estudiantes matriculados en el curso (entre 50 y 60), las PA
se  desdoblaron en dos grupos, PA1 y PA2. Todas las clases,
tanto las te\'{o}ricas como las dos PA estaban a mi cargo.

La Gu\'{\i}a Docente  de la Facultad de Ciencia y
Tecnolog\'{\i}a de la UPV/EHU, para el curso acad\'{e}mico 2007/2008,
2008/2009 y  2009/2010,
 establec\'{\i}a las competencias, metodolog\'{\i}a y m\'{e}todo de
evaluaci\'{o}n de la {\it F\'{\i}sica I} del primer curso de
Licenciatura en Qu\'{\i}mica basado en las directrices del ``Plan Bolonia''.
Los contenidos del programa no se modificaron respecto a
los del plan anterior, vigente desde 2000/2001.
De acuerdo con dicha Gu\'{\i}a Docente la {\bf evaluaci\'{o}n} se  llev\'{o} a 
cabo siguiendo los siguientes criterios:

\begin{enumerate}
\item Asistencia, participaci\'{o}n y respuestas dadas a las preguntas 
formuladas en las clases te\'{o}ricas: 5\%.
\item Participaci\'{o}n, trabajo, \'{e}xito en la resoluci\'{o}n 
de ejercicios y m\'{e}todos empleados en las PA: 25\%.
\item Examen escrito: Resoluci\'{o}n de problemas, contestaci\'{o}n
a preguntas y cuestiones te\'{o}ricas: 70\%.
\end{enumerate}

Puesto que el n\'{u}mero de estudiantes  asignado a  cada grupo de PA 
segu\'{\i}a siendo muy elevado, no era posible 
 hacer una evaluaci\'{o}n continua personalizada seg\'{u}n el criterio 2.
Por este motivo,  para complementar la evaluaci\'{o}n del 
 trabajo realizado en clase por el alumnado,
 introduje una tarea extra: la entrega semanal
de una selecci\'{o}n de problemas que  correg\'{\i}a y evaluaba.
Adem\'{a}s, les hac\'{\i}a una prueba escrita (control),
hacia la mitad de cada cuatrimestre, 
cuya finalidad no era otra que ``{o}bligar'' a los 
estudiantes  a llevar la materia al d\'{\i}a. 
Este control no era obligatorio y, como 
incentivo, aquellos  que lo superaban obten\'{\i}an  0.5 puntos
extra en la nota final del cuatrimestre. Los que no lo superaban
no eran penalizados.



Un problema recurrente que  detect\'{e} en la ense\~{n}anza de la F\'{\i}sica
en  la Licenciatura en Qu\'{\i}mica era la gran heterogeneidad de los 
estudiantes que llegaban al primer curso de la Facultad. En efecto, la mayor parte
de ellos, sorprendentemente,
no hab\'{\i}an cursado la F\'{\i}sica de  2{$^\circ$} de Bachiller
ya que la normativa vigente se lo permit\'{\i}a.
En general ``rehu\'{\i}an'' la F\'{\i}sica.
Por otra parte, los estudiantes proced\'{\i}an de diferentes centros de 
Ense\~{n}anza Media, con lo que llegan a la Facultad con bases
muy dispares de F\'{\i}sica y de Matem\'{a}ticas.
A esto hay que a\~{n}adir la falta de inter\'{e}s de una parte importante
del alumnado que no era consciente de que la F\'{\i}sica y la Qu\'{\i}mica
son disciplinas estrechamente unidas, y contemplaban la F\'{\i}sica tan
s\'{o}lo como un 
``obst\'{a}culo'' que deb\'{\i}an superar en los dos primeros cursos de 
licenciatura.

Esta situaci\'{o}n ha dado lugar a que, en alguna ocasi\'{o}n, se haya
llegado a sugerir ``rebajar'' el nivel de exigencia de la 
''F\'{\i}sica para qu\'{\i}micos'' respecto a la 
''F\'{\i}sica para f\'{\i}sicos''.

La soluci\'{o}n a este problema deber\'{\i}a abordarse desde la 
Administraci\'{o}n, mediante una normativa que  obligue a cursar
en   2{$^\circ$} de Bachiller  una F\'{\i}sica con unos contenidos
m\'{\i}nimos  obligatorios para que los estudiantes de  qu\'{\i}mica y de 
otras carreras cient\'{\i}fico-t\'{e}cnicas lleguen a la Universidad
con unos conocimientos b\'{a}sicos adecuados.



En cuanto al {\bf  alumnado}, salvo excepciones, no  percib\'{\i}
un cambio de actitud que implicara su participaci\'{o}n activa en las clases,
 aunque, s\'{o}lo a base
de insistir, constat\'{e}  un cierto  cambio en  algunos de ellos
 a medida que transcurr\'{\i}a el  curso.
 En general exist\'{\i}a una  correlaci\'{o}n entre
los estudiantes que no superaban el curso y los que tomaban
una actitud completamente pasiva en clase, limit\'{a}ndose en el 
mejor de los casos a copiar lo que se escrib\'{\i}a en la pizarra.
Estos son  los que no han entendido que deben ``aprender a aprender''
y que para ello deben actuar con  iniciativa propia
 y cierta autonom\'{\i}a para poder resolver
un problema.

Es indudable que el sistema ECTS aumenta la carga docente del {\bf  profesor}
que debe atender varios grupos de PA y 
preparar con exacta  minuciosidad los contenidos a impartir en cada grupo
 para que ning\'{u}no de ellos  quede desfasado.
Sin embargo, seg\'{u}n mi experiencia, los grupos de PA de 20-30 alumnos
siguen siendo demasiado numerosos para poder hacer un seguimiento
personalizado a todos.
 Adem\'{a}s
debido a la extensi\'{o}n del programa y que gran parte de los estudiantes
no hab\'{\i}an preparado previamente los problemas con los que se iba a trabajar
en clase, era el profesor el que, al final, ten\'{\i}a que acabar
 resolviendo los ejercicios en la pizarra intentando la
 colaboraci\'{o}n de los pocos estudiantes activos de la clase.
Por otra parte, al ser el mismo profesor el responsable de los distintos
subgrupos (PA1 y PA2), procedentes del desdoble, y dado que cada grupo
debe recibir la clase a distinta hora, las PA se convierten  en puras 
repeticiones, lo que resulta  bastante frustrante para el profesor.
Un problema adicional es el posible desfase entre los grupos cuando por 
necesidad de ajuste de horario no se puede impartir la correspondiente 
sesi\'{o}n de PA en el mismo d\'{\i}a a ambos grupos.
Lo ideal es que se impartan el mismo d\'{\i}a, de forma que un d\'{\i}a festivo
no desequilibre la docencia de las PA.

En cuanto a los {\bf  ejercicios}
 que cada alumno deb\'{\i}a entregar semanalmente se daba una cuesti\'{o}n
de complicada soluci\'{o}n para el profesor:
Se trataba de evaluar de forma continua el trabajo que realizaban los estudiantes
utilizando como criterio los problemas entregados.
Sin embargo muchos  se copiaban los problemas entre ellos, 
unas veces bien y otras veces mal. Al cabo de  varias semanas
de corregir estos ejercicios, no era dif\'{\i}cil saber qui\'{e}n
 los hab¡ia trabajado
y qui\'{e}n los hab\'{\i}a copiado 10 minutos antes de la entrega.
La cuesti\'{o}n es: ?`qui\'{e}n debe llevar mejor nota,
 el estudiante que ha entregado 
el ejercicio correcto y que sin  duda lo ha copiado, o 
aquel que el profesor sabe a ciencia cierta que lo ha trabajado y sin embargo
ha cometido alg\'{u}n peque\~{n}o error?
Los criterios de evaluaci\'{o}n deber\'{\i}an ser objetivos y sin 
embargo uno puede estar tentado en este caso  a usar la subjetividad.
Adem\'{a}s son pocos los que modifican sus m\'{e}todos de resoluci\'{o}n
siguiendo las indicaciones que el profesor hace en la correcci\'{o}n.
Esto vuelve a ser causa de frustraci\'{o}n para el profesor
 porque despu\'{e}s de pasar muchas horas corrigiendo ejercicios 
que algunos  estudiantes
se han copiado entre s\'{\i}, queda claro que no alcanzan el objetivo
para el que estaban dise\~{n}ados, a saber, servir de retroalimentaci\'{o}n 
y suministrar informaci\'{o}n tanto al estudiante  como al profesor.
Una soluci\'{o}n  hubiera sido que  realizaran
 los ejercicios  durante la sesi\'{o}n semanal de PA y los entregaran 
al final de la misma.
 Sin embargo esto no 
era factible por la necesidad de cubrir un amplio temario 
en un tiempo limitado. Adem\'{a}s se corr\'{\i}a el riesgo
de que la entrega de ejercicios se conviertiera en un ``examen'' m\'{a}s \'{o}
que el alumnado dejara de asistir a las PA por temor a esta entrega.

En este curso piloto las {\bf tutor\'{\i}as} no fueron 
 de asistencia obligatoria, 
aunque se insisti\'{o} en la conveniencia de su utilizaci\'{o}n regular.
Desafortunadamente, salvo contadas excepciones, los estudiantes
s\'{o}lo las utilizaron la v\'{\i}spera de entrega de ejercicios, 
 examenes o controles, lo que demuestra que 
 no se  mentalizaron de la importancia de llevar 
la asignatura al d\'{\i}a, ni fueron  conscientes
de las ventajas que les supon\'{\i}a disponer de
un apoyo de estas caracter\'{\i}sticas.

Por lo que se refiere a la {\bf evaluaci\'{o}n final},
seg\'{u}n los criterios de evaluci\'{o}n establecidos,
 se pod\'{\i}a dar el caso parad\'{o}jico de que, 
alumnos con una puntuaci\'{o}n de 3.0 (en una escala de 0 a 10)
en el examen escrito, pod\'{\i}an aprobar con un 5 de calificaci\'{o}n final.
Mi experiencia demostr\'{o} que este caso se daba con cierta frecuencia,
especialmente entre aquellos estudiantes cuyas tareas semanales
mostraban claros  indicios de haber sido copiadas, y en la valoraci\'{o}n
global de la prueba escrita se demostraba  f\'{a}cilmente  que no hab\'{\i}an
adquirido las capacidades y destrezas necesarias para superar la asignatura.
Esta situaci\'{o}n parad\'{o}jica era consecuencia de la aplicaci\'{o}n de
los nuevos criterios de evaluaci\'{o}n, anteriormente mencionados, que
asignan a la prueba escrita el 70\%, a las ``tareas de clase'' el 25\%
y a la ``participaci\'{o}n'' el 5\%.
Para evitar esta situaci\'{o}n se  estim\'{o} conveniente  establecer
 en la prueba escrita una nota m\'{\i}nima ``de corte'' no inferior a 4,0.
Los estudiantes con una calificaci\'{o}n  igual 
o superior a \'{e}sta   en la prueba escrita, se  beneficiaban
por este sistema de evaluaci\'{o}n ya que el 30\% restante les eleva la
nota final hasta el aprobado.
Aquellos que hab\'{\i}an obtenido una buena nota en el 
examen, generalmente tambi\'{e}n la ten\'{\i}an en el resto
de las tareas y por lo tanto su calificaci\'{o}n final nunca 
se ve\'{\i}a perjudicada. 


\section{Primer a\~{n}o de la implantaci\'{o}n del ``Plan Bolonia'' en el Grado de F\'{\i}sica, Grado de Matem\'{a}ticas y Grado de Ingenier\'{\i}a Electr\'{o}nica: 2010-2011}

Durante el curso aced\'{e}mico 2010/2011 se implantaron en la Facultad de Ciencia 
y Tecnolog\'{\i}a 9 nuevos grados.
Tres de ellos, 
 el Grado en F\'{\i}sica, el Grado en Mat\'{e}maticas 
y el Grado en Ingenier\'{\i}a Electr\'{o}nica 
comparten la asignatura {\it F\'{\i}sica General} de 12 cr\'{e}ditos ECTS repartidos
en 2 cuatrimestres de 15 semanas cada uno.
La matr\'{\i}cula super\'{o} con creces a las previsiones realizadas por la UPV/EHU y en
 el grupo que se me asign\'{o} me encontr\'{e} con m\'{a}s 140 alumnos, lo que 
me oblig\'{o} a impartir mi docencia en el Paraninfo de la facultad durante 2 meses con 
la consiguiente incomodidad de no disponer de  una pizarra que todos los alumnos puedieran
ver. As\'{\i}, esos dos meses se convirtieron en un marathon de preparaci\'{o}n 
de material en formato electr\'{o}nico para las clases magistrales. Las clases
de problemas (PA) las resolvimos utilizando un retroproyector con una c\'{a}mara CCD que proyectaba a la pantalla del paraninfo lo que se escrib\'{\i}a sobre un papel 
en blanco.

Evidentemente, en estas circunstancias se descart\'{o} totalmente
 el seguimiento 
individualizado de cada alumno seg\'{u}n la filosof\'{\i}a del EEES.
Tras dos meses, el grupo de 140 alumnos se dividi\'{o} en dos grupos con 
horarios distintos: un grupo con los  alumnos
matriculados en el Grado en Matem\'{a}ticas (57) 
y otro grupo con los alumnos del 
Grado en F\'{\i}sica y del Grado en Ingenier\'{\i}a Electr\'{o}nica (83). Ambos
grupos estuvieron a mi cargo durante todo el primer cuatrimestre.

En general mi experiencia de los cursos piloto con los alumnos 
de la Licenciatura
 en Qu\'{\i}micas me fu\'{e} muy \'{u}til. 
Basado en mi experiencia anterior, decid\'{\i} eliminar la entrega de problemas,
porque se hab\'{\i}a demostrado que no era un instrumento \'{u}til
 para evaluar el trabajo realizado en clase,  ya que no hab\'{\i}a dado 
el resultado que se esperaba y adem\'{a}s con tantos alumnos
era totalmente inviable para el profesor.
Esta parte de la evaluaci\'{o}n se sustituy\'{o} por auto-tests realizados v\'{\i}a
{\it  Moodle} que cada alumno realizaba al final de cada tema. Dado que esta era una tarea 
no presencial y pod\'{\i}an discutir sus resultados tanto con sus compa\~{n}eros
como con el profesor antes de entregarla (y de hecho se les animaba a hacerlo)
se redujo su  valor
 en la evaluaci\'{o}n final al 15\% y se aument\'{o} la de los
controles (del 5\% al 15\%) que pasaron a ser de car\'{a}cter obligatorio.
As\'{\i} la evaluaci\'{o}n se 
realiz\'{o} tal y como se ha propuesto en este proyecto docente 
(ver apartado~\ref{PD-evaluacion}).

En lo que se refiere al {\bf alumnado},
 pienso que los estudiantes ten\'{\i}an ya m\'{a}s
interiorizado  la necesidad  de trabajar todos los d\'{\i}as,
 que cuando impart\'{\i}a la docencia en el
 curso piloto. 
Quiz\'{a}s el hecho de que todos los estudiantes de 
primer curso estuvieran utilizando la misma metodolog\'{\i}a y no ser un grupo reducido
al que se le ``exig\'{\i}a'' trabajar de un modo continuo mientras 
 a los dem\'{a}s 
se les aplicaba  el m\'{e}todo tradicional, influyera en este cambio 
de actitud.

En cuanto a las {\bf  tutor\'{\i}as} 
mi experiencia ha sido similar a la que tuve en el curso
piloto de la Licenciatura en Qu\'{\i}micas. 

En general, y teniendo en cuenta todas las dificultades que se
presentan durante una 
transici\'{o}n tras un cambio tan dr\'{a}stico
 como el que supone la implantaci\'{o}n de los 
nuevos grados,
la experiencia ha sido positiva.
 Sin embargo exige, sin duda alguna, mucho m\'{a}s
 esfuerzo, planificaci\'{o}n y coordinaci\'{o}n por parte del profesor.

\section{En la Universidad de Cambridge: 2004-2006}

Durante mi estancia postdoctoral en la Universidad de Cambridge, tuve la 
oportunidad de conocer su sistema universitario de 
``supervisions'' y participar en \'{e}l como supervisora de 
la asignatura ``Physics IA'' en el Robinson College.
 La filosof\'{\i}a no es muy distinta
a la del ``Plan Bolonia'', sin embargo, es extraordinariamente eficiente.
Los estudiantes, al igual que  en la 
Facultad de Ciencia y Tecnolog\'{\i}a, eligen
  entre las asignaturas comunes aquellas que son obligatorias
del grado que desean cursar. En el caso de F\'{\i}sica asisten
a 3 horas de clases magistrales  y a una ``supervision'' por semana.
Una ``supervision'' es similar a una tutor\'{\i}a en la que los alumnos, 
en grupos de {\bf dos} o {\bf tres},
 discuten los problemas que se les han asignado 
en clase con un ``supervisor''. El d\'{\i}a anterior a la ``supervision''
le entregan los ejercicios resueltos para que \'{e}ste los corrija.
As\'{\i} parte de la ``supervision'' se convierte en una discusi\'{o}n
tomando como base el trabajo realizado por el alumno durante la semana.
La mayor\'{\i}a de los que desempe\~{n}an  la labor de  ``supervisor'' son
doctorandos o investigadores postdoctorales, como fue
mi caso, lo que supone una ventaja tanto para el ``supervisor'' como
para el alumno, pues una mayor cercan\'{\i}a entre ambos supone una
mayor confianza y por tanto mayor fluidez en el intercambio de 
conocimientos, lo  que influir\'{a}, sin duda, muy positivamente en los resultados,
que es lo que se pretende.

Este sistema permite hacer un seguimiento 
muy cercano de la evoluci\'{o}n de los alumnos.
La actitud de los alumnos es muy positiva ya que
  son plenamente conscientes del privilegio que
tienen de participar de  un sistema educativo de \'{e}lite.
Previamente tuvieron que pasar por procesos de selecci\'{o}n 
muy duros para ser admitidos, y por ello  hacen
un gran esfuerzo por participar activamente en el proceso de aprendizaje
incluso si las asignaturas no son las espec\'{\i}ficas de su grado. Por 
supuesto, no se plantean la inasistencia a clase o a una ``supervision''.
La tasa de \'{e}xito es practicamente de un 100\%.


 Quiz\'{a}s
sin llegar al modelo de la Universidad de Cambridge,
en el que las PA se hacen en grupos de dos alumnos,
ser\'{\i}a conveniente reducir en todo caso las PA a
grupos de menos de 10 alumnos. Por supuesto, para ello hace falta una
plantilla docente suficiente que pueda absorber tal encargo docente.
Soy consciente de que en la situci\'{o}n actual y con 
las dotaciones econ\'{o}micas de la universidad espa\~{n}ola
 esto no deja de ser una 
utop\'{\i}a.

\section{Conclusiones}


En mi opini\'{o}n el sistema  adaptado al EEES podr\'{\i}a ser muy provechoso
para el estudiante, ya que est\'{a} dise\~{n}ado
``a su medida''.


Para ello, el alumnado deber\'{a} mentalizarse que debe cambiar
su  actitud pasiva  por una actitud activa, para  sacar
el mayor provecho al proceso de ense\~{n}anza-aprendizaje.
Han sido muchos a\~{n}os de ense\~{n}anza magistral y ex\'{a}menes finales
para que un cambio de mentalidad tan amplio como  exige  la 
aplicaci\'{o}n del ``Plan Bolonia'' sea asimilado en poco tiempo 
por las actuales generaciones
de estudiantes.

 El d\'{\i}a que los estudiantes sean conscientes de que:
\begin{itemize} 
\item Estudiar en la Universidad es un privilegio y no una obligaci\'{o}n.
\item Tienen que ``aprender'' y no s\'{o}lo ``dejarse ense\~{n}ar''.
\item No tener clase no significa ``no tener que estudiar''.
\item Una clase bien aprovechada no equivale a ``haberlo copiado todo''.
\item No todo lo que hay que saber est\'{a} en los ``apuntes''.
\item Hay muchas formas de estudiar sin tener un libro delante.
\item Los conocimientos que no hayan adquirido por su cuenta, mediante todos los
recursos a su alcance, no les servir\'{a}n para resolver los problemas
que se le plantear\'{a}n en el futuro.
\end{itemize}
ser\'{a}n capaces de comprender que la innovaci\'{o}n que supone la 
aplicaci\'{o}n del ``Plan Bolonia'' les van a favorecer tanto m\'{a}s cuanto 
mayor sea su implicaci\'{o}n en \'{e}l.

Conf\'{\i}o, porque as\'{\i} me gustar\'{\i}a que fuera, que esto pueda
lograrse en el menor periodo de tiempo posible.



%\end{chapter}
