
\newpage
\section {Contenidos}

En el marco del EEES, 
y para facilitar la movilidad entre titulaciones de una misma rama de
conocimiento se establecen  contenidos formativos comunes a varias
titulaciones. Esta formaci\'{o}n com\'{u}n
no consiste en cursos id\'{e}nticos para todas las titulaciones
sino conocimientos b\'{a}sicos compartidos o competencias comunes
a todas ellas.

La {\it F\'{\i}sica General} de
la Facultad de Ciencia y Tecnolog\'{\i}a de la UPV/EHU es com\'{u}n
para los Grados en F\'{\i}sica,  Ingenier\'{\i}a Electr\'{o}nica,
y Matem\'{a}ticas. 
Es una asignatura anual de 12 ECTS que se imparte durante 2 cuatrimestres de 15 
semanas cada uno. Se proponen 36 clases magistrales y 
 21 clases de pr\'{a}cticas de aula por cuatrimestre. Las 6 clases restantes
se realizar\'{a}n en formato seminario, 2 en el primer cuatrimestre y 4
en el segundo.


Con respecto al plan anterior, el n\'{u}mero de horas presenciales
de F\'{\i}sica de primer curso
para el Grado en Matem\'{a}ticas no se modifica. Sin embargo, 
  los Grados en  F\'{\i}sica  e 
Ingenier\'{\i}a Electr\'{o}nica  reciben una hora menos por semana.
Dado que el contenido  
est\'{a} basado fundamentalmente en el programa de
{\it Fundamentos de F\'{\i}sica} de la Licenciatura en F\'{\i}sica
del plan 2000-2001, ha sido necesario ajustar los contenidos
del programa para cumplir el n\'{u}mero de cr\'{e}ditos asignado. 

En cuanto a la asignatura pr\'{a}ctica de laboratorio,
 contendr\'{a} \'{u}nicamente  pr\'{a}cticas de F\'{\i}sica
con la denominaci\'{o}n
{\it T\'{e}cnicas Experimentales I}.


A continuaci\'{o}n se especifica
el programa  de   {\it F\'{\i}sica General} y
{\it T\'{e}cnicas Experimentales I}, adaptado al sistema ECTS.



\subsection{Programa  de teor\'{\i}a: F\'{\i}sica General}
\subsubsection{\large Antecedentes y justificaci\'{o}n}

El programa de la asignatura {\it  F\'{\i}sica General}, objeto
del concurso,
contiene, en principio, gran parte de la F\'{\i}sica a 
 nivel b\'{a}sico.

El programa que presento a continuaci\'{o}n est\'{a} basado 
en el de la asignatura  {\it Fundamentos de F\'{\i}sica}
del plan de estudios  2000-2001.
La implantaci\'{o}n del plan  de estudios en el curso
2000-2001 ya supuso una reducci\'{o}n en el
  n\'{u}mero de cr\'{e}ditos asignado a la
asignatura {\it Fundamentos de F\'{\i}sica} 
con respecto a la asignatura {\it F\'{\i}sica General} del plan anterior,
 de 18 cr\'{e}ditos (6 horas semanales) a 15 cr\'{e}ditos (5 horas semanales), lo que
 implic\'{o} un recorte del programa de la {\it F\'{\i}sica General} ``tradicional''.
Es importante se\~{n}alar, que esta modificaci\'{o}n del temario no se produjo
de forma arbitraria, sino que se realiz\'{o} teniendo en cuenta 
los contenidos del programa oficial de ``F\'{\i}sica''  de 2{$^\circ$}
de Bachiller (Vibraciones y Ondas, \'{O}ptica, interacci\'{o}n gravitatoria,
e introducci\'{o}n a la F\'{\i}sica Moderna)
y el 
contexto en el que se enmarcaba la asignatura {\it Fundamentos de F\'{\i}sica} 
dentro de la Licenciatura en F\'{\i}sica.


De este modo, teniendo en cuenta la composici\'{o}n de aquel  plan
y la reducci\'{o}n de cr\'{e}ditos impuesta en el programa de la asignatura {\it Fundamentos de F\'{\i}sica}, 
se modificaron los siguientes temas correspondientes al programa 
de {\it F\'{\i}sica General} del plan anterior al 2000:
\begin {itemize}
\item {\bf  Est\'{a}tica y din\'{a}mica de fluidos}.
 Se elimin\'{o} este cap\'{\i}tulo por la mayor importancia que adquiri\'{o} 
esta materia en el  plan 2000-2001, 
al aparecer expl\'{\i}citamente en las asignaturas
 {\it Mec\'{a}nica y Ondas} de segundo curso y 
{\it F\'{\i}sica de los medios continuos} de tercero.
\item {\bf  Termodin\'{a}mica}.
 Se suprimi\'{o} por completo la parte dedicada a la
 Termodin\'{a}mica debido a que, como ya ocurr\'{\i}a en el plan antiguo, 
la asignatura {\it Termodin\'{a}mica} del segundo curso part\'{\i}a desde
  principios b\'{a}sicos.
\item {\bf Introducci\'{o}n a la F\'{\i}sica Cu\'{a}ntica}.
 Se elimin\'{o} por la presencia en el plan 2000-2001 de una 
nueva asignatura denominada 
{\it Introducci\'{o}n a la estructura de la materia}
 que, fundamentalmente, es una introducci\'{o}n a 
la F\'{\i}sica Moderna, que no exist\'{\i}a en el plan anterior.
\item {\bf Din\'{a}mica del S\'{o}lido R\'{\i}gido}:
 Con la implantaci\'{o}n del plan de estudios 2000-2001 el 
n\'{u}mero de horas dedicadas al estudio del
 s\'{o}lido r\'{\i}gido se vi\'{o} reducido,
limit\'{a}ndose al estudio de la rotaci\'{o}n de un s\'{o}lido
 alrededor de sus ejes principales de inercia, teorema de Steiner, 
radio de giro, energ\'{\i}a cin\'{e}tica del s\'{o}lido, 
movimiento de rodadura y condiciones de equilibrio de un cuerpo r\'{\i}gido.
 La parte  eliminada, que correspond\'{\i}a al estudio del
 tensor de inercia, \'{a}ngulos de Euler, 
ecuaciones de movimiento y movimiento girosc\'{o}pico 
se desarrollar\'{\i}a en la asignatura {\it Mec\'{a}nica y Ondas}
 de segundo curso.
\item{\bf  \'{O}ptica}: 
Esta parte se vi\'{o}  reducida y
limitada al estudio de la \'{o}ptica geom\'{e}trica, 
ya que en la asignatura {\it \'{O}ptica} de tercer curso se presupon\'{\i}an
 unos conocimientos m\'{\i}nimos al respecto.
\end{itemize}

El nuevo Grado en F\'{\i}sica ha impuesto en la 
 asignatura  {\it F\'{\i}sica General} 
 una nueva reducci\'{o}n en el n\'{u}mero clases presenciales 
semanales, de 5 a 4 horas, 
y su objetivo no es abarcar {\it toda} la F\'{\i}sica General
``tradicional'', sino que pretende establecer unas bases s\'{o}lidas
de conocimiento y razonamiento que le sirvan al alumno
para abordar con \'{e}xito las materias en la que se profundizar\'{a}
en los cursos superiores.

A pesar de la reducci\'{o}n de horas presenciales de la asignatura en el grado, 
el contenido respecto al plan de 2000-2001 no se ha reducido sustancialmente. 
Al contrario. Se ha recuperado el tema correspondiente a la est\'{a}tica 
y din\'{a}mica de fluidos de forma que los estudiantes del Grado en Matem\'{a}ticas
 e Ingenier\'{\i}a Electr\'{o}nica tengan una formaci\'{o}n b\'{a}sica en esta \'{a}rea
ya que no cursar\'{a}n la asignatura optativa {\it F\'{\i}sica de los medios continuos} en 3{$^\circ$} o 4{$^\circ$} curso.


 Debido a la extensi\'{o}n del programa, he cre\'{\i}do conveniente 
dividirlo en  tres bloques: Introducci\'{o}n,
 Mec\'{a}nica y Electromagnetismo. Estos  tres
 bloques tem\'{a}ticos tienen entidad propia, si bien el estudio de 
algunos de ellos requiere unos conocimientos adquiridos 
previamente.
 Sobre algunos de los temas los estudiantes  ya 
tienen alguna informaci\'{o}n b\'{a}sica recibida en Bachiller,
 en particular del bloque dedicado al Electromagnetismo. 
Sin embargo, el bloque dedicado a la Mec\'{a}nica 
supone una  mayor dificultad  para gran
 parte del alumnado. Una de las razones es que, en la actualidad,
la asignatura {\it Mec\'{a}nica} ya no se imparte  en Bachiller.
Por otra parte, todos los bloques tem\'{a}ticos aqu\'{\i} 
presentados corresponden a asignaturas completas de cursos superiores.


El programa actual de  {\it  F\'{\i}sica General}
comienza con una introducci\'{o}n sobre la naturaleza 
y m\'{e}todo de la F\'{\i}sica, seguida por un conjunto de temas dedicados
 al \'{a}lgebra vectorial. El segundo bloque  corresponde al estudio 
de la Mec\'{a}nica, que 
se volver\'{a} a ver en el  segundo curso
con un enfoque  m\'{a}s elevado: formalismo lagrangiano y hamiltoniano. 
A continuaci\'{o}n se incluye un apartado sobre el Electromagnetismo, 
disciplina que en cursos superiores se desdobla en varias asignaturas, 
por un lado {\it Electromagnetismo I y II} en el segundo y tercer curso 
respectivamente, y por otro, {\it \'{O}ptica}
 en el tercer curso del grado.
 En ambos casos se presuponen unos conocimientos previos por parte 
de los alumnos que cursen los Grados en F\'{\i}sica e Ingenier\'{\i}a Electr\'{o}nica. 

\newpage
\subsubsection{\large Esquema del programa te\'{o}rico}
\label{esquema-bolonia}


\noindent {\bf I - INTRODUCCI\'{O}N}

\begin{enumerate}[{\bf 1. }]
%
\item {\bf  Naturaleza y m\'{e}todo de la F\'{\i}sica}\hfill (0.1 ECTS)
\begin{itemize}\addtolength{\itemsep}{-0.25\baselineskip}
\noindent
\item  ?`Qu\'{e} es la F\'{\i}sica?
\item  El m\'{e}todo cient\'{\i}fico.
\item Part\'{\i}culas e interacciones. 
\item La estructura de las leyes f\'{\i}sicas, simetr\'{\i}a y leyes de conservaci\'{o}n. 
\item El Mundo material:  estados de agregaci\'{o}n de la materia.
\end{itemize}
%

\item {\bf  Magnitudes F\'{\i}sicas y vectores} \hfill    (0.4 ECTS)
\begin{itemize} \addtolength{\itemsep}{-0.25\baselineskip}
\noindent
\item   Magnitudes f\'{\i}sicas. Escalares y vectores.
\item  Magnitudes fundamentales y derivadas. Ecuaci\'{o}n de dimensiones.
 An\'{a}lisis dimensional.
\item  {O}rdenes de magnitud y cifras significativas.
\item Vectores libres.
% {\color{red} Cursores, pseudovectores}.
\item Operaciones con vectores libres: adici\'{o}n, producto de un vector por 
un escalar, producto escalar, producto vectorial, producto mixto, 
 doble producto vectorial. 
\item Sistemas de coordenadas y componentes de un vector. 
Triedro directo e inverso.
\item Campos escalares y vectoriales. Ejemplos.
\item Derivada de un vector respecto de un escalar.
\end{itemize}
%


\suspend{enumerate}
\noindent {\bf   II - MEC\'{A}NICA}

\resume{enumerate}[{[\bf 1.] }]
\item {\bf Cinem\'{a}tica del punto material} \hfill (0.6 ECTS)
\begin{itemize} \addtolength{\itemsep}{-0.25\baselineskip}
\noindent
\item  Sistemas de referencia.
\item Posici\'{o}n, velocidad y aceleraci\'{o}n de una part\'{\i}cula.
\item Movimiento con aceleraci\'{o}n constante.
\item Componentes intr\'{\i}nsecas de la velocidad y de la aceleraci\'{o}n.
\item Movimiento circular.
\item Movimiento curvil\'{\i}neo en el plano. Coordenadas polares. Velocidad radial y areolar.
\end{itemize}
%

\item {\bf  Movimiento relativo}\hfill (0.5 ECTS)
\begin{itemize} \addtolength{\itemsep}{-0.25\baselineskip}
\noindent
\item  Sistemas de referencia en movimiento relativo con velocidad constante.
 Transformaci\'{o}n de Galileo.
\item Sistemas de referencia en movimiento relativo de rotaci\'{o}n.
 Aceleraci\'{o}n de Coriolis. Ejemplos.
\item Ecuaciones de transformaci\'{o}n de la velocidad y de la
 aceleraci\'{o}n en el caso general.
\end{itemize}
%

\item {\bf  Din\'{a}mica del punto material}\hfill  (0.6 ECTS)
\begin{itemize} \addtolength{\itemsep}{-0.25\baselineskip}
\noindent
\item Principio de inercia: primera ley de Newton. Sistemas de referencia inerciales.
\item Momento lineal. Masa inercial.
\item Definici\'{o}n de fuerza: segunda ley de Newton. Principio de conservaci\'{o}n del momento lineal.
\item  Ley de acci\'{o}n-reacci\'{o}n, tercera ley de Newton.
\item  Fuerzas de contacto: reacci\'{o}n normal y resistencia al deslizamiento. Fuerzas a distancia.
\item  Resoluci\'{o}n de las ecuaciones del movimiento bajo distintos tipos de fuerzas:
Fuerzas constantes, fuerza ejercida por un muelle, fuerza de rozamiento proporcional a la velocidad.
\item  Momento angular. Momento de una fuerza respecto a un punto.
\item  Fuerzas centrales. Conservaci\'{o}n del momento angular. 
Movimiento de una part\'{\i}cula sometida a una fuerza central.
\item  Sistemas de referencia no inerciales. Fuerzas de inercia.
 La Tierra como sistema de referencia. El p\'{e}ndulo de Foucault.
\end{itemize}
%

\item {\bf Trabajo y energ\'{\i}a} \hfill (0.6 ECTS)
\begin{itemize} \addtolength{\itemsep}{-0.25\baselineskip}
\noindent
\item Trabajo de una fuerza. Potencia.
\item Trabajo y energ\'{\i}a cin\'{e}tica.
\item Fuerzas conservativas. Energ\'{\i}a potencial. Gradiente de un campo escalar.
\item Conservaci\'{o}n de la energ\'{\i}a mec\'{a}nica.
\item Energ\'{\i}a y equilibrio. Equilibrio estable e inestable.
\item Fuerzas no conservativas.
\end{itemize}
%

\item {\bf Din\'{a}mica de los sistemas de part\'{\i}culas}\hfill  (0.6 ECTS)
\begin{itemize} \addtolength{\itemsep}{-0.25\baselineskip}
\noindent
\item Fuerzas internas y externas. Ecuaciones del movimiento. Conservaci\'{o}n del momento lineal.
\item Centro de masa. Sistema de referencia del centro de masa.
 Descripci\'{o}n del movimiento  del centro de masa del sistema.
\item Momento angular. Conservaci\'{o}n del momento angular.
\item Trabajo realizado por las fuerzas internas y externas. Energ\'{\i}a
 cin\'{e}tica.
\item Fuerzas internas conservativas. Energ\'{\i}a potencial interna. 
Teorema de conservaci\'{o}n de la energ\'{\i}a. Energ\'{\i}a propia.
\item Fuerzas impulsivas y de impacto. Colisiones. Leyes de conservaci\'{o}n.
\item Experimentos en aceleradores. Creaci\'{o}n de part\'{\i}culas.
\item Sistemas de masa variable.
\end{itemize}
%

\item {\bf  Din\'{a}mica del s\'{o}lido r\'{\i}gido} \hfill (0.6 ECTS)
\begin{itemize} \addtolength{\itemsep}{-0.25\baselineskip}
\noindent
\item Energ\'{\i}a cin\'{e}tica de rotaci\'{o}n.
\item Ejes principales y momentos inercia.
\item C\'{a}lculo de momentos de inercia. F\'{o}rmula de Steiner y teorema de los ejes perpendiculares. Radio de giro.
\item Din\'{a}mica del s\'{o}lido r\'{\i}gido: Momento angular. Teorema del momento angular. Energ\'{\i}a. 
\item Estudio de casos particulares: rotaci\'{o}n alrededor de un eje fijo, 
movimiento de rodadura.
\item Condiciones de equilibrio de un cuerpo r\'{\i}gido.
\end{itemize}
%

\item {\bf  Interacci\'{o}n gravitatoria} \hfill (0.5 ECTS)
\begin{itemize} \addtolength{\itemsep}{-0.25\baselineskip}
\noindent
\item Introducci\'{o}n hist\'{o}rica.
\item Leyes de Kepler.
\item Ley de la gravitaci\'{o}n universal.
\item Experimento  de Cavendish.
\item Campo y potencial gravitatorio. Ley de Gauss.
\item Energ\'{\i}a potencial gravitatoria.
\item El problema de dos cuerpos. Reducci\'{o}n al problema equivalente de 
un solo cuerpo. Masa reducida.
\item Clasificaci\'{o}n de las \'{o}rbitas. Movimiento orbital: {O}rbitas circulares y \'{o}rbitas el\'{\i}pticas. Excentricidad y energ\'{\i}a.
\end{itemize}
%


\item {\bf  Fluidos}  \hfill (0.3 ECTS)
\begin{itemize} \addtolength{\itemsep}{-0.25\baselineskip}
\noindent
\item Propiedades y clasificaci\'{o}n de los fluidos.
\item Hidrost\'{a}tica: Ecuaci\'{o}n fundamental de la hidrost\'{a}tica.
\item Flotaci\'{o}n, empuje y  Principio de Arqu\'{\i}medes.
\item Hidrodin\'{a}mica: Flujo de un campo vectorial y ecuaci\'{o}n de continuidad. 
\item Ecuaci\'{o}n de Bernouilli y aplicaciones.
\item Viscosidad. Ley de Poiseuille.
\end{itemize}
%


\item {\bf  Movimiento oscilatorio}\hfill   (0.4 ECTS)
\begin{itemize} \addtolength{\itemsep}{-0.25\baselineskip}
\noindent
\item  El oscilador arm\'{o}nico simple. Ecuaci\'{o}n del movimiento.
\item Energ\'{\i}a del oscilador arm\'{o}nico.
\item Ejemplos de osciladores: Resorte lineal. P\'{e}ndulo simple.
 P\'{e}ndulo f\'{\i}sico.  P\'{e}éndulo de torsi\'{o}n.
\item Composici\'{o}n de movimientos arm\'{o}nicos.
\item Oscilaciones amortiguadas. Clases de amortiguamiento.
\item Oscilaciones forzadas.  Resonancia.
\end{itemize}
%

\item {\bf  Movimiento ondulatorio.}\hfill      (0.5 ECTS)
\begin{itemize} \addtolength{\itemsep}{-0.25\baselineskip}
\noindent
\item   Descripci\'{o}n matem\'{a}tica de la propagaci\'{o}n de una perturbaci\'{o}n.
\item La ecuaci\'{o}n de ondas.
\item Ondas en una dimensi\'{o}n. Ondas transversales en una cuerda. Ondas longitudinales en una barra met\'{a}lica.
\item Ondas arm\'{o}nicas. Magnitudes fundamentales. Superposici\'{o}n. Interferencia de ondas arm\'{o}nicas. 
\item Ondas estacionarias en una cuerda y en una columna de aire. Modos de vibraci\'{o}n. 
\item Efecto Doppler. 
\end{itemize}
%

\suspend{enumerate}
\noindent {\bf   III - ELECTROMAGNETISMO}

\resume{enumerate}[{[\bf 1.] }]
\item {\bf Interacci\'{o}n electrost\'{a}tica}    \hfill (0.9 ECTS)
\begin{itemize} \addtolength{\itemsep}{-0.25\baselineskip}
\noindent
\item Introducci\'{o}n. %hist\'{o}rica.
\item Carga el\'{e}ctrica. Conservaci\'{o}n y cuantificaci\'{o}n de la carga
 el\'{e}ctrica. Estructura del \'{a}tomo.
\item Ley de Coulomb. Campo el\'{e}ctrico. Potencial el\'{e}ctrico.
\item C\'{a}lculo del campo y potencial el\'{e}ctricos debidos a distintas
 distribuciones de carga. L\'{\i}neas de campo y superficies equipotenciales.
\item  Flujo del campo el\'{e}ctrico. Ley de Gauss. Aplicaciones.
\item  Energ\'{\i}a de una distribuci\'{o}n de cargas.
\item Dipolo el\'{e}ctrico. Momento dipolar. 
Acci\'{o}n del campo el\'{e}ctrico sobre un dipolo.
\item Clasificaci\'{o}n de los materiales por su comportamiento frente a un
 campo el\'{e}ctrico. Conductores y aislantes.
\item Materiales diel\'{e}ctricos en campos electrost\'{a}ticos.
 Polarizabilidad de un medio diel\'{e}ctrico. Polarizaci\'{o}n.
\item  Densidades de carga de polarizaci\'{o}n.
\end{itemize}
%
%
\item {\bf Materiales conductores en campos electrost\'{a}ticos} \hfill (0.4 ECTS) 
\begin{itemize} \addtolength{\itemsep}{-0.25\baselineskip}
%\noindent
\item  Campo y potencial el\'{e}ctricos en el interior de un conductor en equilibrio. Reparto de las cargas.
\item  Campo en la superficie de un conductor. Poder de las puntas.
\item Capacidad de un conductor aislado.
\item Condensadores. Capacidad. Energ\'{\i}a.
\item  Asociaci\'{o}n de condensadores: capacidad equivalente y energ\'{\i}a.
\item Densidad de energ\'{\i}a asociada al campo el\'{e}ctrico.
%\item Materiales diel\'{e}ctricos en campos electrost\'{a}ticos.
% Polarizabilidad de un medio diel\'{e}ctrico. Polarizaci\'{o}n.
%\item  Densidades de carga de polarizaci\'{o}n.
\item  Efecto de un diel\'{e}ctrico sobre la capacidad de un condensador.

\end{itemize}

% %%%%%%%%%%%%5
% HAU LABURTUTA aurreko bi gaietan
%%%%%%%%%%%%%%%%%%%
%\item {\bf Materiales diel\'{e}ctricos en campos electrost\'{a}ticos} \hfill{\color{red} ( 6 horas ) }
%\begin{itemize} \addtolength{\itemsep}{-0.25\baselineskip}
%\noindent
%\item   Estructura molecular y propiedades diel\'{e}ctricas: la aproximaci\'{o}n dipolar.
%\item Polarizabilidad de un medio diel\'{e}ctrico. Polarizaci\'{o}n.
%\item  Densidades de carga de polarizaci\'{o}n.
%\item Susceptibilidad y permitividad diel\'{e}ctrica. Desplazamiento el\'{e}ctrico. 
%Generalizaci\'{o}n de la ley de Gauss en presencia de diel\'{e}ctricos.
%\item  Efecto de un diel\'{e}ctrico sobre la capacidad de un condensador.
%\item  Energ\'{\i}a electrost\'{a}tica en presencia de diel\'{e}ctricos.
%\end{itemize}


\item {\bf Corriente el\'{e}ctrica. Circuitos de corriente continua} \hfill (0.9 ECTS)
\begin{itemize} \addtolength{\itemsep}{-0.25\baselineskip}
\noindent
\item   Corriente y densidad de corriente.
\item  Ecuaci\'{o}n de continuidad.
\item  Corrientes de conducci\'{o}n. Ley de Ohm. Conductividad. Resistencia.
\item  Punto de vista microsc\'{o}pico. Modelo de Drude.
\item  Fuerza electromotriz.
\item  Disipaci\'{o}n de energ\'{\i}a. Ley de Joule.
\item  Combinaciones de resistencias. Leyes de Kirchhoff.
\item  Circuitos RC.
\end{itemize}


\item {\bf Interacci\'{o}n magn\'{e}tica} \hfill (1 ECTS) 
\begin{itemize} \addtolength{\itemsep}{-0.25\baselineskip}
\noindent
\item    Introducci\'{o}n.
\item  Fuerza magn\'{e}tica sobre una carga en movimiento.
\item  Movimiento de una part\'{\i}cula cargada en un campo magn\'{e}tico. 
Aplicaciones: selector de velocidades, espectr\'{o}metro de masas, ciclotr\'{o}n.
\item  Acci\'{o}n de un campo magn\'{e}tico sobre una corriente el\'{e}ctrica.
\item  Momento dipolar  magn\'{e}tico. Momento sobre una espira en un campo magn\'{e}tico.
%\item  Efecto Hall.
\item  Campo magn\'{e}tico creado por una carga en movimiento.
\item  Campo creado por una corriente el\'{e}ctrica: ley de Biot y Savart. Ejemplos.
\item  Fuerzas entre corrientes.
\item  Ley de Amp\`{e}re. Ejemplos.
%\item Dipolo magn\'{e}tico. Campo creado por un dipolo magn\'{e}tico.
\item  Ley de Gauss para el campo magn\'{e}tico.
\item Campo magn\'{e}tico en la materia.  Imanaci\'{o}n.
\item Susceptibilidad y permeabilidad magn\'{e}ticas.
\item Diamagnetismo, paramagnetismo y ferromagnetismo.
\end{itemize}

%%%%%%%%%%%%%%%%%%%%%%%%%%%%%%%%%%%%%%%%%%%%%%%%%%%%%%%%%%%%%%%
%%%%%% HAU LABURTUTA \'{I}NTERACCION MAGENTICA' GAIAN %%%%%%%%%
%%%%%%%%%%%%%%%%%%%%%%%%%%%%%%%%%%%%%%%%%%%%%%%%%%%%%%%%%%%%%%%
%\item {\bf Campo magn\'{e}tico en la materia} \hfill{\color{red} ( 6 horas ) }
%\begin{itemize} \addtolength{\itemsep}{-0.25\baselineskip}
%\noindent
%\item    Imanaci\'{o}n
%\item  Densidad de corriente de imanaci\'{o}n. Generalizaci\'{o}n de la ley de Amp\`{e}re. Excitaci\'{o}n magn\'{e}tica. Susceptibilidad y permeabilidad magn\'{e}ticas.
%\item El momento dipolar magn\'{e}tico de los \'{a}tomos.
%\item  Diamagnetismo, paramagnetismo y ferromagnetismo.
%\item  Magnetismo y superconductividad. Efecto Meissner.
%\end{itemize}

\item {\bf Campos electromagn\'{e}ticos dependientes del tiempo} \hfill (1.2 ECTS) 
\begin{itemize} \addtolength{\itemsep}{-0.25\baselineskip}
\noindent
\item Introducci\'{o}n.
\item  Flujo de un campo magn\'{e}tico.
\item  Ley de Faraday-Henry. Ley de Lenz. Aplicaciones.
\item  Autoinducci\'{o}n. Inducci\'{o}n mutua.
\item Energ\'{\i}a del campo electromagn\'{e}tico.
\item Oscilaciones electromagn\'{e}ticas: circuitos RLC. Analog\'{\i}a con el movimiento arm\'{o}nico. Resonancia.
\item Circuitos de corriente alterna.
\item Corriente de desplazamiento de Maxwell. Ley de Amp\`{e}re-Maxwell.
\item  Ecuaciones de Maxwell.
\end{itemize}


\item {\bf Ondas electromagn\'{e}ticas} \hfill (0.7 ECTS) 
\begin{itemize} \addtolength{\itemsep}{-0.25\baselineskip}
\noindent
\item  Obtenci\'{o}n de la ecuaci\'{o}n de ondas a partir de las ecuaciones
 de Maxwell. Velocidad de las ondas electromagn\'{e}ticas.
\item  Ondas electromagn\'{e}ticas planas. Car\'{a}cter transversal de las ondas.
 Relaciones entre los campos el\'{e}ctrico y magn\'{e}tico. Polarizaci\'{o}n.
\item  Energ\'{\i}a y momento de las ondas electromagn\'{e}ticas. Vector de Poynting.
\item  Fuentes de ondas electromagn\'{e}ticas.
\item  Propagaci\'{o}n de ondas electromagn\'{e}ticas en la materia: dispersi\'{o}n.
\item  Espectro de la radiaci\'{o}n electromagn\'{e}tica. 
%Efecto Doppler en ondas electromagn\'{e}ticas.
\end{itemize}



\item {\bf \'{O}ptica} \hfill (0.6 ECTS)
\begin{itemize} \addtolength{\itemsep}{-0.25\baselineskip}
\noindent
\item  Teor\'{\i}a ondulatoria y corpuscular.
\item  Principio de Huygens. Reflexi\'{o}n y refracci\'{o}n. Reflexi\'{o}n total.
% {\color{red} Experimento de Young de la doble rendija}.
\item  Aproximaci\'{o}n geom\'{e}trica. Rayo de luz. Principio de Fermat.
\item  Espejos y lentes. Diagramas de rayos. Combinaciones de lentes.
\item  Instrumentos \'{o}pticos: C\'{a}mara fotogr\'{a}fica. Microscopio. Telescopio. Ojo humano.
%\item  Aberraciones.
\end{itemize}

\end{enumerate}



\newpage
\subsubsection{\large Programa te\'{o}rico comentado}

En esta secci\'{o}n se comentan de forma detallada todos los
 cap\'{\i}tulos del programa de teor\'{\i}a,
 destacando los aspectos m\'{a}s importantes de los mismos.\\

\noindent {\bf  I - INTRODUCCI\'{O}N}
\begin{enumerate} [{\bf 1. }]


\item  {\bf  Naturaleza y m\'{e}todo de la F\'{\i}sica.}
Este cap\'{\i}tulo, de car\'{a}cter introductorio, comienza con una
 visi\'{o}n general e hist\'{o}rica de la F\'{\i}sica. Se establecen los
 objetivos que dicha ciencia persigue y los procedimientos utilizados para ello.
 En particular se hace hincapi\'{e} en el car\'{a}cter experimental de la F\'{\i}sica,
 y en la utilizaci\'{o}n del m\'{e}todo cient\'{\i}fico como procedimiento
 adecuado para el establecimiento de las leyes de la naturaleza. 


\item {\bf  Magnitudes F\'{\i}sicas y vectores.}
Se introducen los conceptos de magnitud, unidad,
 orden de magnitud y n\'{u}mero de cifras significativas, y se efect\'{u}a
 una revisi\'{o}n de los principales sistemas de unidades y las relaciones
 entre los mismos. A continuaci\'{o}n
se definen los vectores libres y las operaciones m\'{a}s importantes entre 
vectores y entre vectores y escalares, mostrando ejemplos sencillos de 
su aplicaci\'{o}n posterior. Se definen los distintos sistemas de 
coordenadas y se introduce el concepto de campo escalar y campo vectorial.
Se finaliza con la derivada y la integral de un vector respecto a un escalar.


\suspend{enumerate}
\noindent {\bf  II - MEC\'{A}NICA}

\resume{enumerate}[{[\bf 1.] }]
\item {\bf Cinem\'{a}tica del punto material.}
Con este cap\'{\i}tulo se inicia el estudio de la mec\'{a}nica. 
Se revisan los conceptos de part\'{\i}cula y sistema de referencia, y se estudia 
el movimiento desde un punto de vista puramente geom\'{e}trico, sin atender
a las causas que lo producen.
Se definen los conceptos de posici\'{o}n, velocidad,  aceleraci\'{o}n,
y su naturaleza vectorial,  y se formulan las ecuaciones 
generales del movimiento. A continuaci\'{o}n se estudian algunos ejemplos: movimiento
 en una dimensi\'{o}n con velocidad constante, con aceleraci\'{o}n constante y
 con aceleraci\'{o}n conocida dependiente del tiempo y se extiende el an\'{a}lisis a 
dos dimensiones: tiro parab\'{o}lico y movimiento circular.
Finalmente se presta atenci\'{o}n al movimiento curvil\'{\i}neo en el plano y 
su descripci\'{o}n en coordenadas polares, por su posterior aplicaci\'{o}n al estudio
del movimiento bajo fuerzas centrales, y en particular, al estudio del movimiento
planetario.


\item {\bf  Movimiento relativo.}
Se describe el movimiento relativo desde el punto de vista cinem\'{a}tico. Se 
obtienen las transformaciones que relacionan
los vectores de posici\'{o}n, velocidad y aceleraci\'{o}n de una part\'{\i}cula
con respecto a dos sistemas de referencia en movimiento relativo de
 traslaci\'{o}n (transformaciones de Galileo), movimiento relativo solo 
de rotaci\'{o}n o general (traslaci\'{o}n y rotaci\'{o}n). 
Mediante una serie de ejemplos, como el p\'{e}ndulo de Foucault,
 se resalta la importancia del movimiento relativo 
de rotaci\'{o}n. 


\item {\bf  Din\'{a}mica del punto material.}
En este cap\'{\i}tulo se introducen los principios y conceptos de la
 mec\'{a}nica newtoniana aplic\'{a}ndola en el caso m\'{a}s sencillo, el de una
 \'{u}nica part\'{\i}cula puntual. En primer lugar se enuncia la ley de la inercia
 y se definen los sistemas de referencia inerciales. A continuaci\'{o}n se
 introduce el concepto de masa inercial y se definen el momento lineal y la fuerza, 
a partir de los cuales se enuncian la segunda y tercera leyes de Newton.
 Como ejemplos de aplicaci\'{o}n se establecen las ecuaciones de movimiento
 para algunos sistemas con diferentes tipos de fuerzas: los casos m\'{a}s
 sencillos de fuerzas constantes a distancia y de contacto
 as\'{\i} como el de una fuerza variable, como por ejemplo el rozamiento en un fluido.
As\'{\i}, con el ejemplo de un paraca\'{\i}das se 
introduce el concepto de velocidad l\'{\i}mite. Seguidamente se define el 
momento angular y el momento de una fuerza para estudiar posteriormente el
 movimiento de una part\'{\i}cula sometida a una fuerza central y enunciar
 el teorema de conservaci\'{o}n del momento angular.
 Finalmente, mediante ejemplos concretos como el de una bombilla colgada en un tren, lanzamientos
desde una plataforma giratoria, etc... , se analiza el movimiento de una 
part\'{\i}cula desde un sistema de referencia no inercial,
 definiendo las denominadas fuerzas de inercia. 

\item {\bf Trabajo y energ\'{\i}a.}
Se definen el trabajo realizado por una fuerza sobre una part\'{\i}ícula y
 la energ\'{\i}a cin\'{e}tica, estableci\'{e}ndose la relaci\'{o}n entre ambos.
 En especial, se estudia el caso de las fuerzas conservativas, introduciendo
 los conceptos de energ\'{\i}a potencial y de energ\'{\i}a mec\'{a}nica para
 enunciar el principio de conservaci\'{o}n de la energ\'{\i}a mec\'{a}nica.
 Seguidamente se utilizan las nuevas magnitudes definidas para estudiar el 
movimiento de una part\'{\i}cula en casos particulares, desde el punto de vista
 de la conservaci\'{o}n de la energ\'{\i}a (movimiento bajo una fuerza constante en
una dimensi\'{o}n y movimiento bajo el efecto de una fuerza que cumple la Ley de Hooke).
 Se debe hacer especial hincapi\'{e} 
en que esta nueva forma de obtener la trayectoria de la part\'{\i}cula proviene 
directamente de las leyes de la din\'{a}mica y es, por tanto, equivalente a ellas.
 Finalmente se analiza el movimiento de la part\'{\i}cula a partir de curvas
 de energ\'{\i}a y se estudian ejemplos en los que intervienen fuerzas 
no conservativas.

\item {\bf Din\'{a}mica de los sistemas de part\'{\i}culas.}
En este cap\'{\i}tulo se extiende el an\'{a}lisis din\'{a}mico a sistemas
 compuestos por varias part\'{\i}culas puntuales. En primer lugar se hace 
la distinci\'{o}n entre fuerzas internas y externas del sistema y se define el 
centro de masa del mismo. A continuaci\'{o}n se enuncian y demuestran los teoremas 
de conservaci\'{o}n del momento lineal, del momento angular y de la energ\'{\i}a
 mec\'{a}nica. Se realiza el an\'{a}lisis desde el sistema de referencia 
del laboratorio y desde el sistema de referencia del centro de masa, 
se\~{n}alando la conveniencia de separar el movimiento respecto a cualquier
 sistema de referencia en dos partes: el del centro de masa y el movimiento propio 
del sistema (relativo a su propio centro de masa). 
Finalmente se estudian algunos casos de particular inter\'{e}s:
 el problema de dos cuerpos y su reducci\'{o}n al de un solo cuerpo previa
 definici\'{o}n de la masa reducida, 
las colisiones entre dos part\'{\i}culas tanto en una como en dos dimensiones
 y los sistemas de masa variable (cohetes).

\item {\bf  Din\'{a}mica del s\'{o}lido r\'{\i}gido.}
El s\'{o}lido r\'{\i}gido es un caso muy especial de sistema de part\'{\i}culas,
 por lo que es preferible estudiarlo en un cap\'{\i}tulo aparte. En el caso ideal,
 las distancias entre las part\'{\i}culas que componen el sistema permanecen 
constantes, lo que reduce el n\'{u}mero de grados de libertad, haci\'{e}ndose 
patente la necesidad de introducir nuevas t\'{e}cnicas en su estudio.
 En primer lugar se estudia la cinem\'{a}tica del s\'{o}lido r\'{\i}gido, para la 
cual se introduce el sistema de referencia propio, se se\~{n}alan los
 grados de libertad del sistema y se analiza el campo de velocidades. 
A continuaci\'{o}n se plantean las ecuaciones din\'{a}micas del sistema, 
se define el momento de inercia y  se calcula para algunas simetr\'{\i}as. 
Se aplica la f\'{o}rmula de Steiner para calcular los momentos de inercia 
respecto a ejes arbitrarios paralelos a los principales. 
Seguidamente se obtiene la expresi\'{o}n para la energ\'{\i}a cin\'{e}tica del
 s\'{o}lido, separ\'{a}ndola en dos partes: la energ\'{\i}a cin\'{e}tica de 
traslaci\'{o}n 
y la de rotaci\'{o}n, analizando el caso particular del movimiento de rodadura. 
Finalmente se aborda el problema de la est\'{a}tica, estableciendo las condiciones
 de equilibrio del s\'{o}lido r\'{\i}gido  aplic\'{a}ndolas a algunos ejemplos en dos
 y tres dimensiones.

\item {\bf  Interacci\'{o}n gravitatoria.}
Como caso particular de un sistema de dos cuerpos, en este tema analizaremos uno
 de los problemas m\'{a}s interesantes en la historia de la F\'{\i}sica:
 la interacci\'{o}n gravitatoria y el movimiento planetario. 
Se repasan brevemente los antecedentes hist\'{o}ricos que condujeron al 
enunciado de las leyes de Kepler y a la ley de la gravitaci\'{o}n universal.
 Se se\~{n}ala la diferencia conceptual entre masa inercial y masa gravitatoria,
 se enuncia el principio de equivalencia a partir del cual se introduce la
 constante de gravitaci\'{o}n universal y se describe el experimento 
 de Cavendish.
 Seguidamente se definen el campo gravitatorio y el potencial gravitatorio 
para describir la interacci\'{o}n a distancia, y se calculan los campos y
 potenciales creados por objetos de geometr\'{\i}a sencilla. Finalmente,
se estudia en detalle el problema de dos cuerpos con fuerzas centrales, se resuelve 
la ecuaci\'{o}n de movimiento, se obtiene la ecuaci\'{o}n de las \'{o}rbitas
y se estudia el movimiento de los planetas y sat\'{e}lites.


\item {\bf  Fluidos.}
En este tema se comienza con la clasificaci\'{o}n de los fluidos y sus propiedades.
 Se estudia la hidrost\'{a}tica, obteniendo la ecuaci\'{o}n fundamental de la 
hidrost\'{a}tica y la ley de Arqu\'{\i}medes. A continuaci\'{o}n se define 
el flujo de un campo vectorial para pasar a la parte de hidrodin\'{a}mica e 
introducir la ecuaci\'{o}n de continuidad y 
la ecuaci\'{o}n de Bernouilli para los fluidos ideales y sus aplicaciones.
Finalmente se trata la viscosidad y la ley de Poiseuille.

\item {\bf  Movimiento oscilatorio.}
Se introduce el concepto de oscilador, destacando su importancia como 
modelo en muchos problemas de f\'{\i}sica. En primer lugar se analiza el
 ejemplo de una part\'{\i}cula sujeta a un resorte lineal que satisface la
 ley de Hooke.
 Se plantea la ecuaci\'{o}n de movimiento y la de la energ\'{\i}a. 
A continuaci\'{o}n se
 presentan otros sistemas que poseen la misma ecuaci\'{o}n din\'{a}mica en primera 
aproximaci\'{o}n: p\'{e}ndulo simple, p\'{e}ndulo f\'{\i}sico, 
sistemas ligeramente desplazados
 de la situaci\'{o}n de equilibrio din\'{a}mico, etc... 
Se analiza la composici\'{o}n de movimientos arm\'{o}nicos en una y dos dimensiones.
 Seguidamente se introduce en el sistema una fuerza de amortiguamiento proporcional
 a la velocidad y se analizan los diferentes movimientos, en funci\'{o}n 
de la constante de amortiguamiento. 
Finalmente se a\~{n}ade al sistema una fuerza externa peri\'{o}dica para estudiar
 las oscilaciones forzadas y los fen\'{o}menos de resonancia.



\item {\bf  Movimiento ondulatorio.}
Para terminar el estudio de la Mec\'{a}nica se analiza el movimiento ondulatorio
 en medios materiales. El cap\'{\i}tulo comienza con la descripci\'{o}n
 matem\'{a}tica de la propagaci\'{o}n de una perturbaci\'{o}n sobre un medio en
 equilibrio din\'{a}mico, obteni\'{e}ndose la ecuaci\'{o}n diferencial que
 satisface una onda. Despu\'{e}s se analiza la propagaci\'{o}n en una dimensi\'{o}n,
 tanto de las ondas longitudinales en una barra met\'{a}lica como de las ondas
 transversales en una cuerda. Se obtienen las ecuaciones que satisfacen esos dos
 sistemas y se comparan con la ecuaci\'{o}n de ondas para obtener la velocidad 
de propagaci\'{o}n. Despu\'{e}s se introducen las ondas arm\'{o}nicas y se estudia 
la interferencia de las mismas. 
A continuaci\'{o}n se introducen las ondas estacionarias en una cuerda y en una
 columna de aire y finalmente se describe el efecto Doppler.



\suspend{enumerate}
\noindent {\bf  III - ELECTROMAGNETISMO}

\resume{enumerate}[{[\bf 1.] }]
\item {\bf Interacci\'{o}n electrost\'{a}tica.} 
Se comienza el estudio del electromagnetismo con una breve introducci\'{o}n
 hist\'{o}rica para llegar al hecho experimental de la existencia de la
 carga el\'{e}ctrica, y se enuncian los principios de conservaci\'{o}n y 
cuantificaci\'{o}n de la misma. Seguidamente se presenta la ley de Coulomb 
y se comparan cuantitativamente las interacciones gravitatoria y 
electrost\'{a}tica en el sistema prot\'{o}n-electr\'{o}n, con el fin de mostrar 
la importancia de la segunda frente a la primera. 
Una vez puesto de manifiesto el car\'{a}cter conservativo de la interacci\'{o}n 
se calcula el campo y el potencial electrost\'{a}ticos. 
Seguidamente se define el concepto de flujo del campo el\'{e}ctrico 
y se deduce la ley de Gauss, que se utilizar\'{a} para calcular los campos
 y potenciales debidos a distribuciones concretas de carga. Despu\'{e}s 
se calcular\'{a} su energ\'{\i}a electrost\'{a}tica. A continuaci\'{o}n se 
analiza el dipolo el\'{e}ctrico como un ejemplo muy importante de 
distribuci\'{o}n de cargas, se introduce el concepto de momento dipolar y se
 estudia su interacci\'{o}n con un campo electrost\'{a}tico externo.
Se realiza una clasificaci\'{o}n de los materiales en funci\'{o}n de su 
comportamiento en presencia de un campo electrost\'{a}tico.
Se finaliza con un breve an\'{a}lisis del comportamiento de los materiales 
diel\'{e}ctricos en presencia de un campo electrost\'{a}tico externo
introduciendo el concepto de polarizaci\'{o}n y densidad de carga de polarizaci\'{o}n. 


\item {\bf Materiales conductores en campos electrost\'{a}ticos.}
En este cap\'{\i}tulo se estudian los materiales conductores en equilibrio,
 su distribuci\'{o}n de cargas, y el campo y potencial electrost\'{a}ticos tanto 
en su interior como en el exterior. Se introduce el concepto de capacidad y se
 estudian diferentes tipos de condensadores y los diferentes tipos de 
combinaci\'{o}n de condensadores. A continuaci\'{o}n se obtiene la expresi\'{o}n 
de la energ\'{\i}a asociada al campo el\'{e}ctrico.
Para finalizar 
se estudian las consecuencias de introducir un diel\'{e}ctrico entre las placas de un
 condensador de caras planas. 


\item {\bf Corriente el\'{e}ctrica. Circuitos de corriente continua.}
Se analiza la conducci\'{o}n el\'{e}ctrica en r\'{e}gimen estacionario. 
En primer lugar se definen la intensidad el\'{e}ctrica y la densidad de
 corriente para obtener la ecuaci\'{o}n de continuidad a partir del principio
 de conservaci\'{o}n de la carga. A continuaci\'{o}n se define la conductividad
 de un material y se enuncia la ley de Ohm que relaciona el campo el\'{e}ctrico
 y la densidad de corriente en cada punto. Mediante el modelo de electrones
 libres de Drude se obtiene la resistividad de un material a partir de
 par\'{a}metros microsc\'{o}picos: masa y carga de los portadores, el tiempo
 medio entre colisiones y la densidad. Una vez definida la resistencia se introduce
 el concepto de fuerza electromotriz y se generaliza la ley de Ohm. Se aplica la
 ley de Ohm en alg\'{u}n circuito y se analiza la combinaci\'{o}n de resistencias.
 Seguidamente se estudia el efecto Joule y se deducen las leyes de Kirchhoff
 a partir de la ecuaci\'{o}n de continuidad y de la conservaci\'{o}n de la 
energ\'{\i}a. Finalmente se utilizar\'{a}n estas leyes para resolver algunos 
circuitos.




\item {\bf Interacci\'{o}n magn\'{e}tica.}
En este cap\'{\i}tulo se estudia el campo magn\'{e}tico en 
r\'{e}gimen estacionario. En primer lugar se introduce el concepto de 
campo magn\'{e}tico a partir de los efectos que produce sobre una carga
 en movimiento  sin mencionar los agentes que lo producen.
 Se introduce la expresi\'{o}n de la fuerza magn\'{e}tica y se analiza el 
movimiento de una carga en un campo magn\'{e}tico uniforme. 
Seguidamente se presentan algunos ejemplos de aplicaci\'{o}n de este fen\'{o}meno:
 selector de velocidades para un chorro de part\'{\i}culas cargadas,
 espectr\'{o}metro de masas, ciclotr\'{o}n, etc... 
Se extiende la expresi\'{o}n de la fuerza magn\'{e}tica al caso de corrientes
 el\'{e}ctricas, en particular sobre espiras planas, y se define el 
momento dipolar magn\'{e}tico. 
%Como ejemplo de efecto de un campo magn\'{e}tico sobre una corriente se 
%analiza el efecto Hall. 
A continuaci\'{o}n se analiza el campo creado por una carga en movimiento
y se extiende el an\'{a}lisis al campo
 creado por una corriente el\'{e}ctrica estacionaria introduciendo
la ley de Biot y Savart. 
 Despu\'{e}s se calcula la fuerza entre dos hilos infinitos y paralelos por 
los que circulan corrientes el\'{e}ctricas.
Se introduce la ley de Amp\`{e}re y 
junto con la ley de Biot y Savart se utilizan
 para determinar el campo creado por diferentes 
distribuciones de corrientes. En particular, se calcula el campo magn\'{e}tico
 creado por una espira circular a grandes distancias de su centro, 
y se compara su dependencia con  la del campo el\'{e}ctrico
 creado por un dipolo el\'{e}ctrico. 
Seguidamente se presenta la ley de Gauss del campo magn\'{e}tico y se
 discuten las diferencias con el caso el\'{e}ctrico.
%{\color{red} 
Para finalizar se hace una breve discusi\'{o}n del campo magnetost\'{a}tico 
en un medio material, comenzando por describir el comportamiento de la 
materia en presencia de un campo magn\'{e}tico externo. Se introducen los
 conceptos de imanaci\'{o}n y de densidad de corriente de imanaci\'{o}n.
 Seguidamente se generaliza la ley de Amp\`{e}re, definiendo el vector
 excitaci\'{o}n magn\'{e}tica, la susceptibilidad y la permeabilidad
 magn\'{e}ticas. A continuaci\'{o}n, mediante un sencillo modelo de electrones 
girando en \'{o}rbitas circulares alrededor de los n\'{u}cleos at\'{o}micos, 
se relaciona la imanaci\'{o}n de un material con sus caracter\'{\i}sticas
microsc\'{o}picas. A partir de este modelo se establecen las diferencias entre 
los materiales diamagn\'{e}ticos, paramagn\'{e}ticos y ferromagn\'{e}ticos.
%}


\item {\bf Campos electromagn\'{e}ticos dependientes del tiempo.}
En este cap\'{\i}tulo se analizan los campos electromagn\'{e}ticos dependientes 
del tiempo. Comienza con una descripci\'{o}n de los experimentos de Faraday y tras
introducir el concepto de flujo del campo magn\'{e}tico,
 se enuncian la ley de Faraday-Henry y la ley de Lenz. 
Se presentan algunas aplicaciones pr\'{a}cticas de dichas leyes como son el
 generador y el motor el\'{e}ctrico. A continuaci\'{o}n se estudian los 
fen\'{o}menos de inducci\'{o}n (autoinducci\'{o}n e inducci\'{o}n mutua) y, 
a modo de ejemplo, se calculan el coeficiente de autoinducci\'{o}n de un 
solenoide y el coeficiente de inducci\'{o}n mutua de dos solenoides. 
Despu\'{e}s se calcula la energ\'{\i}a asociada al campo magn\'{e}tico almacenada
 en una bobina para obtener a continuaci\'{o}n la densidad de energ\'{\i}a del
 campo magn\'{e}tico y la del campo electromagn\'{e}tico en general. 
Seguidamente se analizan los circuitos RLC, poniendo de manifiesto 
la equivalencia de su ecuaci\'{o}n fundamental y la del oscilador arm\'{o}nico 
forzado y con amortiguamiento. En el siguiente apartado se introduce la 
correcci\'{o}n llevada a cabo por Maxwell a la ley de Amp\`{e}re para 
hacerla compatible con la ley de conservaci\'{o}n de la carga en el caso no estacionario.
 Finalmente se presentan las cuatro ecuaciones de Maxwell, que constituyen las leyes fundamentales del electromagnetismo.


\item {\bf Ondas electromagn\'{e}ticas.}
Comienza el cap\'{\i}tulo obteniendo la ecuaci\'{o}n de ondas en el vac\'{\i}o,
  para el campo el\'{e}ctrico y para el campo magn\'{e}tico. En primer
 lugar se estudian las ondas planas, poniendo de manifiesto el car\'{a}cter 
transversal de las mismas, se establece la relaci\'{o}n entre los campos 
el\'{e}ctrico y magn\'{e}tico y se definen los estados de polarizaci\'{o}n.
 Se pone de manifiesto el hecho de que las ondas electromagn\'{e}ticas
 transportan momento lineal y energ\'{\i}a y se define el vector de Poynting.
 A continuaci\'{o}n se describen algunas fuentes de ondas electromagn\'{e}ticas:
 dipolo el\'{e}ctrico o magn\'{e}tico oscilante, carga acelerada, etc... 
Se realiza tambi\'{e}n un breve an\'{a}lisis de la propagaci\'{o}n de las ondas 
en la materia, relacionando la velocidad de propagaci\'{o}n con la permitividad y 
permeabilidad del medio, y se define el \'{\i}ndice de refracci\'{o}n.
 Finalmente se describe el espectro electromagn\'{e}tico.
%, y se presenta la ecuación del efecto Doppler para las ondas electromagnéticas. Esta expresión se justificará más adelante, en el capítulo dedicado a la relatividad especial



\item {\bf \'{O}ptica.}
Para terminar el bloque tem\'{a}tico dedicado al electromagnetismo se estudia la luz,
 haciendo hincapi\'{e} en que la \'{O}ptica es el estudio de los fen\'{o}menos 
asociados a una parte del espectro electromagn\'{e}tico descrito en el tema anterior.
 En la primera parte del cap\'{\i}tulo se enuncia el principio de Huygens y se
 obtienen las leyes de reflexi\'{o}n y refracci\'{o}n de la luz en una superficie
 plana que separa dos medios con \'{\i}ndice de refracci\'{o}n diferente.
 Seguidamente se pone de manifiesto la dependencia del \'{\i}ndice de refracci\'{o}n
 con la longitud de onda mediante el ejemplo de la dispersi\'{o}n en un prisma.
 La segunda parte del tema est\'{a} dedicada a la \'{O}ptica Geom\'{e}trica.
 Despues de establecer los l\'{\i}mites de validez de la aproximaci\'{o}n que 
supone  representar la propagaci\'{o}n de las ondas mediante rayos,
 se introduce 
el concepto de camino \'{o}ptico y se presenta el principio de Fermat.
 A continuaci\'{o}n comienza el estudio de los sistemas \'{o}pticos: espejos, lentes, 
combinaciones de lentes, y sus aplicaciones m\'{a}s inmediatas:  
Lupa, c\'{a}mara fotogr\'{a}fica, microscopio y el telescopio. Mediante algunos ejemplos 
sencillos se calcula la posici\'{o}n de la imagen a trav\'{e}s de los sistemas 
\'{o}pticos anteriores, su aumento lateral y sus características. 


\end{enumerate}


% Python
\subsubsection{\large Pr\'{a}cticas de ordenador con {\it python}}
\begin{enumerate}
\item Ejemplo. Movimiento bajo una fuerza constante
\item P\'{e}ndulo de Foucault
\item Oscilador arm\'{o}nico de dos dimensiones. Diagramas de Lissajous
\item Oscilador amortiguado
\item Oscilador forzado
\item Movimiento bajo una fuerza disipativa
\item Circuito RLC en serie
\item Circuito RC
\item Circuito RL
\item Circuito LC
\end{enumerate}


\newpage
% Practicas
\subsection{Programa de pr\'{a}cticas: T\'{e}cnicas Experimentales I}

\subsubsection{\large Introducci\'{o}n}


Como ya se ha se\~{n}alado anteriormente, 
el trabajo de laboratorio es fundamental en el proceso de 
formaci\'{o}n de un Graduado en F\'{\i}sica o Graduado en Ingenier\'{\i}a Electr\'{o}nica.
Es por ello que,
aunque la
asignatura   {\it  T\'{e}cnicas Experimentales I} no constituye
el objeto de este Proyecto Docente, incluyo aqu\'{\i} los contenidos
del programa de pr\'{a}cticas  como
parte complementaria de la asignatura {\it F\'{\i}sica General}.

 El trabajo de laboratorio implica una participaci\'{o}n directa del alumno 
en el proceso de aprendizaje y cubre dos aspectos esenciales:
 en primer lugar, la aplicaci\'{o}n de los diferentes conceptos
 aprendidos en el curso te\'{o}rico, facilitando una mejor comprensi\'{o}n
 de los mismos y, por otro lado, el desarrollo de habilidades necesarias 
para la labor experimental, adquiriendo experiencia en el manejo de
 instrumentos de medida.

El programa de {\it  T\'{e}cnicas Experimentales I} (6.0 ECTS) est\'{a}
formado por una introducci\'{o}n te\'{o}rico-pr\'{a}ctica  impartida
en el aula y por 10  pr\'{a}cticas de laboratorio que se realizar\'{a}n
en el laboratorio en sesiones de  cuatro horas de duraci\'{o}n.
 Adem\'{a}s, se pueden incluir algunas clases de laboratorio 
adicionales para la repetici\'{o}n de pr\'{a}cticas y correcci\'{o}n
 de los informes.

Asimismo, se hace entrega a los alumnos de un cuaderno de 
pr\'{a}cticas en el que se explica la forma de realizar cada una de 
ellas y se incluye adem\'{a}s una serie de normas para la entrega de los 
informes de las pr\'{a}cticas.


\subsubsection{\large Programa}

\noindent
\begin{enumerate} [{\bf I.}]\addtolength{\itemsep}{-0.25\baselineskip}
\item {\bf  Introducci\'{o}n} \hfill{(0.7 ECTS)}
 \begin{enumerate}[{1.}] \addtolength{\itemsep}{-0.25\baselineskip} 
 \item C\'{a}lculo  de errores y tratamiento de datos
 \item Manejo de programas de gr\'{a}ficos y tratamiento de datos
 \item Presentaci\'{o}n de informes
 \end{enumerate}
\item {\bf  Instrumentos de medida}\hfill{ (0.3 ECTS)}
\begin{enumerate}[{1.}] \addtolength{\itemsep}{-0.25\baselineskip} 
	\item Nonius y micr\'{o}metro
        \item Fuentes de alimentaci\'{o}n
        \item Osciloscopio
        \item Mult\'{\i}metro
	\item Componentes el\'{e}ctricos
\end{enumerate}
\item {\bf  Complementos te\'{o}ricos preparatorios} \hfill{ (1.0 ECTS)}
\begin{enumerate}[{1.}] \addtolength{\itemsep}{-0.25\baselineskip}
        \item Teor\'{\i}a de circuitos
\end{enumerate}
\item {\bf  Pr\'{a}cticas de Mec\'{a}nica, Electromagnetismo y \'{O}ptica}
\hfill{(4.0 ECTS)}
\noindent
\begin{enumerate} [{\bf 1. }]
%\item{\bf Ca\'{\i}da libre. Medida de g} \\
%{\color{red} IDATZI}

\item {\bf El p\'{e}ndulo f\'{\i}sico. Medida de g}.\\
Se investiga la relaci\'{o}n entre el per\'{\i}odo de 
oscilaci\'{o}n de una varilla delgada y la distancia del eje de 
oscilaci\'{o}n al centro de gravedad de la misma.
 Con estos datos se calcula el valor de la aceleraci\'{o}n de la gravedad. 
\item {\bf  Movimiento arm\'{o}nico simple. Ley de Hooke.}\\
Se analiza experimentalmente el movimiento
 peri\'{o}dico de una masa suspendida de un muelle.
 Se determina la dependencia entre el per\'{\i}odo de oscilaci\'{o}n
 y la masa suspendida, y se obtiene el valor de la constante
 el\'{a}stica del muelle. Este resultado  se compara con el 
valor deducido de la ley de Hooke, midiendo el alargamiento del 
muelle en funci\'{o}n de la masa colocada.
\item {\bf   Plano inclinado: oscilaciones. Muelles en serie y en paralelo.}\\
En primer lugar se determina la constante el\'{a}stica de un muelle.
 Posteriormente se estudia el acoplamiento de las constantes  
el\'{a}sticas de dos muelles seg\'{u}n est\'{e}n 
dispuestos en serie o en paralelo.
 Finalmente se mide la aceleraci\'{o}n de la gravedad utilizando 
un dispositivo experimental basado en el deslizamiento
 de un cuerpo en un plano inclinado.
\item {\bf Momento de inercia}\\
Se analiza el movimiento de un disco acoplado a un tambor que gira por 
el par  ejercido por una masa suspendida
de una cuerda enrollada en el mismo.
Midiendo el tiempo de ca\'{\i}da de la masa y su aceleraci\'{o}n
se determina el momento de inercia del disco.

%  Pendulo de torsion sustituida por momento de inercia del disco%%%%
%\item {\bf   P\'{e}ndulo de torsi\'{o}n. Momento de inercia.}\\
%Se construye un p\'{e}ndulo de torsi\'{o}n con un hilo met\'{a}lico
% y un tubo hueco en su parte inferior. Utilizando el m\'{e}todo de 
%Maxwell se determina la constante de torsi\'{o}n del hilo y 
%el momento de inercia de una barra cil\'{\i}ndrica.
%\item{\bf P\'{e}ndulo bal\'{\i}stico. Conservaci\'{o}n del momento lineal.} \\
%{\color{red} IDATZI}

\item {\bf   Medida de la velocidad del sonido. Tubo de resonancia.}\\
Utilizando el fen\'{o}meno de resonancia de las ondas sonoras en un 
tubo de longitud variable, se determina experimentalmente 
la velocidad del sonido en el aire a temperatura ambiente.

%\item {\bf Ley de Coulomb. Determinaci\'{o}n de la constante de coulomb}\\
%{\color{red} IDATZI}

\item {\bf   Corriente continua I. Resistencia interna de una fuente.}\\
Se aprende a manejar el mult\'{\i}metro y se verifican la ley de Ohm
y las leyes de Kirchoff para circuitos de corriente continua.
 Se miden las resistencias conectadas en distintas configuraciones
 y se estima
 la resistencia interna y la fuerza electromotriz de una fuente.
\item {\bf   Corriente continua II. Curva caracter\'{\i}stica de una 
l\'{a}mpara.}\\
Se discuten situaciones en las que la inserci\'{o}n de instrumentos
de medida en circuitos de corriente continua altera las caracter\'{\i}sticas
del circuito. A continuaci\'{o}n se mide la curva caracter\'{\i}stica de una 
l\'{a}mpara.
\item {\bf  Instrumentos de medida. Descarga de un condensador.}\\
Se profundiza en el manejo del mult\'{\i}metro y de las fuentes de 
alimentaci\'{o}n
realizando medidas de la descarga de un condensador.
Se estudian las caracter\'{\i}sticas de los circuitos con condensadores
y las soluciones de las ecuaciones diferenciales simples. 
 \item {\bf   Corriente alterna. Circuito RLC. Manejo del osciloscopio.}\\
Se estudia la corriente alterna en circuitos con resistencias, condensadores 
y bobinas. Se determina experimentalmente el desfase entre
 la intensidad de corriente y la fuerza electromotriz en un circuito RLC,
 as\'{\i} como la ca\'{\i}da de potencial en cada uno de sus componentes.
 Todas estas medidas se realizan utilizando el osciloscopio como
 instrumento de visualizaci\'{o}n y medida de se\~{n}ales.
 Asimismo, se estudia el fen\'{o}meno de la resonancia en circuitos RLC.
\item {\bf   Corriente inducida en  un solenoide. El transformador.}\\
Se observan los fen\'{o}menos de
 inducci\'{o}n electromagn\'{e}tica entre imanes y circuitos.
 En particular se estudia el transformador y se comprueba 
la ley de Lenz.

%\item {\bf \'{O}ptica geom\'{e}trica. Formaci\'{o}n de im\'{a}genes
%con espejos y lentes.}\\
%{\color{red} IDATZI}

\end{enumerate}



\end{enumerate}





