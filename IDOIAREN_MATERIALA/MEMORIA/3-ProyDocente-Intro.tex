% Proyecto Docente
\chapter{Introducci\'{o}n}
\label{PD-intro}



El Proyecto Docente que se presenta y defiende a continuaci\'{o}n constituye
 parte de la documentaci\'{o}n requerida   para el  concurso de acceso a la plaza 
vacante de los Cuerpos Docentes Universitarios con n\'{u}mero de 
referencia TUC8/1-D00063-13  correspondiente  al 
Cuerpo de Profesores Titulares  de Universidad en el \'{a}rea de
conocimiento de  F\'{\i}sica de 
la Materia Condensada, 
convocada por la Universidad del Pa\'{\i}s Vasco-Euskal Herriko Unibertsitatea
de acuerdo con la resoluci\'{o}n del 7 de marzo de 2012 y
 publicada en el BOE del 13 de abril de 2012.
 La  plaza est\'{a} adscrita al Departamento de F\'{\i}sica 
de la Materia Condensada de la Facultad de Ciencia y Tecnolog\'{\i}a,
siendo su perfil ling\"{u}\'{\i}stico biling\"{u}e (castellano-euskera), 
su r\'{e}gimen de dedicaci\'{o}n a tiempo completo y las actividades
docentes e investigadoras a realizar ``F\'{\i}sica General''.


\section{Marco legal}

La Universidad espa\~{n}ola se encuentra, hoy en d\'{\i}a,
 legalmente sujeta a la Ley Org\'{a}nica 4/2007, de 12 de abril publicada 
en el BOE n\'{u}mero 89 del 13 de abril y que modifica la anterior 
Ley Org\'{a}nica 6/2001 (LOU), publicada el 24 de diciembre de 2001. 
Esta \'{u}ltima entr\'{o} en vigor a partir del 13 de enero de 2002 
y derog\'{o} la anterior Ley de Reforma Universitaria (Ley Org\'{a}nica
 11/1983) 
de 25 de agosto publicada en el BOE de 1 de Septiembre de 1983 (LRU).

Las modificaciones m\'{a}s importantes introducidas por la LOU respecto a
 la LRU se centraron en tres aspectos principales:
 a) creaci\'{o}n de la Agencia Nacional de Evaluaci\'{o}n de la Calidad 
y Acreditaci\'{o}n (ANECA), b) la implantaci\'{o}n del Sistema 
de Habilitaci\'{o}n y c) la revisi\'{o}n de los \'{o}rganos de gobierno.


Asimismo, los acuerdos  en pol\'{\i}tica de educaci\'{o}n
 superior en Europa y el impulso que la Uni\'{o}n Europea pretende
 dar a la investigaci\'{o}n, ha hecho que la LOU haya tenido que ser 
modificada incorporando algunos elementos que apuntan hacia la mejora 
de la calidad de las universidades espa\~{n}olas,  la sustituci\'{o}n
 de las pruebas de hablitaci\'{o}n por un sistema de acreditaci\'{o}n
 y la elaboraci\'{o}n del estatuto del estudiante universitario, entre otros. 

En definitiva, la nueva reforma pretende dar el paso necesario para 
la organizaci\'{o}n del sistema universitario espa\~{n}ol hacia una
 estructura m\'{a}s abierta y flexible, que sit\'{u}e a las universidades
 espa\~{n}olas en una mejor posici\'{o}n para la cooperaci\'{o}n interna
 y la competencia internacional, a trav\'{e}s de la creaci\'{o}n,
 transmisi\'{o}n, desarrollo del conocimiento cient\'{\i}fico y 
tecnol\'{o}gico y de la transferencia de sus beneficios a la sociedad.



La Resoluci\'{o}n de 15 de septiembre de 2009, del Vicerrector de la  
Universidad del Pa\'{\i}s Vasco/Euskal Herriko Unibertsitatea, UPV/EHU	y publicada en el BOPV el 9 de octubre de 2009 regula los concursos de acceso a cuerpos de 
funcionarias y funcionarios docentes en el \'{a}mbito de la
 Universidad del Pa\'{\i}s Vasco/Euskal Herriko Unibertsitatea, UPV/EHU.



\newpage
\section{Condicionantes del Proyecto}
La elaboraci\'{o}n del proyecto docente de una asignatura est\'{a} sometida
 a una serie de condicionantes, que en mayor o menor medida, 
establecen y delimitan su contenido y en gran parte su desarrollo. 
Muchos de estos condicionantes son ajenos a la persona que realiza la
 programaci\'{o}n, aunque  tambi\'{e}n
 existen condicionantes subjetivos impuestos por la persona que 
elabora el proyecto.

La estructuraci\'{o}n de cualquier programa te\'{o}rico y pr\'{a}ctico
 de una asignatura depende no s\'{o}lo  de la disciplina cient\'{\i}fica
 en la que se encuadra, sino tambi\'{e}n de la configuraci\'{o}n del plan de
 estudios en el que \'{e}sta se inscribe, e, incluso, de su pertenencia a
 un determinado curso del plan. El proyecto docente aqu\'{\i} presentado 
intenta ajustarse, en la medida de lo posible, a una situaci\'{o}n real 
que se enmarca en el plan de estudios de la Universidad del Pa\'{\i}s Vasco, 
donde la concursante viene desarrollando su labor docente e investigadora.


En estos momentos, el sistema universitario se encuentra en una
 situaci\'{o}n de cambio marcado por dos factores, la reforma de la LOU
 y la necesidad de integrarse en el Espacio Europeo de Educaci\'{o}n Superior.
{\it Este cambio no es s\'{o}lo  estructural sino que adem\'{a}s impulsa
un cambio en las metodolog\'{\i}as docentes,
 que centra el objetivo en el proceso
de aprendizaje del estudiante, en un contexto que se extiende
ahora a lo largo de la vida}  (BOE N{$^\circ$}260, 30 octubre de 2007).
De hecho, la adaptaci\'{o}n de las universidades al EEES ha sido el 
detonante de la modificaci\'{o}n de la LOU por la Ley Org\'{a}nica 4/2007.
 El primer paso de esta reforma qued\'{o} establecido mediante la 
creaci\'{o}n de un registro de las nuevas titulaciones adaptadas al Espacio 
Europeo y la presentaci\'{o}n de las directrices a respetar por 
cada universidad a la hora de desarrollar sus planes de estudios.
Este nuevo plan de estudios (``Plan Bolonia'') se ha implantado  
en el  pasado curso 2010/2011.

Dada la reciente entrada en vigor del nuevo plan en el momento de redactar esta 
memoria, la concursante se ha planteado la necesidad de hacer referencia a sus 
principales directrices.
Esta es la finalidad del 
 apartado~\ref{PD-bolonia}. En este apartado 
tras una breve introducci\'{o}n para situar la asignatura
{\it F\'{\i}sica General} en el contexto de dicho plan,
se describen las competencias, la metodolog\'{\i}a y la evaluaci\'{o}n.
A continuaci\'{o}n se muestra un esquema del programa te\'{o}rico 
de los contenidos de la asignatura, 
que se describen detalladamente en la  secci\'{o}n 
siguiente. Por \'{u}ltimo, como complemento, se expone
 el programa de pr\'{a}cticas, que constituye parte de otra asignatura 
({\it  T\'{e}cnicas Experimentales I}) que no 
es objeto de este proyecto docente. 

Sin embargo  considero importante
exponerla junto con el programa de la asignatura te\'{o}rica por varias razones:
Por una parte, en los planes anteriores las pr\'{a}cticas de laboratorio formaban parte de la 
asignatura ''{\it F\'{\i}sica General}''.  Por otra,
 su contenido est\'{a} estrechamente 
relacionado con lo que se estudia en la asignatura te\'{o}rica 
y constituye un complemento necesario y fundamental  a la teor\'{\i}a 
y a los problemas para que los alumnos tengan la oportunidad de 
 aplicar en el laboratorio
lo que han aprendido en las clases magistrales.
Por \'{u}ltimo en la planificaci\'{o}n actual, y con el nombre de 
{\it  T\'{e}cnicas Experimentales I}, es una asignatura de segundo cuatrimestre
que  me ha sido asignada en el reparto de carga docente del departamento. Esto 
supone  para los alumnos  una gran ventaja,  desde el 
punto de vista pr\'{a}ctico y formativo,
ya que les permite afianzar sus conocimientos sobre la asignatura.



Asimismo 
se estima conveniente dedicar
 el apartado~\ref{PD-comp}
a describir las modificaciones que el nuevo plan de estudios adaptado 
al EEES ha introducido en la asignatura {\it  F\'{\i}sica General}
con respecto al plan 2000/2001, en lo que se refiere a competencias,
 metodolog\'{\i}a,  evaluaci\'{o}n, y  programa de la asignatura. 

Finalmente, concluyo con una  reflexi\'{o}n cr\'{\i}tica basada en
 mi experiencia personal de la docencia de esta asignatura durante el primer a\~no de su implantaci\'{o}n.


%\end{chapter}

