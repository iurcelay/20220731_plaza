% -----------------Cap-------------------
%\begin{chapter} {Modificaciones introducidas por el EEES respecto a los planes anteriores}
\chapter {Modificaciones introducidas por el EEES respecto a los planes anteriores}
\label{PD-comp}

La ense\~{n}anza de materias de car\'{a}cter cient\'{\i}fico y en particular
la F\'{\i}sica, por sus propias caracter\'{\i}sticas, no se 
reduce a una mera transmisi\'{o}n de conocimientos. 
El estudiante parte de hip\'{o}tesis iniciales, aplica leyes y principios
para resolver problemas nuevos que  deber\'{a} analizar
para ver si est\'{a}n de acuerdo con la experiencia.
As\'{\i} se fomenta el an\'{a}lisis cr\'{\i}tico y la capacidad de 
observaci\'{o}n.

Por eso en nuestro caso, los cambios que se han  producido con 
la implantaci\'{o}n del ``Plan Bolonia'' afectan m\'{a}s a la {\bf forma}
que al {\bf fondo}, puesto que ya ense\~{n}abamos a los
alumnos {\it a aprender}, 
y ya fomentabamos en el alumnado el {\it ser capaz de hacer} o
{\it saber hacer} en vez de s\'{o}lo {\it hacer}. 

He considerado que puede ser de inter\'{e}s hacer un breve resumen
de las modificaciones que introduce la filosof\'{\i}a del ``Plan Bolonia'',
 basado
en mi propia experiencia y en la compartida, dado que ya hace
 m\'{a}s de un curso 
que est\'{a} implantado en la Facultad de Ciencia y Tecnolog\'{\i}a de la UPV/EHU.

\begin{enumerate}[a)]
\item Modificaciones de fondo:

\begin{itemize}
\item De una docencia basada en la ense\~{n}anza a una docencia basada
en el aprendizaje.
\item Profundo cambio de mentalidad en los sujetos del proceso.
\item Potenciaci\'{o}n de la autonom\'{\i}a de los alumnos.
\item 
\begin{minipage}{0.35\linewidth}
{Redefinici\'{o}n de conceptos:}
\end{minipage}
\raisebox{-0.6cm}{
\begin{minipage} {0.38\linewidth} {
	\begin{itemize}
	\setlength{\itemsep}{0cm}%
        \setlength{\parskip}{0cm}%
	\item [$-$] Cr\'{e}ditos
	\item [$-$] Movilidad
	\item [$-$] Objetivos/Competencias
	\end{itemize}
}
\end{minipage}
}

\item 
\begin{minipage}{0.45\linewidth}
{Acu\~{n}aci\'{o}n de nuevos conceptos:}
\end{minipage}
\raisebox{-0.6cm}{
\begin{minipage} {0.30\linewidth} {
	\begin{itemize}
        \setlength{\itemsep}{0cm}%
        \setlength{\parskip}{0cm}%
        \item [$-$] Tarea
        \item [$-$] Carga de trabajo
        \item [$-$] Indicador
	\end{itemize}
}
\end{minipage}
}

\item 
\begin{minipage}{0.32\linewidth}
{Ampliaci\'{o}n de criterios:}
\end{minipage}
\raisebox{-0.6cm}{
\begin{minipage} {0.40\linewidth} {
	\begin{itemize}
        \setlength{\itemsep}{0cm}%
        \setlength{\parskip}{0cm}%
        \item [$-$] Formaci\'{o}n continua
        \item [$-$] Educaci\'{o}n social
        \item [$-$] Valoraci\'{o}n de actitudes
	\end{itemize}
}
\end{minipage}
}
\item Conocimientos: Los basados en la ense\~{n}anza caducan y los
basados en el aprendizaje no caducan.
\item 
\begin{minipage}{0.35\linewidth}
{Objetivos $\neq$ competencias:}
\end{minipage}
\raisebox{-0.6cm}{
\begin{minipage} {0.60\linewidth} {
        \begin{itemize}
	\setlength{\itemsep}{0cm}%
        \setlength{\parskip}{0cm}%
        \item [$-$] De conocimiento (valorados desde siempre)
        \item [$-$] Aptitudinales (valorados a veces)
        \item [$-$] Actitudinales (valorados s\'{o}lo desde ahora)
        \end{itemize}
}
\end{minipage}
}

\item Profesor: de actor principal y transmisor de conocimientos a asesor,
observador, gu\'{\i}a, etc.
\item Estudiantes: de receptores de concimiento a sujetos activos implicados en
el proceso de ense\~{n}anza.

\end{itemize}

\item
Todas estas innovaci\'{o}nes de  ``{fondo}''  se corresponden con otras 
muchas innovaciones  de  ``{forma}''  que influir\'{a}n directamente
en la manera de concebir la docencia y por tanto impartirla y evaluarla.

En cuanto a la metodolog\'{\i}a los cambios fundamentales son:
\begin{itemize}
\item Disminuci\'{o}n  del n\'{u}mero de alumnos para cada tipo de 
clase
\item Reducci\'{o}n de las tareas presenciales y potenciaci\'{o}n de las 
semipresenciales y a distancia.
\item
\begin{minipage} {0.18\linewidth}
{Se mantienen:}
\end{minipage}
\raisebox{-0.7cm}{
\begin{minipage} {0.80\linewidth} {
        \begin{itemize}
        \setlength{\itemsep}{0cm}%
        \setlength{\parskip}{0cm}%
        \item [$-$] Las clases te\'{o}ricas (m\'{a}ximo 60\%).
        \item [$-$] Las de problemas con el nombre de 
Pr\'{a}cticas de Aula (PA).
        \item [$-$] Las tutor\'{\i}as adquieren mayor relevancia
        \end{itemize}
}
\end{minipage}
}

\item Se incorporan los seminarios.
\item Se predetermina el momento y el tiempo de realizaci\'{o}n de las
clases te\'{o}ricas, PA y seminarios. Horarios menos flexibles.
\item Necesidad de coordinaci\'{o}n de los contenidos impartidos en 
 los distintos tipos de clase. Mayor inversi\'{o}n de tiempo.
\item Nuevo enfoque de actividades educativas. Renovaci\'{o}n de los
materiales de ense\~{n}anza.
\end{itemize}

Todo esto se refleja en cambios en el sistema de evaluaci\'{o}n:
 \begin{itemize}
\item Se transforma de final a continua.
\item Se convierte en instrumento que genera expectativas positivas
a los alumnos y que proporciona informaci\'{o}n:
        \begin{itemize}
	 \setlength{\itemsep}{0cm}%
        \setlength{\parskip}{0cm}%
        \item [$-$] al alumno de lo que debe mejorar
        \item [$-$] al profesor sobre los cambios a introducir para mejorar los m\'{e}todos
        \end{itemize}
\item No s\'{o}lo se valoran conceptos y aptitudes sino tambien actitudes.
\item Frente al examen final de tipo cl\'{a}sico, evaluaci\'{o}n continua mediante 
distintos indicadores e instrumentos de evaluaci\'{o}n.

\end{itemize}

\end{enumerate}

%\end{chapter}
