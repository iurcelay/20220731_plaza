
% -----------------Cap-------------------
%\begin{chapter} {El plan docente en el EEES: Bolonia}
\chapter {El plan docente en el EEES: Bolonia}
\label{PD-bolonia}
\section{Introducci\'{o}n}
La Declaraci\'on de la Sorbona, firmada por varios pa\'{\i}ses europeos en 
 1998, establece las primeras actuaciones encaminadas a la
 convergencia europea en materia universitaria. Los principios generales 
de esta declaraci\'on fueron respaldados y apoyados por 29 estados europeos 
con la firma de la Declaraci\'on de Bolonia en 1999, declaraci\'on que
 de hecho da nombre al proyecto de construcci\'on del Espacio Europeo de
 Educaci\'on Superior (EEES): ``Proceso de Bolonia''  o ``Plan Bolonia''.
 



La definici\'on del EEES implica una profunda reestructuraci\'on de la 
docencia universitaria en lo concerniente al dise\~no curricular, a 
las metodolog\'{\i}as de ense\~nanza y a los modelos de evaluaci\'on.
 Los objetivos b\'asicos del Espacio Europeo de Educaci\'on Superior son
 los siguientes:



\begin{itemize}
\item La adopci\'{o}n de un sistema de titulaciones flexible, f\'{a}cilmente
comprensible y comparable, que facilite la comprensi\'{o}n entre los 
diferentes sistemas educativos. Con este objetivo se ha implantado el 
denominado ``Suplemento al Diploma'' (Diploma Supplement), 
aportando una informaci\'{o}n detallada sobre los estudios cursados 
por el alumno  y describiendo las capacitaciones y habilidades adquiridas.
\item Un sistema basado en dos ciclos formativos principales: grado y 
postgrado.
\item El establecimiento de un sistema de cr\'{e}ditos com\'{u}n, el
  ECTS ({\it European Credit Transfer System} -- Sistema Europeo de 
Transferencia de Cr\'{e}ditos), para promover una mayor 
movilidad de los estudiantes.
\item La promoci\'{o}n de la movilidad: 
libre circulaci\'{o}n de estudiantes, profesores y personal administrativo 
de las universidades y otras instituciones de Educaci\'{o}n Superior europea.
\item El impulso  de la cooperaci\'{o}n europea para asegurar un 
nivel de calidad mediante el
desarrollo de criterios y metodolog\'{\i}as comparables.
\item Promoci\'{o}n del aprendizaje y evaluaci\'{o}n continuados.
\end{itemize}


En el sistema ECTS un cr\'{e}dito representa unas 25 horas de trabajo por
 parte del alumno, estableciendo la cantidad de 60 cr\'{e}ditos para un
 a\~no acad\'{e}mico y por tanto de 240 cr\'{e}ditos para el grado (4 a\~nos).

\section{La nueva estructura de los estudios universitarios}

El objetivo fundamental es adaptar los estudios universitarios al contexto 
europeo de tal manera que se armoniza la duraci\'{o}n de los estudios, 
los m\'{e}todos de aprendizaje y la evaluaci\'{o}n de las actividades 
acad\'{e}micas con la voluntad de promover la movilidad de los estudiantes
 y de los titulados y tituladas, posibilitando un sistema de reconocimiento y
 transferencia de cr\'{e}ditos.

El Real decreto 1393/2007, de 29 de octubre, desarrolla la estructura de las 
ense\~nanzas universitarias oficiales, de acuerdo con las premisas del
 EEES, y establece tres tipos de estudios conducentes a la obtenci\'{o}n
 de los correspondientes t\'{\i}tulos universitarios oficiales:
 grado, m\'{a}ster y doctorado.

As\'{\i} las ense\~{n}anzas universitarias conducentes a la 
obtenci\'{o}n de t\'{\i}tulos de car\'{a}cter oficial y con validez en 
todo el territorio nacional se estructuran en 2 niveles 
y tres ciclos: 

\begin{itemize}
\item {\bf Estudios de Grado}\\
El primer ciclo conduce a t\'{\i}tulos oficiales de Grado, que tienen por
 finalidad la obtenci\'{o}n de una formaci\'{o}n de car\'{a}cter general,
 principalmente orientada a la preparaci\'{o}n del alumno para el ejercicio de actividades de car\'{a}cter profesional, y
que suprime la antigua divisi\'{o}n  existente entre diplomaturas 
y licenciaturas con la intenci\'{o}n de facilitar un acceso m\'{a}s \'{a}gil
 y adecuado al mercado laboral y a la propia sociedad al proporcionar
 la mayor parte de las competencias profesionales necesarias. 
Los planes de estudios de las titulaciones de Grado tendr\'{a}n
 240 cr\'{e}ditos ECTS, repartidos en 4 cursos de 60~ECTS cada uno,
 que contendr\'{a}n toda la formaci\'{o}n
 te\'{o}rica y pr\'{a}ctica que el estudiante debe adquirir:
 aspectos b\'{a}sicos de la rama de conocimiento, materias obligatorias
u optativas, seminarios, pr\'{a}cticas externas, trabajos dirigidos,
trabajo de fin de Grado u otras actividades formativas.

\item{\bf Estudios de m\'{a}ster}\\
El Postgrado
comprende el segundo y tercer ciclo,  que conducen respectivamente
 a los t\'{\i}tulos 
oficiales de  M\'{a}ster (en clave de especializaci\'{o}n) y
t\'{\i}tulos Oficiales de  Doctor (en clave de investigaci\'{o}n).
La superaci\'{o}n de cada ciclo dar\'{a} lugar a la obtenci\'{o}n del
correspondiente t\'{\i}tulo.

Los estudios de m\'{a}ster tienen la finalidad de ofrecer una formaci\'{o}n
 m\'{a}s avanzada, orientada a la especializaci\'{o}n acad\'{e}mica y/o 
investigadora.
Los planes de estudios tendr\'{a}n una duraci\'{o}n de entre 60 y 120 
cr\'{e}ditos.


\item{\bf Estudios de doctorado}\\
Los estudios de doctorado constan de un periodo de formaci\'{o}n y de un periodo de investigaci\'{o}n que integran lo que se denomina
 ``programa de doctorado''.
Para acceder al periodo de formaci\'{o}n es 
necesario cumplir los mismos requisitos de acceso que para los estudios de 
m\'{a}ster.
Para acceder al periodo de investigaci\'{o}n es necesario estar en posesi\'{o}n de un t\'{\i}tulo oficial de m\'{a}ster u otro t\'{\i}tulo expedido por una instituci\'{o}n de educaci\'{o}n superior del EEES.

\end{itemize}

En el sistema ECTS, se entiende por {carga total de trabajo} del estudiante 
el n\'{u}mero total de horas de trabajo que dedica a las tareas que se 
le encomiendan para el logro de los objetivos de un programa. 
En las horas totales de trabajo que mide un cr\'{e}dito ECTS se incluyen,
por tanto, 
no s\'{o}lo las horas de aula, te\'{o}ricas y pr\'{a}cticas, sino 
tambi\'{e}n las horas de estudio, las horas dedicadas a la 
realizaci\'{o}n de seminarios, de trabajos individuales o en grupo, 
de pr\'{a}cticas o proyectos, a la resoluci\'{o}n de ejercicios, 
a la consulta de bibliograf\'{\i}a, las exigidas para preparar 
y realizar las pruebas de evaluaci\'{o}n, etc. 
 La gran diferencia con el antiguo sistema de c\'{o}mputo
 es que todas las actividades no presenciales
 que se realizaban 
para superar una asignatura, no se ten\'{\i}an  en cuenta a 
la hora de contabilizar cr\'{e}ditos.
 En el sistema anterior, el cr\'{e}dito representaba el n\'{u}mero 
de horas de clase que un profesor impart\'{\i}a. En concreto, un cr\'{e}dito
 correspond\'{\i}a a 10 horas lectivas (o 10 horas ``de clase''). 
El cr\'{e}dito europeo, sin embargo, mide el volumen o carga total 
del trabajo de aprendizaje del estudiante para alcanzar los objetivos 
previstos en el plan de estudios, y equivale a una carga de 
trabajo del estudiante de 25 horas.


La nueva organizaci\'{o}n de las ense\~{n}anzas universitarias
responde no s\'{o}lo a un cambio estructural sino
que adem\'{a}s impulsa un cambio en las metodolog\'{\i}as docentes
que centra el objetivo en el proceso de aprendizaje del
estudiante, y todo ello en un contexto que se extiende ahora a lo
largo de la vida.
Por lo tanto, no basta con reformar los planes de estudio anteriores
redimension\'{a}ndolos en base a los cr\'{e}ditos ECTS. Los
objetivos docentes y el modelo de aprendizaje exigen hacer una
transformaci\'{o}n profunda de los planes de estudio que hemos conocido
hasta ahora cuestionando las bases de su  sistema  y proponiendo
un sistema alternativo radicalmente distinto.
La tendencia actual asociada al EEES, es que vamos hacia una sociedad
 del aprendizaje. Esto supone el desplazamiento de una educaci\'{o}n centrada 
en la ense\~{n}anza hacia una educaci\'{o}n centrada en el aprendizaje.

En resumen, el ECTS es un {  sistema centrado en el estudiante}
y basado en la { carga de trabajo} que debe realizar para el logro de 
los objetivos que deben ser especificados, preferiblemente,
en t\'{e}rminos de {  resultados de aprendizaje (competencias)}
 que el estudiante deber\'{a} adquirir.


\subsection{ ``F\'{\i}sica General'' en la Facultad de Ciencia y Tecnolog\'{\i}a de la UPV/EHU.}


La F\'{\i}sica juega un papel importante en la sociedad, contribuyendo a 
la generaci\'{o}n del conocimiento y el desarrollo tecnol\'{o}gico.
 Una econom\'{\i}a competitiva  cobra fuerza mediante la
 innovaci\'{o}n generada a trav\'{e}s del conocimiento,
  gracias a la conjuncion de
la investigaci\'{o}n b\'{a}sica y de una labor m\'{a}s aplicada ligada al
 desarrollo tecnol\'{o}gico.


Probablemente lo m\'{a}s novedoso del  plan de estudios de la facultad
es que las asignaturas del primer a\~{n}o  son comunes
para los distinto Grados ofertados en la rama de Ciencias.
De hecho, la asignatura objeto de este proyecto docente es com\'{u}n al
Grado en F\'{\i}sica, Grado en Ingenier\'{\i}a Electr\'{o}nica y Grado en 
 Matem\'{a}ticas.
Principalmente va dirigida  a los Grados  que en cursos
superiores tienen asignaturas
estrechamente relacionadas con la F\'{\i}sica.
Es una adaptaci\'{o}n de {\it Fundamentos de F\'{\i}sica} del
plan anterior.

Se trata de una asignatura  anual de 12 cr\'{e}ditos ECTS, lo 
que supone  un curso anual con un promedio de 
 4 horas semanales de clase presencial para el alumno. 

Los {\bf grados} est\'{a}n estructurados en {\bf m\'{o}dulos}
 en los que se trabajan grupos 
m\'{a}s espec\'{\i}ficos de {\bf competencias}\footnote{Me referir\'{e}
 a las competencias con detalle en el apartado \ref{compet}.},
 o forma de expresi\'{o}n de
 los objetivos,
y se desarrollan {\bf destrezas} concretas.
En el caso del Grado en Matem\'{a}ticas la asignatura forma parte del m\'{o}dulo
 ``F\'{\i}sica'', en el Grado en F\'{\i}sica la asignatura forma parte del m\'{o}dulo
 ``Conceptos
B\'{a}sicos'' y en el caso del Grado en  
Ingenier\'{\i}a Electr\'{o}nica forma parte del m\'{o}dulo
``Fundamentos cient\'{\i}ficos para la ingenier\'{\i}a".


Las competencias de estos m\'{o}dulos (CM) son:
\begin{enumerate} [CM01:]
\item Adquirir los conocimientos necesarios para comprender con claridad los 
principios b\'{a}sicos de la F\'{\i}sica
Cl\'{a}ásica, la Qu\'{\i}mica y la Electr\'{o}nica b\'{a}sicas y sus aplicaciones
\item  Plantear correctamente y resolver problemas que involucren los principales conceptos de la F\'{\i}sica Cl\'{a}sica,
la Qu\'{\i}mica y la Electr\'{o}nica y sus aplicaciones
\item Documentarse y plantear de manera organizada temas relacionados con las 
materias del m\'{o}dulo para afianzar o
ampliar conocimientos y para discernir entre lo importante y lo accesorio
\item Exponer por escrito y oralmente problemas y cuestiones sobre F\'{\i}sica 
Cl\'{a}sica, Qu\'{\i}mica y Electr\'{o}nica, para
desarrollar destrezas en la comunicaci\'{o}n cient\'{\i}fica
\end{enumerate}


\newpage
\section{Competencias}\label{compet}
En el  contexto del EEES, los  objetivos se expresan como 
{ competencias} que se definen en t\'{e}rminos de {  capacidades}.
Una  competencia es la capacidad de desarrollar con \'{e}xito una tarea
en un determinado contexto. 


Las competencias se entienden como {  conocer} y {  comprender} 
(conocimientos te\'{o}ricos), saber c\'{o}mo {  actuar} (aplicaciones
pr\'{a}cticas de \'{e}stos) y saber c\'{o}mo { ser } (valores como
partes integrantes del contexto social).

A  la hora de establecer competencias, deberemos
hacerlo a tres niveles:
\begin{enumerate}
\item Competencias de contenidos: Para todos los conceptos susceptibles de ser aprendidos.
\item Aptitudes procedimentales: Para capacidades que condicionan el
potencial de aprendizaje.
\item Competencias actitudinales:  Para las que regulan la predisposici\'{o}n
para aprender.
\end{enumerate}


Las competencias por ser manifestaciones de la capacidad para realizar
tareas pueden ser {  verificadas} y {  evaluadas}, y adem\'{a}s
en funci\'{o}n del dominio relativo, podr\'{a}n tener distintos
{  niveles} de adquisici\'{o}n.

Los cambios fundamentales realizados sobre el binomio 
ense\~{n}anza-aprendizaje, centran su esfuerzo en el 
{  individuo que aprende} para hacerle capaz de {  manejar
el conocimiento}, actualizarlo y seleccionarlo para lo que deber\'{a} estar
en permanente contacto con las fuentes de informaci\'{o}n, y 
adem\'{a}s comprender lo aprendido para que pueda ser
adaptado a situaciones nuevas y r\'{a}pidamente cambiantes.

El cambio y la variedad de contextos requieren {  la exploraci\'{o}n 
constante de las demandas sociales},  que permitir\'{a}n dise\~{n}ar
los perfiles profesionales y acad\'{e}micos. Por otra parte, el 
lenguaje de las competencias es el m\'{a}s apropiado para la 
{  consulta y el di\'{a}logo} con los representantes de la sociedad
no directamente involucrados en la vida acad\'{e}mica.

Asimismo, la formaci\'{o}n para el empleo debe ir en paralelo a la 
concepci\'{o}n de una {  educaci\'{o}n  para la ciudadan\'{\i}a responsable}
que incluye adem\'{a}s de la necesidad de desarrollarse como persona,
ser capaz de afrontar responsabilidades sociales.


Los {  conocimientos} (adquiridos en una formaci\'{o}n basada en la 
ense\~{n}anza)  {  pueden ``caducar''},  pero
la capacitaci\'{o}n para la resoluci\'{o}n de problemas
 (adquirida en una formaci\'{o}n basada en el
aprendizaje)  no caduca.
En una sociedad que cambia con rapidez, los procesos instructivos deben
{  promover personas capaces}, es decir personas con  cierta
autonom\'{\i}a para abordar situaciones novedosas y responder adecuadamente
a las mismas.


Para organizar las competencias se utilizan distintos criterios seg\'{u}n
su { \'{a}mbito} sea de mayor a menor extensi\'{o}n.
\begin{itemize}
\item Competencia de {\bf  Titulaci\'{o}n}
\item Competencia de {\bf  M\'{o}dulo}
\item Competencia de {\bf  Asignatura}
\end{itemize}
Dentro de cada uno de estos \'{a}mbitos habr\'{a} dos tipos de competencias:
\begin{itemize}
\item Competencias {\bf  transversales} \'{o} gen\'{e}ricas
\item Competencias {\bf  no transversales} \'{o} {\bf  espec\'{\i}ficas} o
{\bf  t\'{e}cnicas}
\end{itemize}

Sobre todos estos tipos de competencias existen datos contrastados y ejemplos 
concretos en el Libro Blanco del T\'{\i}tulo de Grado en F\'{\i}sica y
en el Proyecto Tuning.
De todos estos documentos se puede inferir que no todas las competencias
tienen la misma {  importancia} y que \'{e}sta viene determinada
por los diferentes perfiles profesionales.


El proponer pocas competencias obliga a un tipo
 de redacci\'{o}n
muy general y a veces poco significativa; el proponer muchas complica
la gesti\'{o}n de la propuesta curricular por la necesidad de buscar
tareas para desarrollarlas y luego evaluarlas. 
Hay que tener muy claro antes de abordar el curriculum de la 
materia cu\'{a}l ser\'{a} el n\'{u}mero de competencias y el tipo de 
redacci\'{o}n mediante una {  clarificaci\'{o}n terminol\'{o}gica}
y una {  correcci\'{o}n formal} de las mismas.

Una competencia desde el punto de vista de su estructura est\'{a} formada
por: 
\begin{enumerate} [{\bf a) }]
\item {\bf   Una operaci\'{o}n} materializada en un verbo de acci\'{o}n,
utilizando los muy {\bf  generales} como {\it comprender}, {\it aprender}
y {\it saber} para competencias generales, mientras que aquellos que 
definan {\bf  aprendizajes f\'{a}cilmente reconocibles y evaluables por 
tareas espec\'{\i}ficas} como {\it analizar}, {\it valorar}, {\it definir}, 
{\it resolver}, etc se utilizar\'{a}n para competencias espec\'{\i}ficas.
\item {\bf  Una regulaci\'{o}n} (no siempre), materializada por un 
adverbio que matiza la acci\'{o}n: {\it cr\'{\i}ticamente}, 
{\it minuciosamente}, {\it con precisi\'{o}n}, etc.
\item {\bf  El objeto} sobre el que recae la acci\'{o}n
\item {\bf Un fin} determinado, siempre que sea posible, que indica el 
``para qu\'{e}'' de la acci\'{o}n.
\end{enumerate}

Las competencias de la Titulaci\'{o}n pueden tener como referencia
m\'{a}s de un contexto de aplicaci\'{o}n simult\'{a}nea en funci\'{o}n del
\'{a}mbito de desarrollo del estudiante:
\begin{itemize}
\item Acad\'{e}mico: el contexto de uso hace referencia a la
propia universidad.
\item Preprofesional: el contexto de uso hace referencia al futuro marco
profesional previsible.
\item Social:  el contexto de uso hace referencia al entorno social.
\end{itemize}



Por el lugar en el que se enmarca la asignatura en el Grado, 
esta debe ser
 una introducci\'{o}n al estudio de la F\'{\i}sica
 que  proporcione una visi\'{o}n 
global de lo que el alumno estudiar\'{a} en los  cursos siguientes.
 Adem\'{a}s, algunos de los contenidos que se proponen
 en el programa ya han sido 
estudiados en cursos anteriores y todos ellos se desarrollar\'{a}n en 
profundidad en los cursos superiores del Grado. 


%El objetivo general que se persigue con el Grado en F\'{\i}sica
% es proporcionar una formaci\'{o}n cient\'{\i}fica adecuada en 
%los aspectos b\'{a}sicos y aplicados de la F\'{\i}sica,
% as\'{\i} como desarrollar el esp\'{\i}ritu cient\'{\i}fico del alumno. 
%Ello se materializa en la adquisici\'{o}n  de un sentido f\'{\i}sico, 
%entendido \'{e}ste como capacidad de observaci\'{o}n de los fen\'{o}ómenos
% que ocurren en la Naturaleza y el laboratorio, adquiriendo as\'{\i} 
%una comprensi\'{o}ón de los fen\'{o}ómenos que llevar\'{a}
% a la identificaci\'{o}n de las magnitudes que en ellos intervienen.
% Asimismo, se intenta que el alumno adquiera una metodolog\'{\i}a
% f\'{\i}sica mediante el conocimiento del l\'{e}éxico o vocabulario 
%espec\'{\i}ífico, aprendiendo as\'{\i} a formular las leyes
% f\'{\i}sicas b\'{a}sicas y las condiciones de validez de las mismas.
%

La asignatura 
 {\it F\'{\i}sica General}   pretende ofrecer al 
alumno una presentaci\'{o}n l\'{o}gica y unificada de la F\'{\i}sica 
a nivel introductorio, haciendo \'{e}nfasis  en las ideas b\'{a}ásicas
 que constituyen el fundamento de la F\'{\i}sica, introducir al 
estudiante en el m\'{e}todo cient\'{\i}fico para que aprenda
 a razonar de acuerdo con \'{e}l y consiga diferenciar  claramente
 lo que es un principio de lo que es una ley o una teor\'{\i}a, 
 formular las leyes b\'{a}sicas
 conociendo las hip\'{o}tesis de partida, expresarlas con el 
l\'{e}xico adecuado, extraer las consecuencias
 m\'{a}s inmediatas de dichas leyes y  analizar si est\'{a}n de acuerdo 
con la experiencia, o idear mecanismos sencillos de comprobaci\'{o}n. 


A lo largo del curso se debe poner especial \'{e}nfasis en
 se\~{n}alar que la F\'{\i}sica es fundamentalmente una Ciencia Experimental.
 Su objetivo es la descripci\'{o}n de los fen\'{o}menos naturales
 y s\'{o}lamente de ellos se puede 
extraer la informaci\'{o}n necesaria, es decir, que la observaci\'{o}n y
la experimentaci\'{o}n son las bases de nuestro conocimiento f\'{\i}sico.
Por la misma raz\'{o}n se intentar\'{a} despertar o mantener en 
los alumnos una actitud de curiosidad cient\'{\i}fica que les impulse
a profundizar en el conocimiento de la Naturaleza y a desarrollar 
su capacidad cr\'{\i}tica.


Otro aspecto importante es fomentar la utilizaci\'{o}n de libros, revistas
cient\'{\i}ficas y de
 otras fuentes 
 de informaci\'{o}n, como la documentaci\'{o}n {\it on-line} a
 trav\'{e}s de internet. 
Generalmente, los alumnos tienden a conformarse con las explicaciones dadas 
en clase por el profesor, sin embargo es muy importante que el estudiante
 indague    y busque informaci\'{o}n
 en la bibliograf\'{\i}a propuesta y en internet,
 no s\'{o}lo para que conozca diferentes puntos de vista sobre los 
temas de estudio 
 y de ese modo aumente su conocimiento, sino tambi\'{e}n para que adquiera 
 el h\'{a}bito de la consulta y de la documentaci\'{o}n.


Como competencias espec\'{\i}ficas de
 la asignatura (CA) de {\it  F\'{\i}sica General} 
se  enumeran los siguientes:
\begin{enumerate}[{CA}1:]
\item Manejar las magnitudes f\'{\i}sicas, distinguiendo entre magnitudes
 escalares y vectoriales, y asimilar conceptos como el de orden
 de magnitud y empezar a utilizar
 las aproximaciones como herramienta imprescindible en muchos campos.

\item Ser capaces de interpretar
 las leyes y principios b\'{a}sicos de la F\'{\i}sica,
 esenciales para comprender la naturaleza de los 
fen\'{o}menos f\'{\i}sicos.

\item  Relacionar las leyes y principios de la F\'{\i}sica, 
aplic\'{a}ndolos a los diferentes problemas que se plantean.

\item Desarrollar  determinadas t\'{e}cnicas de resoluci\'{o}n de problemas,
  ejercit\'{a}ndose de
 este modo en la valoraci\'{o}n de los resultados obtenidos.

%\item Adquirir las destrezas b\'{a}sicas en el manejo del 
%instrumental del laboratorio, conociendo su 
%fundamento te\'{o}rico, y desarrollar la capacidad de observaci\'{o}n 
%y  cr\'{\i}tica  mediante la interpretaci\'{o}n
%de los resultados obtenidos.

\item Establecer relaciones abiertas y comunicativas entre
 los estudiantes y el profesor, de modo que el 
estudiante reflexione y discuta las ideas y conocimientos adquiridos,
 tanto con los dem\'{a}s estudiantes,
como con el profesor.

\item Adoptar una actitud favorable hacia el aprendizaje de la 
asignatura mostr\'{a}ndose proactivo, participativo y con esp\'{\i}ritu 
de superaci\'{o}n ante las dificultades del aprendizaje.

\end{enumerate}




\newpage
\section{Metodolog\'{\i}a}

La labor docente no debe consistir \'{u}nicamente en la transmisi\'{o}n 
de conocimientos propios de la asignatura, sino tambi\'{e}n en el 
desarrollo de la capacidad cr\'{\i}tica y 
 adquisici\'{o}n de una metodolog\'{\i}a cient\'{\i}fica por parte
del alumno.
 Estas cualidades fundamentales en todo cient\'{\i}fico deben ser 
fomentadas por el profesor mediante el uso de una metodolog\'{\i}a adecuada.
El mayor o menor \'{e}xito obtenido por diferentes m\'{e}todos did\'{a}cticos 
depende, evidentemente, del profesorado, pero tambi\'{e}n de los medios 
y condiciones de los que se dispone, el n\'{u}mero de alumnos por clase, 
 la preparaci\'{o}n previa de los alumnos y los preconceptos
que \'{e}stos poseen sobre la asignatura.
A la hora de presentar la metodolog\'{\i}a  en una asignatura 
universitaria es necesario tener en cuenta esos factores 
para elaborar un m\'{e}todo docente pr\'{a}ctico y aplicable. 

Es precisamente aqu\'{\i} donde la innovaci\'{o}n que supone la aplicaci\'{o}n
de la ``filosof\'{\i}a'' propuesta en el ``Plan Bolonia'' va a tener
un mayor reflejo, por ello, antes de proceder a  definir el contenido de la
 asignatura {\it  F\'{\i}sica General}, es decir, antes 
de concretar {\it qu\'{e} se va a ense\~{n}ar} es conveniente realizar una 
reflexi\'{o}n sobre algunos aspectos importantes como  {\it a qui\'{e}n} se va 
a ense\~{n}ar, {\it qui\'{e}n va a transmitir} los conocimientos,
{\it c\'{o}mo se van a transmitir} \'{e}stos, y 
{\it c\'{o}mo se va a evaluar} el rendimiento y la capacitaci\'{o}n de los 
estudiantes. 
\'{E}ste es el objetivo de  esta secci\'{o}n y la siguiente.



\subsection{Sobre el alumnado}
El alumnado es un aspecto muy importante a la hora de
dise\~{n}ar el proyecto docente. Es de poca utilidad dise\~{n}ar un 
proyecto docente extraordinariamente ambicioso si eso supone
que s\'{o}lo un porcentaje m\'\i nimo de  alumnos puede superar la
asignatura. Por ello, conviene tener en cuenta el tipo de alumnado
que se va a tener enfrente a la hora de ``salir a la pizarra''.

En el caso que nos ocupa, es una asignatura que se imparte en el
primer curso del
Grado en F\'{\i}sica, en el Grado en Ingenier\'{\i}a Electr\'{o}nica
y en el Grado en Matem\'{a}ticas. El alumnado procede de distintos
centros de Ense\~{n}anza Secundaria con diferentes grados de exigencia
 y aunque se supone que todos 
los alumnos y alumnas deber\'{\i}an haber cursado las asignaturas de F\'{\i}sica y de 
Matem\'{a}ticas, bien del itinerario Ciencias de la Naturaleza 
y de la Salud, 
o del itinerario  Tecnol\'{o}gico eso no sucede siempre. Seg\'{u}n mi propia 
experiencia en algunos casos no han cursado la asignatura de F\'{\i}sica
en bachiller, especialmente los que eligen el Grado en Matem\'{a}ticas. Esto se
refleja tambi\'{e}n en la distinta motivaci\'{o}n que presentan frente a la 
asignatura de {\it F\'{\i}sica General} los estudiantes
matriculados en los distintos grados; as\'{\i} var\'{\i}a mucho entre
estudiantes del Grado en Matem\'{a}ticas y  del Grado en F\'{\i}sica, mientras
que tanto la exigencia de horarios como de contenidos son los mismos en ambos grados. Esto es un factor m\'{a}s a tener  en cuenta por parte del profesor a la hora de impartir
esta asignatura con alumnos de los distintos grados.


Seg\'{u}n el EEES la incorporaci\'{o}n del cr\'{e}dito
 ECTS debe conllevar un cambio en 
la actitud del estudiante, {  que deja de ser mero receptor}
 de conocimientos
(docencia basada en la {  ense\~{n}anza}),
 para pasar a asumir una {  actitud activa y aut\'{o}noma}
 con relaci\'{o}n a las actividades planificadas
que ha de realizar (docencia basada en el {  aprendizaje}).
El nuevo sistema 
exige al alumno m\'{a}s protagonismo y cuotas m\'{a}s altas de 
compromiso.

Por ello, es necesario reducir la utilizaci\'{o}n de tareas presenciales
y potenciar las {  tareas semipresenciales} y { a distancia} con 
el objeto de ense\~{n}ar a aprender para que el estudiante  pueda 
``aprender a aprender'' concibiendo la educaci\'{o}n universitaria como una 
etapa m\'{a}s del aprendizaje a lo largo de la vida. Esto nos
conduce al concepto de {  formaci\'{o}n continua} mediante la cu\'{a}l
los individuos son capaces no s\'{o}lo de manejar el conocimiento, 
actualizarlo,  seleccionar lo que es adecuado para un contexto determinado,
sino tambi\'{e}n {  comprender lo aprendido } para adaptarlo a nuevas 
situaciones.


\subsection{Sobre el profesor}
La labor docente comprende b\'{a}sicamente las siguientes actividades: 
 clases te\'{o}ricas,  pr\'{a}cticas de aula, tutor\'{\i}as, seminarios, 
la comunicaci\'{o}n {\it on-line} y pr\'{a}cticas de laboratorio. 
Si todas estas actividades no son impartidas por el mismo profesor es
 importante que exista una estrecha { coordinaci\'{o}n}
 entre los profesores, 
tanto al comienzo del curso para la planificaci\'{o}n de la asignatura, 
como a lo largo del mismo para intercambiar opiniones sobre la manera en que
 se desarrolla. Asimismo, dada la afinidad entre ciertas materias del 
programa y otras asignaturas que se imparten en cursos posteriores,
 es conveniente { delimitar los contenidos} de los temas de modo que no
 se produzcan solapamientos o lagunas sino que, por el contrario,
 el distinto modo de enfocarlos resulte enriquecedor para el alumno.

Al comenzar el curso es importante que el profesor presente una 
visi\'{o}n general de la asignatura, los objetivos propuestos y sus 
campos de aplicaci\'{o}n. Para ello es necesario exponer el programa 
y comentarlo extensamente. Asimismo, es preciso proporcionar una 
bibliograf\'{\i}a apropiada que sirva de referencia y de apoyo al alumno. 
Esta bibliograf\'{\i}a debe ser f\'{a}cilmente accesible y adecuada al 
nivel del curso, debe contener un n\'{u}mero razonable de textos 
 te\'{o}ricos y de problemas y, a ser posible, presentar una notaci\'{o}ón
 homog\'{e}nea. Afortunadamente, existe una amplia y variada bibliograf\'{\i}a
 sobre F\'{\i}sica b\'{a}sica para los primeros
 cursos, y los estudiantes disponen de suficientes 
ejemplares en las bibliotecas de la universidad. 
Es importante insistir con frecuencia sobre
 la importancia de consultar los 
libros de texto para obtener diferentes puntos de vista de un mismo problema. 

Uno de los mayores cambios introducidos por
el sistema ECTS es la forma de desarrollar la labor docente del profesor.
Su labor fundamental  es la de { ense\~{n}ar a aprender}.
No se limita s\'{o}lo a transmitir conocimientos, sino que ha de organizar 
tareas, seminarios, evaluaciones continuas y ex\'{a}menes,
para fomentar en el estudiante la adquisici\'{o}n de conocimientos,
capacidades y destrezas que le permitan responder adecuadamente a las futuras
demandas de su desempe\~{n}o profesional y progresar humana y
acad\'{e}micamente.
El profesor debe realizar el esfuerzo necesario
para adaptarse a la nueva metodolog\'{\i}a docente,
 al mismo tiempo que estimula el protagonismo  del estudiante.

Si bien el  profesor contin\'{u}a siendo fundamental, su papel
{  se desplaza cada vez m\'{a}s hacia el de un consejero, orientador
y motivador}.
El profesor ha pasado de ser el protagonista
principal de la ense\~{n}anza, supervisor y evaluador de trabajos
y conocimientos, a ser un acompa\~{n}ante en el proceso de
aprendizaje, que ayuda al estudiante a alcanzar ciertas competencias.
En todo caso, en la pr\'{a}ctica, la nueva metodolog\'{\i}a
le va a  suponer una mayor inversi\'{o}n de tiempo para conseguir
los objetivos. 


\subsection{El m\'{e}todo docente}
Los condicionamientos que suponen la p\'{e}rdida de protagonismo del 
profesor frente a un alumno activo y participativo, conllevan la necesidad
de cambios dr\'{a}sticos en cuanto al enfoque de las {  actividades 
formativas} as\'{\i} como a los {  materiales de ense\~{n}anza}.
Esto dar\'{a} lugar a una gran variedad de ``situaciones did\'{a}cticas''
que estimulen el compromiso del estudiante.


En el desarrollo de una metodolog\'{\i}a bajo
la perspectiva del cr\'{e}dito ECTS, se define el 
concepto de  {\bf  tarea} como la {  propuesta de trabajo} que un docente 
realiza a un estudiante para organizar un proceso de ense\~{n}anza.
Las tareas deben estar 
estrechamente relacionadas con las  
competencias existiendo una relaci\'{o}n expl\'{\i}cita e intencionada
entre las tareas propuestas y las competencias a lograr.
El conjunto de las tareas propuestas debe abarcar todas las competencias
a conseguir, es decir, para todas las competencias propuestas han de 
programarse sus correspondientes tareas. 


Respecto a los recursos pedag\'{o}gicos en el sistema ECTS 
siguen  existiendo las {\bf   clases te\'{o}ricas} y las 
clases de problemas, pasan a denominarse {\bf  pr\'{a}cticas de aula}. 
Se mantiene el sistema de {\bf   tutor\'{\i}as}.  La
novedad es que se incorporan {\bf seminarios} para 
discutir temas m\'{a}s concretos en grupos m\'{a}s reducidos.  

Se establece un n\'{u}mero m\'{a}ximo de alumnos para cada uno de estos tipos
de
``clases''. As\'{\i} el n\'{u}mero m\'{a}ximo de alumnos en las clases
te\'{o}ricas es de 100, en las pr\'{a}cticas de aula 40 y en los
seminarios de 15 a 20. En caso de superar estos cupos, los grupos
se pueden  desdoblar en grupos m\'{a}s peque\~{n}os  siempre que haya recursos.


A diferencia con los planes anteriores, en los que el profesor
pod\'{\i}a distribuir las clases te\'{o}ricas y de problemas
seg\'{u}n el desarrollo del temario
y el ritmo de la clase, en los nuevos Grados viene determinado
{   cu\'{a}ndo} deben impartirse las clases te\'{o}ricas,
las pr\'{a}cticas de aula, o los seminarios.
Por esta raz\'{o}n los horarios deben realizarse con especial
cuidado para que todos los alumnos hayan visto la misma cantidad de
materia al cabo de cierto per\'{\i}odo de
 tiempo, por ejemplo al finalizar la semana,  y as\'{\i} evitar desfases
entre los distintos grupos.

\subsubsection{Clases te\'{o}ricas}

El sistema ECTS determina que las
 clases te\'{o}ricas abarcan como m\'{a}ximo el 60\%  del total
de las horas lectivas.

En el planteamiento de las clases te\'{o}ricas el profesor debe 
asegurarse de que los estudiantes sean capaces de seguir en todo momento 
sus explicaciones. Por esta raz\'{o}n, la exposici\'{o}n de los temas 
debe ser adecuada al nivel del curso y la construcci\'{o}n de las 
explicaciones debe estar claramente estructurada. 
Antes del comienzo de una explicaci\'{o}n debe quedar claro cu\'{a}les
 son {  los objetivos} de \'{e}sta y cu\'{a}les son las hip\'{o}tesis de partida. 
Asimismo, para fomentar el inter\'{e}s del alumno por el tema, 
 es muy importante presentar hechos experimentales
 concretos en los que  sea necesaria la
 aplicaci\'{o}n de los conceptos tratados. 

El desarrollo de cualquier asignatura de ciencias, y en particular  
la F\'{\i}sica, puede dar lugar a largos desarrollos y demostraciones 
matem\'{a}ticas. Este hecho puede f\'{a}cilmente distraer al alumno del 
principal objetivo de la demostraci\'{o}n, centr\'{a}ndo excesivamente 
su atenci\'{o}n en la comprensi\'{o}n de los pasos matem\'{a}ticos 
que se llevan a cabo en ella. Para evitar esto, en la medida de lo posible, 
el profesor debe hacer hincapi\'{e} en {  el fin que se persigue},
 as\'{\i} como en la interpretaci\'{o}n f\'{\i}sica de los resultados obtenidos.
 No obstante, el alumno tambi\'{e}n debe familiarizarse con las
 t\'{e}cnicas matem\'{a}ticas y ser capaz de llevar a cabo los desarrollos 
matem\'{a}ticos
m\'{a}s importantes que vayan surgiendo. Despu\'{e}s de cada explicaci\'{o}n
 te\'{o}rica es conveniente { resumir} lo expuesto, poniendo especial inter\'{e}s
 en recalcar las hip\'{o}tesis de partida, las aproximaciones realizadas,
 el significado de las conclusiones obtenidas y su aplicabilidad. 
Finalmente, es de gran ayuda presentar varios {  ejemplos} de aplicaci\'{o}n
 inmediata de los resultados y, si es posible, relacionarlos con alg\'{u}n
 fen\'{o}meno conocido para el alumno.

Las demostraciones pr\'{a}cticas o experiencias de c\'{a}tedra de
 montaje sencillo y gran valor ilustrativo son ideales para la 
introducci\'{o}n de ciertos temas te\'{o}ricos,
 a la vez que sirven para hacer \'{e}nfasis en el car\'{a}cter
 experimental de la F\'{\i}sica e imprimen cierto dinamismo a las 
clases de teor\'{\i}a.

Un aspecto primordial a tener en cuenta durante el transcurso de las 
clases te\'{o}ricas es la asimilaci\'{o}n de los contenidos por parte 
del alumno. Para comprobar el grado de captaci\'{o}n de los conceptos 
 es \'{u}til plantear cuestiones durante el transcurso de las 
explicaciones. Este m\'{e}todo tiene una doble misi\'{o}n:
 por un lado permite comprobar el nivel de asimilaci\'{o}n de los
 contenidos por los estudiantes y, por otro lado, exige de \'{e}stos 
un esfuerzo de reflexi\'{o}n durante la explicaci\'{o}n que facilita 
la mejor asimilaci\'{o}n de  \'{e}sta. Por otra parte, el profesor 
debe estimular a los alumnos para que pregunten sobre cualquier punto 
oscuro de la explicaci\'{o}n y que expongan cualquier duda surgida 
durante la misma.

La presentaci\'{o}n de los temas que forman parte del curso debe seguir
 un esquema l\'{o}gico de tal forma que el alumno disponga de las 
herramientas necesarias para abordar un nuevo tema. Cuanto menor 
sea el n\'{u}mero de conceptos que el alumno deba asumir como conocidos,
 m\'{a}s profundo ser\'{a} el aprendizaje de la asignatura. 
En general, se trata de que, en la medida de lo posible, el curso 
sea autocontenido, lo que hace comprender al estudiante que se trata
 de una disciplina coherente y bien estructurada, basada en
 unos principios claros y bien establecidos.

Los recursos materiales en las exposiciones te\'{o}ricas son 
fundamentalmente la pizarra y los medios audiovisuales
(el retroproyector de transparencias y un ordenador con un ca\~{n}\'{o}n).
 La pizarra es un elemento imprescindible para la exposici\'{o}n
 de desarrollos matem\'{a}ticos o esquemas sencillos. 
El retroproyector de transparencias y el ordenador con  ca\~{n}\'{o}n,
 con los que se cuenta en todas las aulas de la Facultad de Ciencia y
 Tecnolog\'{\i}a, es de gran utilidad para visualizar esquemas
 y figuras complejas. 

Uno de los problemas que plantean las lecciones magistrales es que, en muchas 
ocasiones, los alumnos no siguen las explicaciones, sino que se centran
m\'{a}s en tomar notas, limit\'{a}ndose a copiar lo que el profesor
escribe en la pizarra.
Es por \'{e}sto que se considera  muy conveniente suministrar con
 antelaci\'{o}n al estudiante 
fotocopias de
 las trasparencias o los ficheros electr\'{o}nicos
 con las figuras que se mostrar\'{a}n y/o  un esquema de los apuntes que se 
desarrollar\'{a}n en clase.
Esta decisi\'{o}n puede ser discutible desde algunos puntos de vista,
y soy claramente consciente de estos potenciales inconvenientes. 
El principal riesgo es que el alumno decida sustituir 
las clases presenciales por otras actividades.

Por otro lado, considero  que existen ventajas en esta manera de proceder.
La entrega previa al estudiante  de un material  con el contenido de las clases
 permite al profesor
(quiz\'{a}s c\'{a}ndidamente) asumir que el alumno
interesado ha consultado al menos de qu\'{e} se le va a hablar.
La no necesidad de copiar
\emph{apuntes} permite al alumno concentrarse en la explicaci\'{o}n,
ampliar las notas de clase con sus propios gr\'{a}ficos o explicaciones, etc..
Mi experiencia personal es positiva respecto
al balance neto de esta pol\'\i tica de entregar por anticipado 
el material docente, y tambi\'{e}n ha sido muy positivamente valorada por
mis alumnos.
Este material lo he puesto a su disposici\'{o}n
 en forma de documentos PDF accesibles a trav\'{e}s da la 
plataforma {\it Moodle} de la Universidad. 

Debido a la diferente extensi\'{o}n de los temas del programa, 
resulta complicado preparar unidades did\'{a}cticas cerradas para cada clase
de 50 minutos. Es por ello que al principio de cada clase debe revisarse
lo \'{u}ltimo que se vi\'{o} en la clase anterior de forma que los alumnos 
se centren en la explicaci\'{o}n. Asimismo, al final de 
cada clase adem\'{a}s de recapitular sobre los puntos discutidos, 
tambi\'{e}n debe hacerse un breve comentario sobre lo que se abordar\'{a}
en la siguiente clase  en conexi\'{o}n con el contexto del tema.



\subsubsection{Pr\'{a}cticas de Aula (PA)}
La resoluci\'{o}n de problemas y el an\'{a}lisis de cuestiones son un
 medio ideal para poner en pr\'{a}ctica los conocimientos adquiridos 
en las clases te\'{o}ricas, por lo que deben ocupar una parte importante 
del curso. Los ejercicios pr\'{a}cticos constituyen una forma eficaz 
de asimilar y madurar  conocimientos,  a la vez
que un m\'{e}todo insustituible para poner de relieve detalles dif\'{\i}ciles
de resaltar en las explicaciones te\'{o}ricas. 
El desarrollo acad\'{e}mico
 de la F\'{\i}sica sin el planteamiento y resoluci\'{o}n de problemas,
 adem\'{a}s de resultar aburrido, es in\'{u}til.


Adem\'{a}s de se\~{n}alar a los estudiantes la conveniencia de utilizar
  libros de problemas indicados en la bibliograf\'{\i}a, 
es interesante que el profesor proponga una serie de problemas por cada tema.
 Los problemas propuestos deben corresponder con lo explicado en las clases 
de teor\'{\i}a y estar adaptados al nivel del estudiante.
Considero  conveniente incluir con cada problema 
 las soluciones algebr\'{a}icas o num\'{e}ricas del mismo,
 de forma que el alumno pueda 
comprobar que lo ha resuelto satisfactoriamente.

 Un problema bien elegido debe ser ilustrativo del tema que trata, 
ha de dar lugar a debate y de \'{e}l se han de extraer conclusiones 
interesantes. 
Despu\'{e}s de un tiempo razonable para su resoluci\'{o}n,
 se debe proceder a su correcci\'{o}n en clase. 
Para que las clases de problemas sean fruct\'{\i}feras es 
absolutamente indispensable la participaci\'{o}n activa de los estudiantes.
 En los grupos peque\~{n}os deben ser ellos los que corrijan los ejercicios en 
la pizarra mientras los dem\'{a}s alumnos y alumnas 
realizan los comentarios o preguntas 
que crean oportunas. 
Evidentemente, este proceso debe ser guiado por el profesor, 
quien a su vez debe plantear durante la correcci\'{o}n cuestiones
 que exijan al alumno un esfuerzo de reflexi\'{o}n.
 De esta forma se logra, por otra parte, una interacci\'{o}n mayor 
entre los estudiantes y el profesor, y \'{e}ste  puede obtener 
informaci\'{o}n sobre el grado de asimilaci\'{o}n de los contenidos por su
parte. Un aspecto sobre el que se debe incidir
 a menudo durante el curso es la importancia de que el estudiante
 trate de solucionar los problemas por s\'{\i} mismo, 
y no se limite a observar c\'{o}mo los soluciona el profesor.
 Esta \'{u}ltima manera de  proceder provoca una sensaci\'{o}n
 equivocada de dominio de la asignatura y suele suponer un
 fracaso a la hora de afrontar los ex\'{a}menes.


\subsubsection{Tutor\'{\i}as}

Los estudiantes disponen de seis horas semanales dedicadas a las tutor\'{\i}as,
 en las cuales pueden dirigirse al profesor para aclarar cualquier duda
 sobre la asignatura.
 Es tarea del profesor fomentar
 y potenciar el uso de las tutor\'{\i}as, de gran valor tanto 
para el alumnado como para el profesor, a la hora de calificarlos.
Sin embargo, mi experiencia personal es que no importa lo que  se insista a lo 
largo del curso para que acudan a  las tutor\'{\i}as. Los estudiantes
 en general esperan a 
la v\'{\i}spera del ex\'{a}men para hacerlo, lo que demuestra que a\'{u}n 
no han adoptado el papel que les corresponde en el proceso de ense\~{n}anza-aprendizaje.


\subsubsection{Seminarios}

Se  programan sobre distintos formatos, dependiendo del ritmo 
y las necesidades puntuales del curso. Se utilizan para 
 plantear problemas abiertos,
simulaciones inform\'{a}ticas, realizaci\'{o}n de trabajos, etc.
Se pueden plantear en tres fases:
\begin{enumerate}
\item Los estudiantes en horario no presencial y en grupos peque\~{n}os
realizan una primera aproximaci\'{o}n al problema.
\item En sesiones de tutor\'{\i}a o de seminario
el profesor orienta el trabajo de los distintos grupos.
\item En  sesiones sucesivas de seminario con todos los grupos,
 los alumnos  y alumnas
entregan un informe escrito, se realiza una puesta en com\'{u}n y 
se expone el trabajo.
\end{enumerate}

En particular, se propone la realizaci\'{o}n de una simulaci\'{o}n de ordenador
de alg\'{u}n problema f\'{\i}sico sencillo
 que se haya estudiado durante el curso y
 que implique la resoluci\'{o}n de ecuaciones diferenciales (algunas de 
las cuales los alumnos de primer curso no son a\'{u}n capaces de resolver).
Ser\'{a} una tarea a realizar en grupo durante la segunda mitad del 
segundo cuatrimestre de la asignatura.
Estas simulaciones se realizar\'{a}n con el lenguaje de programaci\'{o}n
{\it python} que los estudiantes han aprendido durante el primer
cuatrimestre del curso en la asignatura
 {\it Introducci\'{o}n a la Computaci\'{o}n}.



\subsubsection {\bf \emph{Moodle} y la comunicaci\'{o}n \emph{on-line} }

Desde hace unos a\~{n}os, la universidad dispone de un campus virtual 
del que dependen aulas virtuales de apoyo a la docencia presencial
 gestionadas a trav\'{e}s de la plataforma {\it } Moodle.
Todos los estudiantes matriculados en la asignatura son autom\'{a}ticamente
matriculados en el aula virtual. Las principales secciones de las que se 
hace uso son:

\begin{itemize} 
\item {\bf  Cuestionarios de autoevaluaci\'{o}n (auto-tests)}: Tras finalizar cada tema o un bloque de cada tema cada  estudiante realiza un test de 10 cuestiones
 de opci\'{o}n multiple con una sola respuesta correcta.
 Al cerrarse el cuestionario los alumnos y alumnas
 obtienen su puntuaci\'{o}n y la respuesta correcta a todas las preguntas.

\item {\bf  Ficheros}: En esta secci\'{o}n los alumnos y alumnas 
encuentran los apuntes,
esquemas, el listado y la soluci\'{o}n de los problemas desarrollados en clase,
 as\'{\i} como cualquier otro documento que tanto el profesor como los estudiantes
consideren de inter\'{e}s en relaci\'{o}n con la asignatura.
Est\'{a}n organizados por temas. 
\item {\bf  Enlaces}: En cada tema pueden a\~{n}adirse enlaces a p\'{a}ginas web
interesantes, con aplicaciones sencillas de lo explicado en clase, simulaciones
y  {\it applets}, figuras en tres dimensiones, \ldots
\item {\bf Foros}: En los foros
 tanto los estudiantes como el profesor pueden 
plantear preguntas y proponer respuestas.
El estudiante tiene la posibilidad de recibir todos los mensajes en su correo
electr\'{o}nico.
\item{ \bf Noticias}: Esta secci\'{o}n est\'{a} dedicada a la publicaci\'{o}n,
de aquellas noticias relacionadas con el curso: anuncio de ex\'{a}menes,
 publicaci\'{o}n de notas, anuncio de conferencias, \ldots
El estudiante  recibe las noticias en su correo
electr\'{o}nico.
\item{\bf Calendario}: El profesor puede marcar en el calendario fechas
 se\~{n}aladas, como fechas de controles o examenes, fechas de entrega de 
ejercicios, \ldots. Permite adem\'{a}s que los estudiantes reciban por 
correo electr\'{o}nico un recordatorio de las fechas se\~{n}aladas.
\end{itemize}

El acceso a {\it Moodle}, as\'{\i} como a Internet, por parte de los alumnos y alumnas
no deber\'{\i}a suponer  problema alguno para ninguno de ellos,
ya que tanto en la Facultad como en la Biblioteca del Campus existe un
n\'{u}mero suficiente de ordenadores a su disposici\'{o}n. Todos los estudiantes
reciben una cuenta de correo elect\'{o}nico corporativa, as\'{\i} como 
claves para acceder a la intranet de la universidad. 
Debido a que los alumnos y alumnas deben acceder a esta plataforma introduciendo
 un nombre de usuario y una contrase\~{n}a y todo queda registrado,
  el profesor puede conocer su utilizaci\'{o}n por parte de cada
 estudiante, y le permite hacer un seguimiento personal de la evoluci\'{o}n 
de cada uno a trav\'{e}s de sus intervenciones.


\subsubsection{Pr\'{a}cticas de laboratorio}

Las pr\'{a}cticas de laboratorio son esenciales
 en el aprendizaje de la F\'{\i}sica y proporcionan al estudiante  una idea
 m\'{a}s realista sobre los conceptos adquiridos en las clases te\'{o}ricas.
Es por ello que,
aunque la
asignatura {\it T\'{e}cnicas Experimentales I} en la que se realizan
las pr\'{a}cticas, no constituye el objeto de este Proyecto Docente, 
se citan aqu\'{\i} como un recurso metodol\'{o}gico m\'{a}s.

 La ense\~{n}anza de la F\'{\i}sica, como ciencia experimental que es, 
debe realizarse de tal manera que se transmita al alumno,
 tanto en las clases te\'{o}ricas como de problemas o de laboratorio,
 la idea de que los resultados de la F\'{\i}sica son fruto de la 
observaci\'{o}n detallada de la naturaleza y que, en l\'{\i}neas generales,
 el origen de las teor\'{\i}as f\'{\i}sicas hay que buscarlo en leyes 
emp\'{\i}ricas. Adem\'{a}s, es en estas clases donde el alumno tiene 
ocasi\'{o}n de participar m\'{a}s activamente y establecer una
 relaci\'{o}n m\'{a}s directa con el profesor.

Hay que tener en cuenta que esta  es la primera
 vez  que el alumno se va a introducir
en un laboratorio de F\'{\i}sica, y por tanto habr\'{a} que dedicar un tiempo 
a que se familiarice con las actividades de tipo pr\'{a}ctico, 
instrumentos b\'{a}sicos de medida, c\'{a}lculo de errores, 
an\'{a}lisis y discusi\'{o}n de datos, realizaci\'{o}n de informes 
de laboratorio, etc... Por  ello es importante incluir en la programaci\'{o}n
algunas 
sesiones de introducci\'{o}n al laboratorio sobre aspectos como el 
c\'{a}lculo de errores, representaci\'{o}n gr\'{a}fica e instrumentos 
de medida, sin olvidar por
otra parte las normas b\'{a}sicas sobre la correcta utilizaci\'{o}n
de los aparatos, as\'{\i} como de la seguridad en el laboratorio. 

Antes de comenzar una pr\'{a}ctica es importante que el alumno 
entienda los fundamentos en los que se basa y los objetivos que
 se persiguen con ella. Debe contar con la suficiente antelaci\'{o}n
 con un gui\'{o}n esquem\'{a}tico y claro de lo que se pretende
 con la pr\'{a}ctica y del modo de proceder. Una vez realizada la pr\'{a}ctica,
 el alumno deber\'{a} entregar al profesor un informe detallado de la misma,
 en el que presente los resultados obtenidos y realice un an\'{a}lisis
 cr\'{\i}tico de los mismos y de la pr\'{a}ctica en s\'{\i},
 sobre sus limitaciones e incluso posibles mejoras.

El dise\~{n}o de las pr\'{a}cticas debe ser el adecuado para que
 el alumno se vea obligado a reflexionar sobre los resultados 
obtenidos y no debe presentar una complejidad excesiva para 
que no pierda de vista los objetivos de las mismas. El conjunto
 de las pr\'{a}cticas debe abordar la mayor parte posible del
 contenido te\'{o}rico y su principal misi\'{o}n ser\'{a} la de 
desarrollar, en conjunto, una actitud sistem\'{a}tica y pr\'{a}ctica
 en el estudiante.



\newpage
\section{La evaluaci\'{o}n} \label{PD-evaluacion}
Por  \'{u}ltimo, es importante hacer referencia al m\'{e}todo de 
evaluaci\'{o}n de la asignatura. Es evidente que los 
m\'{e}todos de evaluaci\'{o}n afectan no s\'{o}lo a la calidad del 
aprendizaje, sino tambi\'{e}n a la actitud de los estudiantes hacia la materia
 de estudio.

Como consecuencia de la innovaci\'{o}n que supone la aplicaci\'{o}n del
``Plan Bolonia'', el cambio se debe reflejar en la {  evaluaci\'{o}n}
del estudiante que de estar centrada en el ``conocimiento declarativo'' como
referencia dominante y a veces \'{u}nica, pasa a incluir una evaluci\'{o}n
basada en las competencias, capacidades y procesos estrechamente relacionados
con el trabajo y las actividades que conducen al progreso del estudiante
y de una manera formal se traducen en la {  evaluaci\'{o}n continua}.

 Un m\'{e}todo de evaluaci\'{o}n ideal estar\'{\i}a basado en un
 conocimiento individualizado del estudiante, de tal forma que fuera 
posible llevar a cabo un seguimiento personalizado  
d\'{\i}a a d\'{\i}a de cada uno de ellos.
 La observaci\'{o}n diaria de la evoluci\'{o}n de cada estudiante 
elimina errores de evaluaci\'{o}n que inevitablemente
 aparecen en las calificaciones de los ex\'{a}menes finales.

Hasta ahora, 
dado el n\'{u}mero de alumnos y alumnas matriculados por t\'{e}rmino medio
en cada grupo de esta asignatura (alrededor de 80),
 no resulta f\'{\i}sicamente posible para ning\'{u}n profesor que tenga
varios grupos de docencia a su cargo
utilizar el  m\'{e}todo de  evaluaci\'{o}n continua por diversos motivos:
En primer lugar, algunos alumnos (especialmente 
los repetidores) no asisten de
forma habitual a clase. En esas condiciones, no es posible
realizar un seguimiento de su evoluci\'{o}n. Por otro lado, a\'{u}n entre
los que asisten, muchos de ellos rehuyen  salir a la pizarra,  
 especialmente en el primer curso de grado.
Por tanto, el seguimiento de la evoluci\'{o}n de los alumnos es muy
problem\'{a}tico.  Existe la v\'\i a alternativa de corregir las 
hojas de problemas de todos los alumnos, pero
esa es una tarea fuera del alcance de un profesor con  dos o
tres grupos
de  tantos alumnos cada uno, como ha sido mi caso en el \'{u}ltimo a\~{n}o.

Desde la perspectiva del EEES la evaluaci\'{o}n se concibe como un 
{  instrumento integrado} en el proceso de  ense\~{n}anza-aprendizaje
que {  proporciona  retroalimentaci\'{o}n} al estudiante
(informaci\'{o}n precisa sobre las \'{a}reas en las que debe mejorar),
y al profesor (indic\'{a}ndole los cambios que debe realizar 
para mejorar sus m\'{e}todos).
Para que la evaluaci\'{o}n pueda realizar una papel orientador e impulsor
del trabajo de los alumnos  y alumnas debe ser percibida por \'{e}stos como ayuda real,
generadora de {  expectativas positivas}.

Adem\'{a}s, debe extenderse a todos los aspectos {conceptuales},
{procedimentales} y {actitudinales}. Por otro lado, 
ha de  hacerse una evaluaci\'{o}n continua
 a lo largo de {  todo el proceso} y no limitarse a una
evaluaci\'{o}n terminal.



Este es quiz\'{a}s uno de los mayores cambios  que  supone la 
aplicaci\'{o}n de la nueva metodolog\'{\i}a del sistema EEES,
 ya que expl\'{\i}citamente se
alude a la necesidad (obligatoriedad)
 de realizar una {evaluaci\'{o}n continua} del
progreso del estudiante universitario.
Por esta raz\'{o}n el estudiante, repetidor o no,  no podr\'{a} optar
por no asistir a clase, ya que gran parte de su evaluaci\'{o}n proceder\'{a}
de tareas realizadas en clase, y la no asistencia
le acarrear\'{a} una valoraci\'{o}n no positiva en los indicadores 
evaluadores de ciertas competencias.
 Adem\'{a}s el proceso de evaluaci\'{o}n
no debe limitarse a comprobar la progresi\'{o}n del estudiante en la 
adquisici\'{o}n de conocimientos. El  sistema
se encamina m\'{a}s hacia la verificaci\'{o}n de las competencias
(en el sentido de ``demostrar ser competente para algo'')
obtenidas por el propio estudiante en cada materia, con su participaci\'{o}n
activa en un proceso continuo.

Como instrumentos concretos en la evaluaci\'{o}n nos referiremos
a: ex\'{a}menes, controles   auto-test online y simulaciones de ordenador.


\subsubsection{Examen}

El {  examen} debe suponer la culminaci\'{o}n de una {  revisi\'{o}n
global} de la materia considerada y debe incluir {  actividades coherentes}
con las utilizadas en el proceso de ense\~{n}anza-aprendizaje,
abarcando los contenidos conceptuales y procedimentales.

Es conveniente que el examen sea corregido lo antes posible y se discutan
las posibles respuestas, los errores cometidos, la
persistencia de preconcepciones, etc. De este modo el examen se convierte
en una buena actividad de {  autorregulaci\'{o}n}.
Ser\'{\i}a de utilidad que los estudiantes como trabajo no presencial,
rehagan el examen y lo conserven como material de referencia y consulta.

El planteamiento del examen debe ser realizado de una manera cuidadosa,
 ya que debe tener un grado de dificultad acorde con el nivel del curso,
 el enunciado debe ser claro y los resultados  no deben 
presentar ambig\"uedades. Se plantear\'{a} un examen de respuestas 
abiertas, pero concretas. En general, los ex\'{a}menes exclusivamente 
de tipo test no parecen los m\'{a}s
recomendables para las asignaturas de F\'\i sica.
Por otro lado, el n\'{u}mero de alumnos  y alumnas que se presenta a estos
ex\'{a}menes no es 
excesivamente alto, por lo que la econom\'\i a de medios que
suponen los ex\'{a}menes de tipo test no est\'{a} completamente justificada
en este caso.

El tiempo de realizaci\'{o}n del examen por parte de los estudiantes  no conviene
que sea muy ajustado, para no a\~{n}adir a la dificultad del propio examen 
la presi\'{o}n que supone la premura de tiempo.

 En general, el n\'{u}cleo del examen debe estar constituido por 
problemas y cuestiones conceptuales, evitando los ex\'{a}menes que 
requieran un esfuerzo esencialmente memor\'{\i}stico para el estudiante.
 Debe estar estructurado de forma que aparezcan diferentes niveles
 de dificultad y abarcar una parte representativa de los temas
 correspondientes a la asignatura. 
Por \'{u}ltimo, es conveniente evitar la aparici\'{o}n de 
desarrollos farragosos en su resoluci\'{o}n, 
y debe permitir al profesor utilizar criterios objetivos a la
 hora de proceder a su correcci\'{o}n.

\subsubsection{Controles}
Llamamos controles a aquellos ejercicios realizados durante  una hora de 
clase y que el alumno debe
entregar al profesor al finalizar la clase.
 Se trata de uno o dos ejercicios no demasiado extensos que el alumno
 debe realizar individualmente sin posibilidad de consultar
sus notas ni ning\'{u}n libro. 

La principal finalidad de estos
 ``mini-examenes'' es concienciar
al alumno para que lleve la asignatura ``al d\'{\i}a'', es decir, que 
haga un trabajo continuado y no espere al momento del examen para estudiar
 la asignatura en profundidad. 

Cada control se corrige en com\'{u}n en una sesi\'{o}n de pr\'{a}cticas de
 aula y una vez que el profesor ha revisado todos los controles, los
 estudiantes tienen la posibilidad de discutir y comentar su ejercicio
 con el profesor en horas de tutor\'{\i}a.

Bas\'{a}ndome en mi experiencia de a\~{n}os anteriores 
he constatado que el resultado del primer control  ayuda a muchos alumnos
 a darse cuenta de que quiz\'{a}s no estudian de forma adecuada 
o no dominan la materia como ellos
cre\'{\i}an. As\'{\i}, los controles se convierten
 tambi\'{e}n  en un 
buen mecanismo de autorregulaci\'{o}n para el estudiante.



\subsubsection{Auto-tests online}
Al final de cada tema o de cada bloque de un tema, cada alumno realiza
un test de 10 cuestiones. Para cada cuesti\'{o}n se proponen cinco posibles 
respuestas de las cuales solo una es correcta. En general se trata de
 preguntas cortas, tanto conceptuales como num\'{e}ricas.

La finalidad del test es que los estudiantes repasen el tema que se ha
 tratado en el aula recientemente.
 Es una tarea no presencial y disponen de 5 o 6 d\'{\i}as para
 realizarla y enviarla a trav\'{e}s de  la plataforma Moodle,
 de forma que pueden dedicar el tiempo que necesiten
para consultar sus notas, libros o discutir las cuestiones con sus
 compa\~{n}eros o incluso con el profesor. 

Cada una de las 10 cuestiones corresponde a una secci\'{o}n del tema. Para cada
secci\'{o}n se preparan m\'{a}s de 5 preguntas y a cada estudiante
se le asigna una sola pregunta de cada secci\'{o}n de forma aleatoria. 
En los tests de un aula virtual se barajan las preguntas y las respuestas
dentro de cada pregunta.
De esta forma la 
probabilidad de que dos estudiantes tengan el mismo test es m\'{\i}nima.

Una vez finalizado el plazo de entrega del test los alumnos pueden visualizar
sus calificaciones, asi como las soluciones correctas del test que han
 realizado, y pueden discutir con el profesor todas las dudas que les surjan.

\subsubsection{Tabla de evaluaci\'{o}n}

A continuaci\'{o}n se detalla las competencias que  se evaluan mediante los instrumentos mencionados y el valor que se le asigna a cada m\'{e}todo de evaluaci\'{o}n.


\begin{center}
\begin{tabular}{|lll|}
 \rowcolor[rgb]{0.8,0.8,0.8}\hline&&\\[-0.6cm]
%{\rowcolor[rgb]{0.8,0.8,0.8}
  {  Instrumento}& { Valor} & Competencia evaluada\\
 \hline\hline
{Auto-test online} & 15\% & CA1, CA2, CA3, CA5, CA6\\
{Controles} & 15\% & CA1, CA2, CA3, CA4\\
Pr\'{a}cticas de ordenador & 10\% & CA3 CA4, CA5, CA6\\
{Examen} & 60\% & CA1, CA2, CA3, CA4 \\
\hline
\end{tabular}
\end{center}

\begin{itemize}
\item La nota m\'{\i}nima que debe obtenerse en el examen es un 4 (sobre 10). 

\item En el caso de que la calificaci\'{o}n de los auto-tests reduzca la nota promedio 
del curso,  no se tendr\'{a}n en cuenta y el valor del examen ser\'{a} del 75\%. 
Para poder aplicar esta excepci\'{o}n el alumno debe realizar como m\'{\i}nimo 
el 75\% de los auto-tests y debe obtener una nota promedio superior a 5 en los tests
realizados. De esta forma se eval\'{u}a de manera m\'{a}s justa a aquellos alumnos
que durante el curso han progresado y mejorado.

\end{itemize}




%
\newpage
\section {Contenidos}

En el marco del EEES, 
y para facilitar la movilidad entre titulaciones de una misma rama de
conocimiento se establecen  contenidos formativos comunes a varias
titulaciones. Esta formaci\'{o}n com\'{u}n
no consiste en cursos id\'{e}nticos para todas las titulaciones
sino conocimientos b\'{a}sicos compartidos o competencias comunes
a todas ellas.

La {\it F\'{\i}sica General} de
la Facultad de Ciencia y Tecnolog\'{\i}a de la UPV/EHU es com\'{u}n
para los Grados en F\'{\i}sica,  Ingenier\'{\i}a Electr\'{o}nica,
y Matem\'{a}ticas. 
Es una asignatura anual de 12 ECTS que se imparte durante 2 cuatrimestres de 15 
semanas cada uno. Se proponen 36 clases magistrales y 
 21 clases de pr\'{a}cticas de aula por cuatrimestre. Las 6 clases restantes
se realizar\'{a}n en formato seminario, 2 en el primer cuatrimestre y 4
en el segundo.


Con respecto al plan anterior, el n\'{u}mero de horas presenciales
de F\'{\i}sica de primer curso
para el Grado en Matem\'{a}ticas no se modifica. Sin embargo, 
  los Grados en  F\'{\i}sica  e 
Ingenier\'{\i}a Electr\'{o}nica  reciben una hora menos por semana.
Dado que el contenido  
est\'{a} basado fundamentalmente en el programa de
{\it Fundamentos de F\'{\i}sica} de la Licenciatura en F\'{\i}sica
del plan 2000-2001, ha sido necesario ajustar los contenidos
del programa para cumplir el n\'{u}mero de cr\'{e}ditos asignado. 

En cuanto a la asignatura pr\'{a}ctica de laboratorio,
 contendr\'{a} \'{u}nicamente  pr\'{a}cticas de F\'{\i}sica
con la denominaci\'{o}n
{\it T\'{e}cnicas Experimentales I}.


A continuaci\'{o}n se especifica
el programa  de   {\it F\'{\i}sica General} y
{\it T\'{e}cnicas Experimentales I}, adaptado al sistema ECTS.



\subsection{Programa  de teor\'{\i}a: F\'{\i}sica General}
\subsubsection{\large Antecedentes y justificaci\'{o}n}

El programa de la asignatura {\it  F\'{\i}sica General}, objeto
del concurso,
contiene, en principio, gran parte de la F\'{\i}sica a 
 nivel b\'{a}sico.

El programa que presento a continuaci\'{o}n est\'{a} basado 
en el de la asignatura  {\it Fundamentos de F\'{\i}sica}
del plan de estudios  2000-2001.
La implantaci\'{o}n del plan  de estudios en el curso
2000-2001 ya supuso una reducci\'{o}n en el
  n\'{u}mero de cr\'{e}ditos asignado a la
asignatura {\it Fundamentos de F\'{\i}sica} 
con respecto a la asignatura {\it F\'{\i}sica General} del plan anterior,
 de 18 cr\'{e}ditos (6 horas semanales) a 15 cr\'{e}ditos (5 horas semanales), lo que
 implic\'{o} un recorte del programa de la {\it F\'{\i}sica General} ``tradicional''.
Es importante se\~{n}alar, que esta modificaci\'{o}n del temario no se produjo
de forma arbitraria, sino que se realiz\'{o} teniendo en cuenta 
los contenidos del programa oficial de ``F\'{\i}sica''  de 2{$^\circ$}
de Bachiller (Vibraciones y Ondas, \'{O}ptica, interacci\'{o}n gravitatoria,
e introducci\'{o}n a la F\'{\i}sica Moderna)
y el 
contexto en el que se enmarcaba la asignatura {\it Fundamentos de F\'{\i}sica} 
dentro de la Licenciatura en F\'{\i}sica.


De este modo, teniendo en cuenta la composici\'{o}n de aquel  plan
y la reducci\'{o}n de cr\'{e}ditos impuesta en el programa de la asignatura {\it Fundamentos de F\'{\i}sica}, 
se modificaron los siguientes temas correspondientes al programa 
de {\it F\'{\i}sica General} del plan anterior al 2000:
\begin {itemize}
\item {\bf  Est\'{a}tica y din\'{a}mica de fluidos}.
 Se elimin\'{o} este cap\'{\i}tulo por la mayor importancia que adquiri\'{o} 
esta materia en el  plan 2000-2001, 
al aparecer expl\'{\i}citamente en las asignaturas
 {\it Mec\'{a}nica y Ondas} de segundo curso y 
{\it F\'{\i}sica de los medios continuos} de tercero.
\item {\bf  Termodin\'{a}mica}.
 Se suprimi\'{o} por completo la parte dedicada a la
 Termodin\'{a}mica debido a que, como ya ocurr\'{\i}a en el plan antiguo, 
la asignatura {\it Termodin\'{a}mica} del segundo curso part\'{\i}a desde
  principios b\'{a}sicos.
\item {\bf Introducci\'{o}n a la F\'{\i}sica Cu\'{a}ntica}.
 Se elimin\'{o} por la presencia en el plan 2000-2001 de una 
nueva asignatura denominada 
{\it Introducci\'{o}n a la estructura de la materia}
 que, fundamentalmente, es una introducci\'{o}n a 
la F\'{\i}sica Moderna, que no exist\'{\i}a en el plan anterior.
\item {\bf Din\'{a}mica del S\'{o}lido R\'{\i}gido}:
 Con la implantaci\'{o}n del plan de estudios 2000-2001 el 
n\'{u}mero de horas dedicadas al estudio del
 s\'{o}lido r\'{\i}gido se vi\'{o} reducido,
limit\'{a}ndose al estudio de la rotaci\'{o}n de un s\'{o}lido
 alrededor de sus ejes principales de inercia, teorema de Steiner, 
radio de giro, energ\'{\i}a cin\'{e}tica del s\'{o}lido, 
movimiento de rodadura y condiciones de equilibrio de un cuerpo r\'{\i}gido.
 La parte  eliminada, que correspond\'{\i}a al estudio del
 tensor de inercia, \'{a}ngulos de Euler, 
ecuaciones de movimiento y movimiento girosc\'{o}pico 
se desarrollar\'{\i}a en la asignatura {\it Mec\'{a}nica y Ondas}
 de segundo curso.
\item{\bf  \'{O}ptica}: 
Esta parte se vi\'{o}  reducida y
limitada al estudio de la \'{o}ptica geom\'{e}trica, 
ya que en la asignatura {\it \'{O}ptica} de tercer curso se presupon\'{\i}an
 unos conocimientos m\'{\i}nimos al respecto.
\end{itemize}

El nuevo Grado en F\'{\i}sica ha impuesto en la 
 asignatura  {\it F\'{\i}sica General} 
 una nueva reducci\'{o}n en el n\'{u}mero clases presenciales 
semanales, de 5 a 4 horas, 
y su objetivo no es abarcar {\it toda} la F\'{\i}sica General
``tradicional'', sino que pretende establecer unas bases s\'{o}lidas
de conocimiento y razonamiento que le sirvan al alumno
para abordar con \'{e}xito las materias en la que se profundizar\'{a}
en los cursos superiores.

A pesar de la reducci\'{o}n de horas presenciales de la asignatura en el grado, 
el contenido respecto al plan de 2000-2001 no se ha reducido sustancialmente. 
Al contrario. Se ha recuperado el tema correspondiente a la est\'{a}tica 
y din\'{a}mica de fluidos de forma que los estudiantes del Grado en Matem\'{a}ticas
 e Ingenier\'{\i}a Electr\'{o}nica tengan una formaci\'{o}n b\'{a}sica en esta \'{a}rea
ya que no cursar\'{a}n la asignatura optativa {\it F\'{\i}sica de los medios continuos} en 3{$^\circ$} o 4{$^\circ$} curso.


 Debido a la extensi\'{o}n del programa, he cre\'{\i}do conveniente 
dividirlo en  tres bloques: Introducci\'{o}n,
 Mec\'{a}nica y Electromagnetismo. Estos  tres
 bloques tem\'{a}ticos tienen entidad propia, si bien el estudio de 
algunos de ellos requiere unos conocimientos adquiridos 
previamente.
 Sobre algunos de los temas los estudiantes  ya 
tienen alguna informaci\'{o}n b\'{a}sica recibida en Bachiller,
 en particular del bloque dedicado al Electromagnetismo. 
Sin embargo, el bloque dedicado a la Mec\'{a}nica 
supone una  mayor dificultad  para gran
 parte del alumnado. Una de las razones es que, en la actualidad,
la asignatura {\it Mec\'{a}nica} ya no se imparte  en Bachiller.
Por otra parte, todos los bloques tem\'{a}ticos aqu\'{\i} 
presentados corresponden a asignaturas completas de cursos superiores.


El programa actual de  {\it  F\'{\i}sica General}
comienza con una introducci\'{o}n sobre la naturaleza 
y m\'{e}todo de la F\'{\i}sica, seguida por un conjunto de temas dedicados
 al \'{a}lgebra vectorial. El segundo bloque  corresponde al estudio 
de la Mec\'{a}nica, que 
se volver\'{a} a ver en el  segundo curso
con un enfoque  m\'{a}s elevado: formalismo lagrangiano y hamiltoniano. 
A continuaci\'{o}n se incluye un apartado sobre el Electromagnetismo, 
disciplina que en cursos superiores se desdobla en varias asignaturas, 
por un lado {\it Electromagnetismo I y II} en el segundo y tercer curso 
respectivamente, y por otro, {\it \'{O}ptica}
 en el tercer curso del grado.
 En ambos casos se presuponen unos conocimientos previos por parte 
de los alumnos que cursen los Grados en F\'{\i}sica e Ingenier\'{\i}a Electr\'{o}nica. 

\newpage
\subsubsection{\large Esquema del programa te\'{o}rico}
\label{esquema-bolonia}


\noindent {\bf I - INTRODUCCI\'{O}N}

\begin{enumerate}[{\bf 1. }]
%
\item {\bf  Naturaleza y m\'{e}todo de la F\'{\i}sica}\hfill (0.1 ECTS)
\begin{itemize}\addtolength{\itemsep}{-0.25\baselineskip}
\noindent
\item  ?`Qu\'{e} es la F\'{\i}sica?
\item  El m\'{e}todo cient\'{\i}fico.
\item Part\'{\i}culas e interacciones. 
\item La estructura de las leyes f\'{\i}sicas, simetr\'{\i}a y leyes de conservaci\'{o}n. 
\item El Mundo material:  estados de agregaci\'{o}n de la materia.
\end{itemize}
%

\item {\bf  Magnitudes F\'{\i}sicas y vectores} \hfill    (0.4 ECTS)
\begin{itemize} \addtolength{\itemsep}{-0.25\baselineskip}
\noindent
\item   Magnitudes f\'{\i}sicas. Escalares y vectores.
\item  Magnitudes fundamentales y derivadas. Ecuaci\'{o}n de dimensiones.
 An\'{a}lisis dimensional.
\item  {O}rdenes de magnitud y cifras significativas.
\item Vectores libres.
% {\color{red} Cursores, pseudovectores}.
\item Operaciones con vectores libres: adici\'{o}n, producto de un vector por 
un escalar, producto escalar, producto vectorial, producto mixto, 
 doble producto vectorial. 
\item Sistemas de coordenadas y componentes de un vector. 
Triedro directo e inverso.
\item Campos escalares y vectoriales. Ejemplos.
\item Derivada de un vector respecto de un escalar.
\end{itemize}
%


\suspend{enumerate}
\noindent {\bf   II - MEC\'{A}NICA}

\resume{enumerate}[{[\bf 1.] }]
\item {\bf Cinem\'{a}tica del punto material} \hfill (0.6 ECTS)
\begin{itemize} \addtolength{\itemsep}{-0.25\baselineskip}
\noindent
\item  Sistemas de referencia.
\item Posici\'{o}n, velocidad y aceleraci\'{o}n de una part\'{\i}cula.
\item Movimiento con aceleraci\'{o}n constante.
\item Componentes intr\'{\i}nsecas de la velocidad y de la aceleraci\'{o}n.
\item Movimiento circular.
\item Movimiento curvil\'{\i}neo en el plano. Coordenadas polares. Velocidad radial y areolar.
\end{itemize}
%

\item {\bf  Movimiento relativo}\hfill (0.5 ECTS)
\begin{itemize} \addtolength{\itemsep}{-0.25\baselineskip}
\noindent
\item  Sistemas de referencia en movimiento relativo con velocidad constante.
 Transformaci\'{o}n de Galileo.
\item Sistemas de referencia en movimiento relativo de rotaci\'{o}n.
 Aceleraci\'{o}n de Coriolis. Ejemplos.
\item Ecuaciones de transformaci\'{o}n de la velocidad y de la
 aceleraci\'{o}n en el caso general.
\end{itemize}
%

\item {\bf  Din\'{a}mica del punto material}\hfill  (0.6 ECTS)
\begin{itemize} \addtolength{\itemsep}{-0.25\baselineskip}
\noindent
\item Principio de inercia: primera ley de Newton. Sistemas de referencia inerciales.
\item Momento lineal. Masa inercial.
\item Definici\'{o}n de fuerza: segunda ley de Newton. Principio de conservaci\'{o}n del momento lineal.
\item  Ley de acci\'{o}n-reacci\'{o}n, tercera ley de Newton.
\item  Fuerzas de contacto: reacci\'{o}n normal y resistencia al deslizamiento. Fuerzas a distancia.
\item  Resoluci\'{o}n de las ecuaciones del movimiento bajo distintos tipos de fuerzas:
Fuerzas constantes, fuerza ejercida por un muelle, fuerza de rozamiento proporcional a la velocidad.
\item  Momento angular. Momento de una fuerza respecto a un punto.
\item  Fuerzas centrales. Conservaci\'{o}n del momento angular. 
Movimiento de una part\'{\i}cula sometida a una fuerza central.
\item  Sistemas de referencia no inerciales. Fuerzas de inercia.
 La Tierra como sistema de referencia. El p\'{e}ndulo de Foucault.
\end{itemize}
%

\item {\bf Trabajo y energ\'{\i}a} \hfill (0.6 ECTS)
\begin{itemize} \addtolength{\itemsep}{-0.25\baselineskip}
\noindent
\item Trabajo de una fuerza. Potencia.
\item Trabajo y energ\'{\i}a cin\'{e}tica.
\item Fuerzas conservativas. Energ\'{\i}a potencial. Gradiente de un campo escalar.
\item Conservaci\'{o}n de la energ\'{\i}a mec\'{a}nica.
\item Energ\'{\i}a y equilibrio. Equilibrio estable e inestable.
\item Fuerzas no conservativas.
\end{itemize}
%

\item {\bf Din\'{a}mica de los sistemas de part\'{\i}culas}\hfill  (0.6 ECTS)
\begin{itemize} \addtolength{\itemsep}{-0.25\baselineskip}
\noindent
\item Fuerzas internas y externas. Ecuaciones del movimiento. Conservaci\'{o}n del momento lineal.
\item Centro de masa. Sistema de referencia del centro de masa.
 Descripci\'{o}n del movimiento  del centro de masa del sistema.
\item Momento angular. Conservaci\'{o}n del momento angular.
\item Trabajo realizado por las fuerzas internas y externas. Energ\'{\i}a
 cin\'{e}tica.
\item Fuerzas internas conservativas. Energ\'{\i}a potencial interna. 
Teorema de conservaci\'{o}n de la energ\'{\i}a. Energ\'{\i}a propia.
\item Fuerzas impulsivas y de impacto. Colisiones. Leyes de conservaci\'{o}n.
\item Experimentos en aceleradores. Creaci\'{o}n de part\'{\i}culas.
\item Sistemas de masa variable.
\end{itemize}
%

\item {\bf  Din\'{a}mica del s\'{o}lido r\'{\i}gido} \hfill (0.6 ECTS)
\begin{itemize} \addtolength{\itemsep}{-0.25\baselineskip}
\noindent
\item Energ\'{\i}a cin\'{e}tica de rotaci\'{o}n.
\item Ejes principales y momentos inercia.
\item C\'{a}lculo de momentos de inercia. F\'{o}rmula de Steiner y teorema de los ejes perpendiculares. Radio de giro.
\item Din\'{a}mica del s\'{o}lido r\'{\i}gido: Momento angular. Teorema del momento angular. Energ\'{\i}a. 
\item Estudio de casos particulares: rotaci\'{o}n alrededor de un eje fijo, 
movimiento de rodadura.
\item Condiciones de equilibrio de un cuerpo r\'{\i}gido.
\end{itemize}
%

\item {\bf  Interacci\'{o}n gravitatoria} \hfill (0.5 ECTS)
\begin{itemize} \addtolength{\itemsep}{-0.25\baselineskip}
\noindent
\item Introducci\'{o}n hist\'{o}rica.
\item Leyes de Kepler.
\item Ley de la gravitaci\'{o}n universal.
\item Experimento  de Cavendish.
\item Campo y potencial gravitatorio. Ley de Gauss.
\item Energ\'{\i}a potencial gravitatoria.
\item El problema de dos cuerpos. Reducci\'{o}n al problema equivalente de 
un solo cuerpo. Masa reducida.
\item Clasificaci\'{o}n de las \'{o}rbitas. Movimiento orbital: {O}rbitas circulares y \'{o}rbitas el\'{\i}pticas. Excentricidad y energ\'{\i}a.
\end{itemize}
%


\item {\bf  Fluidos}  \hfill (0.3 ECTS)
\begin{itemize} \addtolength{\itemsep}{-0.25\baselineskip}
\noindent
\item Propiedades y clasificaci\'{o}n de los fluidos.
\item Hidrost\'{a}tica: Ecuaci\'{o}n fundamental de la hidrost\'{a}tica.
\item Flotaci\'{o}n, empuje y  Principio de Arqu\'{\i}medes.
\item Hidrodin\'{a}mica: Flujo de un campo vectorial y ecuaci\'{o}n de continuidad. 
\item Ecuaci\'{o}n de Bernouilli y aplicaciones.
\item Viscosidad. Ley de Poiseuille.
\end{itemize}
%


\item {\bf  Movimiento oscilatorio}\hfill   (0.4 ECTS)
\begin{itemize} \addtolength{\itemsep}{-0.25\baselineskip}
\noindent
\item  El oscilador arm\'{o}nico simple. Ecuaci\'{o}n del movimiento.
\item Energ\'{\i}a del oscilador arm\'{o}nico.
\item Ejemplos de osciladores: Resorte lineal. P\'{e}ndulo simple.
 P\'{e}ndulo f\'{\i}sico.  P\'{e}éndulo de torsi\'{o}n.
\item Composici\'{o}n de movimientos arm\'{o}nicos.
\item Oscilaciones amortiguadas. Clases de amortiguamiento.
\item Oscilaciones forzadas.  Resonancia.
\end{itemize}
%

\item {\bf  Movimiento ondulatorio.}\hfill      (0.5 ECTS)
\begin{itemize} \addtolength{\itemsep}{-0.25\baselineskip}
\noindent
\item   Descripci\'{o}n matem\'{a}tica de la propagaci\'{o}n de una perturbaci\'{o}n.
\item La ecuaci\'{o}n de ondas.
\item Ondas en una dimensi\'{o}n. Ondas transversales en una cuerda. Ondas longitudinales en una barra met\'{a}lica.
\item Ondas arm\'{o}nicas. Magnitudes fundamentales. Superposici\'{o}n. Interferencia de ondas arm\'{o}nicas. 
\item Ondas estacionarias en una cuerda y en una columna de aire. Modos de vibraci\'{o}n. 
\item Efecto Doppler. 
\end{itemize}
%

\suspend{enumerate}
\noindent {\bf   III - ELECTROMAGNETISMO}

\resume{enumerate}[{[\bf 1.] }]
\item {\bf Interacci\'{o}n electrost\'{a}tica}    \hfill (0.9 ECTS)
\begin{itemize} \addtolength{\itemsep}{-0.25\baselineskip}
\noindent
\item Introducci\'{o}n. %hist\'{o}rica.
\item Carga el\'{e}ctrica. Conservaci\'{o}n y cuantificaci\'{o}n de la carga
 el\'{e}ctrica. Estructura del \'{a}tomo.
\item Ley de Coulomb. Campo el\'{e}ctrico. Potencial el\'{e}ctrico.
\item C\'{a}lculo del campo y potencial el\'{e}ctricos debidos a distintas
 distribuciones de carga. L\'{\i}neas de campo y superficies equipotenciales.
\item  Flujo del campo el\'{e}ctrico. Ley de Gauss. Aplicaciones.
\item  Energ\'{\i}a de una distribuci\'{o}n de cargas.
\item Dipolo el\'{e}ctrico. Momento dipolar. 
Acci\'{o}n del campo el\'{e}ctrico sobre un dipolo.
\item Clasificaci\'{o}n de los materiales por su comportamiento frente a un
 campo el\'{e}ctrico. Conductores y aislantes.
\item Materiales diel\'{e}ctricos en campos electrost\'{a}ticos.
 Polarizabilidad de un medio diel\'{e}ctrico. Polarizaci\'{o}n.
\item  Densidades de carga de polarizaci\'{o}n.
\end{itemize}
%
%
\item {\bf Materiales conductores en campos electrost\'{a}ticos} \hfill (0.4 ECTS) 
\begin{itemize} \addtolength{\itemsep}{-0.25\baselineskip}
%\noindent
\item  Campo y potencial el\'{e}ctricos en el interior de un conductor en equilibrio. Reparto de las cargas.
\item  Campo en la superficie de un conductor. Poder de las puntas.
\item Capacidad de un conductor aislado.
\item Condensadores. Capacidad. Energ\'{\i}a.
\item  Asociaci\'{o}n de condensadores: capacidad equivalente y energ\'{\i}a.
\item Densidad de energ\'{\i}a asociada al campo el\'{e}ctrico.
%\item Materiales diel\'{e}ctricos en campos electrost\'{a}ticos.
% Polarizabilidad de un medio diel\'{e}ctrico. Polarizaci\'{o}n.
%\item  Densidades de carga de polarizaci\'{o}n.
\item  Efecto de un diel\'{e}ctrico sobre la capacidad de un condensador.

\end{itemize}

% %%%%%%%%%%%%5
% HAU LABURTUTA aurreko bi gaietan
%%%%%%%%%%%%%%%%%%%
%\item {\bf Materiales diel\'{e}ctricos en campos electrost\'{a}ticos} \hfill{\color{red} ( 6 horas ) }
%\begin{itemize} \addtolength{\itemsep}{-0.25\baselineskip}
%\noindent
%\item   Estructura molecular y propiedades diel\'{e}ctricas: la aproximaci\'{o}n dipolar.
%\item Polarizabilidad de un medio diel\'{e}ctrico. Polarizaci\'{o}n.
%\item  Densidades de carga de polarizaci\'{o}n.
%\item Susceptibilidad y permitividad diel\'{e}ctrica. Desplazamiento el\'{e}ctrico. 
%Generalizaci\'{o}n de la ley de Gauss en presencia de diel\'{e}ctricos.
%\item  Efecto de un diel\'{e}ctrico sobre la capacidad de un condensador.
%\item  Energ\'{\i}a electrost\'{a}tica en presencia de diel\'{e}ctricos.
%\end{itemize}


\item {\bf Corriente el\'{e}ctrica. Circuitos de corriente continua} \hfill (0.9 ECTS)
\begin{itemize} \addtolength{\itemsep}{-0.25\baselineskip}
\noindent
\item   Corriente y densidad de corriente.
\item  Ecuaci\'{o}n de continuidad.
\item  Corrientes de conducci\'{o}n. Ley de Ohm. Conductividad. Resistencia.
\item  Punto de vista microsc\'{o}pico. Modelo de Drude.
\item  Fuerza electromotriz.
\item  Disipaci\'{o}n de energ\'{\i}a. Ley de Joule.
\item  Combinaciones de resistencias. Leyes de Kirchhoff.
\item  Circuitos RC.
\end{itemize}


\item {\bf Interacci\'{o}n magn\'{e}tica} \hfill (1 ECTS) 
\begin{itemize} \addtolength{\itemsep}{-0.25\baselineskip}
\noindent
\item    Introducci\'{o}n.
\item  Fuerza magn\'{e}tica sobre una carga en movimiento.
\item  Movimiento de una part\'{\i}cula cargada en un campo magn\'{e}tico. 
Aplicaciones: selector de velocidades, espectr\'{o}metro de masas, ciclotr\'{o}n.
\item  Acci\'{o}n de un campo magn\'{e}tico sobre una corriente el\'{e}ctrica.
\item  Momento dipolar  magn\'{e}tico. Momento sobre una espira en un campo magn\'{e}tico.
%\item  Efecto Hall.
\item  Campo magn\'{e}tico creado por una carga en movimiento.
\item  Campo creado por una corriente el\'{e}ctrica: ley de Biot y Savart. Ejemplos.
\item  Fuerzas entre corrientes.
\item  Ley de Amp\`{e}re. Ejemplos.
%\item Dipolo magn\'{e}tico. Campo creado por un dipolo magn\'{e}tico.
\item  Ley de Gauss para el campo magn\'{e}tico.
\item Campo magn\'{e}tico en la materia.  Imanaci\'{o}n.
\item Susceptibilidad y permeabilidad magn\'{e}ticas.
\item Diamagnetismo, paramagnetismo y ferromagnetismo.
\end{itemize}

%%%%%%%%%%%%%%%%%%%%%%%%%%%%%%%%%%%%%%%%%%%%%%%%%%%%%%%%%%%%%%%
%%%%%% HAU LABURTUTA \'{I}NTERACCION MAGENTICA' GAIAN %%%%%%%%%
%%%%%%%%%%%%%%%%%%%%%%%%%%%%%%%%%%%%%%%%%%%%%%%%%%%%%%%%%%%%%%%
%\item {\bf Campo magn\'{e}tico en la materia} \hfill{\color{red} ( 6 horas ) }
%\begin{itemize} \addtolength{\itemsep}{-0.25\baselineskip}
%\noindent
%\item    Imanaci\'{o}n
%\item  Densidad de corriente de imanaci\'{o}n. Generalizaci\'{o}n de la ley de Amp\`{e}re. Excitaci\'{o}n magn\'{e}tica. Susceptibilidad y permeabilidad magn\'{e}ticas.
%\item El momento dipolar magn\'{e}tico de los \'{a}tomos.
%\item  Diamagnetismo, paramagnetismo y ferromagnetismo.
%\item  Magnetismo y superconductividad. Efecto Meissner.
%\end{itemize}

\item {\bf Campos electromagn\'{e}ticos dependientes del tiempo} \hfill (1.2 ECTS) 
\begin{itemize} \addtolength{\itemsep}{-0.25\baselineskip}
\noindent
\item Introducci\'{o}n.
\item  Flujo de un campo magn\'{e}tico.
\item  Ley de Faraday-Henry. Ley de Lenz. Aplicaciones.
\item  Autoinducci\'{o}n. Inducci\'{o}n mutua.
\item Energ\'{\i}a del campo electromagn\'{e}tico.
\item Oscilaciones electromagn\'{e}ticas: circuitos RLC. Analog\'{\i}a con el movimiento arm\'{o}nico. Resonancia.
\item Circuitos de corriente alterna.
\item Corriente de desplazamiento de Maxwell. Ley de Amp\`{e}re-Maxwell.
\item  Ecuaciones de Maxwell.
\end{itemize}


\item {\bf Ondas electromagn\'{e}ticas} \hfill (0.7 ECTS) 
\begin{itemize} \addtolength{\itemsep}{-0.25\baselineskip}
\noindent
\item  Obtenci\'{o}n de la ecuaci\'{o}n de ondas a partir de las ecuaciones
 de Maxwell. Velocidad de las ondas electromagn\'{e}ticas.
\item  Ondas electromagn\'{e}ticas planas. Car\'{a}cter transversal de las ondas.
 Relaciones entre los campos el\'{e}ctrico y magn\'{e}tico. Polarizaci\'{o}n.
\item  Energ\'{\i}a y momento de las ondas electromagn\'{e}ticas. Vector de Poynting.
\item  Fuentes de ondas electromagn\'{e}ticas.
\item  Propagaci\'{o}n de ondas electromagn\'{e}ticas en la materia: dispersi\'{o}n.
\item  Espectro de la radiaci\'{o}n electromagn\'{e}tica. 
%Efecto Doppler en ondas electromagn\'{e}ticas.
\end{itemize}



\item {\bf \'{O}ptica} \hfill (0.6 ECTS)
\begin{itemize} \addtolength{\itemsep}{-0.25\baselineskip}
\noindent
\item  Teor\'{\i}a ondulatoria y corpuscular.
\item  Principio de Huygens. Reflexi\'{o}n y refracci\'{o}n. Reflexi\'{o}n total.
% {\color{red} Experimento de Young de la doble rendija}.
\item  Aproximaci\'{o}n geom\'{e}trica. Rayo de luz. Principio de Fermat.
\item  Espejos y lentes. Diagramas de rayos. Combinaciones de lentes.
\item  Instrumentos \'{o}pticos: C\'{a}mara fotogr\'{a}fica. Microscopio. Telescopio. Ojo humano.
%\item  Aberraciones.
\end{itemize}

\end{enumerate}



\newpage
\subsubsection{\large Programa te\'{o}rico comentado}

En esta secci\'{o}n se comentan de forma detallada todos los
 cap\'{\i}tulos del programa de teor\'{\i}a,
 destacando los aspectos m\'{a}s importantes de los mismos.\\

\noindent {\bf  I - INTRODUCCI\'{O}N}
\begin{enumerate} [{\bf 1. }]


\item  {\bf  Naturaleza y m\'{e}todo de la F\'{\i}sica.}
Este cap\'{\i}tulo, de car\'{a}cter introductorio, comienza con una
 visi\'{o}n general e hist\'{o}rica de la F\'{\i}sica. Se establecen los
 objetivos que dicha ciencia persigue y los procedimientos utilizados para ello.
 En particular se hace hincapi\'{e} en el car\'{a}cter experimental de la F\'{\i}sica,
 y en la utilizaci\'{o}n del m\'{e}todo cient\'{\i}fico como procedimiento
 adecuado para el establecimiento de las leyes de la naturaleza. 


\item {\bf  Magnitudes F\'{\i}sicas y vectores.}
Se introducen los conceptos de magnitud, unidad,
 orden de magnitud y n\'{u}mero de cifras significativas, y se efect\'{u}a
 una revisi\'{o}n de los principales sistemas de unidades y las relaciones
 entre los mismos. A continuaci\'{o}n
se definen los vectores libres y las operaciones m\'{a}s importantes entre 
vectores y entre vectores y escalares, mostrando ejemplos sencillos de 
su aplicaci\'{o}n posterior. Se definen los distintos sistemas de 
coordenadas y se introduce el concepto de campo escalar y campo vectorial.
Se finaliza con la derivada y la integral de un vector respecto a un escalar.


\suspend{enumerate}
\noindent {\bf  II - MEC\'{A}NICA}

\resume{enumerate}[{[\bf 1.] }]
\item {\bf Cinem\'{a}tica del punto material.}
Con este cap\'{\i}tulo se inicia el estudio de la mec\'{a}nica. 
Se revisan los conceptos de part\'{\i}cula y sistema de referencia, y se estudia 
el movimiento desde un punto de vista puramente geom\'{e}trico, sin atender
a las causas que lo producen.
Se definen los conceptos de posici\'{o}n, velocidad,  aceleraci\'{o}n,
y su naturaleza vectorial,  y se formulan las ecuaciones 
generales del movimiento. A continuaci\'{o}n se estudian algunos ejemplos: movimiento
 en una dimensi\'{o}n con velocidad constante, con aceleraci\'{o}n constante y
 con aceleraci\'{o}n conocida dependiente del tiempo y se extiende el an\'{a}lisis a 
dos dimensiones: tiro parab\'{o}lico y movimiento circular.
Finalmente se presta atenci\'{o}n al movimiento curvil\'{\i}neo en el plano y 
su descripci\'{o}n en coordenadas polares, por su posterior aplicaci\'{o}n al estudio
del movimiento bajo fuerzas centrales, y en particular, al estudio del movimiento
planetario.


\item {\bf  Movimiento relativo.}
Se describe el movimiento relativo desde el punto de vista cinem\'{a}tico. Se 
obtienen las transformaciones que relacionan
los vectores de posici\'{o}n, velocidad y aceleraci\'{o}n de una part\'{\i}cula
con respecto a dos sistemas de referencia en movimiento relativo de
 traslaci\'{o}n (transformaciones de Galileo), movimiento relativo solo 
de rotaci\'{o}n o general (traslaci\'{o}n y rotaci\'{o}n). 
Mediante una serie de ejemplos, como el p\'{e}ndulo de Foucault,
 se resalta la importancia del movimiento relativo 
de rotaci\'{o}n. 


\item {\bf  Din\'{a}mica del punto material.}
En este cap\'{\i}tulo se introducen los principios y conceptos de la
 mec\'{a}nica newtoniana aplic\'{a}ndola en el caso m\'{a}s sencillo, el de una
 \'{u}nica part\'{\i}cula puntual. En primer lugar se enuncia la ley de la inercia
 y se definen los sistemas de referencia inerciales. A continuaci\'{o}n se
 introduce el concepto de masa inercial y se definen el momento lineal y la fuerza, 
a partir de los cuales se enuncian la segunda y tercera leyes de Newton.
 Como ejemplos de aplicaci\'{o}n se establecen las ecuaciones de movimiento
 para algunos sistemas con diferentes tipos de fuerzas: los casos m\'{a}s
 sencillos de fuerzas constantes a distancia y de contacto
 as\'{\i} como el de una fuerza variable, como por ejemplo el rozamiento en un fluido.
As\'{\i}, con el ejemplo de un paraca\'{\i}das se 
introduce el concepto de velocidad l\'{\i}mite. Seguidamente se define el 
momento angular y el momento de una fuerza para estudiar posteriormente el
 movimiento de una part\'{\i}cula sometida a una fuerza central y enunciar
 el teorema de conservaci\'{o}n del momento angular.
 Finalmente, mediante ejemplos concretos como el de una bombilla colgada en un tren, lanzamientos
desde una plataforma giratoria, etc... , se analiza el movimiento de una 
part\'{\i}cula desde un sistema de referencia no inercial,
 definiendo las denominadas fuerzas de inercia. 

\item {\bf Trabajo y energ\'{\i}a.}
Se definen el trabajo realizado por una fuerza sobre una part\'{\i}ícula y
 la energ\'{\i}a cin\'{e}tica, estableci\'{e}ndose la relaci\'{o}n entre ambos.
 En especial, se estudia el caso de las fuerzas conservativas, introduciendo
 los conceptos de energ\'{\i}a potencial y de energ\'{\i}a mec\'{a}nica para
 enunciar el principio de conservaci\'{o}n de la energ\'{\i}a mec\'{a}nica.
 Seguidamente se utilizan las nuevas magnitudes definidas para estudiar el 
movimiento de una part\'{\i}cula en casos particulares, desde el punto de vista
 de la conservaci\'{o}n de la energ\'{\i}a (movimiento bajo una fuerza constante en
una dimensi\'{o}n y movimiento bajo el efecto de una fuerza que cumple la Ley de Hooke).
 Se debe hacer especial hincapi\'{e} 
en que esta nueva forma de obtener la trayectoria de la part\'{\i}cula proviene 
directamente de las leyes de la din\'{a}mica y es, por tanto, equivalente a ellas.
 Finalmente se analiza el movimiento de la part\'{\i}cula a partir de curvas
 de energ\'{\i}a y se estudian ejemplos en los que intervienen fuerzas 
no conservativas.

\item {\bf Din\'{a}mica de los sistemas de part\'{\i}culas.}
En este cap\'{\i}tulo se extiende el an\'{a}lisis din\'{a}mico a sistemas
 compuestos por varias part\'{\i}culas puntuales. En primer lugar se hace 
la distinci\'{o}n entre fuerzas internas y externas del sistema y se define el 
centro de masa del mismo. A continuaci\'{o}n se enuncian y demuestran los teoremas 
de conservaci\'{o}n del momento lineal, del momento angular y de la energ\'{\i}a
 mec\'{a}nica. Se realiza el an\'{a}lisis desde el sistema de referencia 
del laboratorio y desde el sistema de referencia del centro de masa, 
se\~{n}alando la conveniencia de separar el movimiento respecto a cualquier
 sistema de referencia en dos partes: el del centro de masa y el movimiento propio 
del sistema (relativo a su propio centro de masa). 
Finalmente se estudian algunos casos de particular inter\'{e}s:
 el problema de dos cuerpos y su reducci\'{o}n al de un solo cuerpo previa
 definici\'{o}n de la masa reducida, 
las colisiones entre dos part\'{\i}culas tanto en una como en dos dimensiones
 y los sistemas de masa variable (cohetes).

\item {\bf  Din\'{a}mica del s\'{o}lido r\'{\i}gido.}
El s\'{o}lido r\'{\i}gido es un caso muy especial de sistema de part\'{\i}culas,
 por lo que es preferible estudiarlo en un cap\'{\i}tulo aparte. En el caso ideal,
 las distancias entre las part\'{\i}culas que componen el sistema permanecen 
constantes, lo que reduce el n\'{u}mero de grados de libertad, haci\'{e}ndose 
patente la necesidad de introducir nuevas t\'{e}cnicas en su estudio.
 En primer lugar se estudia la cinem\'{a}tica del s\'{o}lido r\'{\i}gido, para la 
cual se introduce el sistema de referencia propio, se se\~{n}alan los
 grados de libertad del sistema y se analiza el campo de velocidades. 
A continuaci\'{o}n se plantean las ecuaciones din\'{a}micas del sistema, 
se define el momento de inercia y  se calcula para algunas simetr\'{\i}as. 
Se aplica la f\'{o}rmula de Steiner para calcular los momentos de inercia 
respecto a ejes arbitrarios paralelos a los principales. 
Seguidamente se obtiene la expresi\'{o}n para la energ\'{\i}a cin\'{e}tica del
 s\'{o}lido, separ\'{a}ndola en dos partes: la energ\'{\i}a cin\'{e}tica de 
traslaci\'{o}n 
y la de rotaci\'{o}n, analizando el caso particular del movimiento de rodadura. 
Finalmente se aborda el problema de la est\'{a}tica, estableciendo las condiciones
 de equilibrio del s\'{o}lido r\'{\i}gido  aplic\'{a}ndolas a algunos ejemplos en dos
 y tres dimensiones.

\item {\bf  Interacci\'{o}n gravitatoria.}
Como caso particular de un sistema de dos cuerpos, en este tema analizaremos uno
 de los problemas m\'{a}s interesantes en la historia de la F\'{\i}sica:
 la interacci\'{o}n gravitatoria y el movimiento planetario. 
Se repasan brevemente los antecedentes hist\'{o}ricos que condujeron al 
enunciado de las leyes de Kepler y a la ley de la gravitaci\'{o}n universal.
 Se se\~{n}ala la diferencia conceptual entre masa inercial y masa gravitatoria,
 se enuncia el principio de equivalencia a partir del cual se introduce la
 constante de gravitaci\'{o}n universal y se describe el experimento 
 de Cavendish.
 Seguidamente se definen el campo gravitatorio y el potencial gravitatorio 
para describir la interacci\'{o}n a distancia, y se calculan los campos y
 potenciales creados por objetos de geometr\'{\i}a sencilla. Finalmente,
se estudia en detalle el problema de dos cuerpos con fuerzas centrales, se resuelve 
la ecuaci\'{o}n de movimiento, se obtiene la ecuaci\'{o}n de las \'{o}rbitas
y se estudia el movimiento de los planetas y sat\'{e}lites.


\item {\bf  Fluidos.}
En este tema se comienza con la clasificaci\'{o}n de los fluidos y sus propiedades.
 Se estudia la hidrost\'{a}tica, obteniendo la ecuaci\'{o}n fundamental de la 
hidrost\'{a}tica y la ley de Arqu\'{\i}medes. A continuaci\'{o}n se define 
el flujo de un campo vectorial para pasar a la parte de hidrodin\'{a}mica e 
introducir la ecuaci\'{o}n de continuidad y 
la ecuaci\'{o}n de Bernouilli para los fluidos ideales y sus aplicaciones.
Finalmente se trata la viscosidad y la ley de Poiseuille.

\item {\bf  Movimiento oscilatorio.}
Se introduce el concepto de oscilador, destacando su importancia como 
modelo en muchos problemas de f\'{\i}sica. En primer lugar se analiza el
 ejemplo de una part\'{\i}cula sujeta a un resorte lineal que satisface la
 ley de Hooke.
 Se plantea la ecuaci\'{o}n de movimiento y la de la energ\'{\i}a. 
A continuaci\'{o}n se
 presentan otros sistemas que poseen la misma ecuaci\'{o}n din\'{a}mica en primera 
aproximaci\'{o}n: p\'{e}ndulo simple, p\'{e}ndulo f\'{\i}sico, 
sistemas ligeramente desplazados
 de la situaci\'{o}n de equilibrio din\'{a}mico, etc... 
Se analiza la composici\'{o}n de movimientos arm\'{o}nicos en una y dos dimensiones.
 Seguidamente se introduce en el sistema una fuerza de amortiguamiento proporcional
 a la velocidad y se analizan los diferentes movimientos, en funci\'{o}n 
de la constante de amortiguamiento. 
Finalmente se a\~{n}ade al sistema una fuerza externa peri\'{o}dica para estudiar
 las oscilaciones forzadas y los fen\'{o}menos de resonancia.



\item {\bf  Movimiento ondulatorio.}
Para terminar el estudio de la Mec\'{a}nica se analiza el movimiento ondulatorio
 en medios materiales. El cap\'{\i}tulo comienza con la descripci\'{o}n
 matem\'{a}tica de la propagaci\'{o}n de una perturbaci\'{o}n sobre un medio en
 equilibrio din\'{a}mico, obteni\'{e}ndose la ecuaci\'{o}n diferencial que
 satisface una onda. Despu\'{e}s se analiza la propagaci\'{o}n en una dimensi\'{o}n,
 tanto de las ondas longitudinales en una barra met\'{a}lica como de las ondas
 transversales en una cuerda. Se obtienen las ecuaciones que satisfacen esos dos
 sistemas y se comparan con la ecuaci\'{o}n de ondas para obtener la velocidad 
de propagaci\'{o}n. Despu\'{e}s se introducen las ondas arm\'{o}nicas y se estudia 
la interferencia de las mismas. 
A continuaci\'{o}n se introducen las ondas estacionarias en una cuerda y en una
 columna de aire y finalmente se describe el efecto Doppler.



\suspend{enumerate}
\noindent {\bf  III - ELECTROMAGNETISMO}

\resume{enumerate}[{[\bf 1.] }]
\item {\bf Interacci\'{o}n electrost\'{a}tica.} 
Se comienza el estudio del electromagnetismo con una breve introducci\'{o}n
 hist\'{o}rica para llegar al hecho experimental de la existencia de la
 carga el\'{e}ctrica, y se enuncian los principios de conservaci\'{o}n y 
cuantificaci\'{o}n de la misma. Seguidamente se presenta la ley de Coulomb 
y se comparan cuantitativamente las interacciones gravitatoria y 
electrost\'{a}tica en el sistema prot\'{o}n-electr\'{o}n, con el fin de mostrar 
la importancia de la segunda frente a la primera. 
Una vez puesto de manifiesto el car\'{a}cter conservativo de la interacci\'{o}n 
se calcula el campo y el potencial electrost\'{a}ticos. 
Seguidamente se define el concepto de flujo del campo el\'{e}ctrico 
y se deduce la ley de Gauss, que se utilizar\'{a} para calcular los campos
 y potenciales debidos a distribuciones concretas de carga. Despu\'{e}s 
se calcular\'{a} su energ\'{\i}a electrost\'{a}tica. A continuaci\'{o}n se 
analiza el dipolo el\'{e}ctrico como un ejemplo muy importante de 
distribuci\'{o}n de cargas, se introduce el concepto de momento dipolar y se
 estudia su interacci\'{o}n con un campo electrost\'{a}tico externo.
Se realiza una clasificaci\'{o}n de los materiales en funci\'{o}n de su 
comportamiento en presencia de un campo electrost\'{a}tico.
Se finaliza con un breve an\'{a}lisis del comportamiento de los materiales 
diel\'{e}ctricos en presencia de un campo electrost\'{a}tico externo
introduciendo el concepto de polarizaci\'{o}n y densidad de carga de polarizaci\'{o}n. 


\item {\bf Materiales conductores en campos electrost\'{a}ticos.}
En este cap\'{\i}tulo se estudian los materiales conductores en equilibrio,
 su distribuci\'{o}n de cargas, y el campo y potencial electrost\'{a}ticos tanto 
en su interior como en el exterior. Se introduce el concepto de capacidad y se
 estudian diferentes tipos de condensadores y los diferentes tipos de 
combinaci\'{o}n de condensadores. A continuaci\'{o}n se obtiene la expresi\'{o}n 
de la energ\'{\i}a asociada al campo el\'{e}ctrico.
Para finalizar 
se estudian las consecuencias de introducir un diel\'{e}ctrico entre las placas de un
 condensador de caras planas. 


\item {\bf Corriente el\'{e}ctrica. Circuitos de corriente continua.}
Se analiza la conducci\'{o}n el\'{e}ctrica en r\'{e}gimen estacionario. 
En primer lugar se definen la intensidad el\'{e}ctrica y la densidad de
 corriente para obtener la ecuaci\'{o}n de continuidad a partir del principio
 de conservaci\'{o}n de la carga. A continuaci\'{o}n se define la conductividad
 de un material y se enuncia la ley de Ohm que relaciona el campo el\'{e}ctrico
 y la densidad de corriente en cada punto. Mediante el modelo de electrones
 libres de Drude se obtiene la resistividad de un material a partir de
 par\'{a}metros microsc\'{o}picos: masa y carga de los portadores, el tiempo
 medio entre colisiones y la densidad. Una vez definida la resistencia se introduce
 el concepto de fuerza electromotriz y se generaliza la ley de Ohm. Se aplica la
 ley de Ohm en alg\'{u}n circuito y se analiza la combinaci\'{o}n de resistencias.
 Seguidamente se estudia el efecto Joule y se deducen las leyes de Kirchhoff
 a partir de la ecuaci\'{o}n de continuidad y de la conservaci\'{o}n de la 
energ\'{\i}a. Finalmente se utilizar\'{a}n estas leyes para resolver algunos 
circuitos.




\item {\bf Interacci\'{o}n magn\'{e}tica.}
En este cap\'{\i}tulo se estudia el campo magn\'{e}tico en 
r\'{e}gimen estacionario. En primer lugar se introduce el concepto de 
campo magn\'{e}tico a partir de los efectos que produce sobre una carga
 en movimiento  sin mencionar los agentes que lo producen.
 Se introduce la expresi\'{o}n de la fuerza magn\'{e}tica y se analiza el 
movimiento de una carga en un campo magn\'{e}tico uniforme. 
Seguidamente se presentan algunos ejemplos de aplicaci\'{o}n de este fen\'{o}meno:
 selector de velocidades para un chorro de part\'{\i}culas cargadas,
 espectr\'{o}metro de masas, ciclotr\'{o}n, etc... 
Se extiende la expresi\'{o}n de la fuerza magn\'{e}tica al caso de corrientes
 el\'{e}ctricas, en particular sobre espiras planas, y se define el 
momento dipolar magn\'{e}tico. 
%Como ejemplo de efecto de un campo magn\'{e}tico sobre una corriente se 
%analiza el efecto Hall. 
A continuaci\'{o}n se analiza el campo creado por una carga en movimiento
y se extiende el an\'{a}lisis al campo
 creado por una corriente el\'{e}ctrica estacionaria introduciendo
la ley de Biot y Savart. 
 Despu\'{e}s se calcula la fuerza entre dos hilos infinitos y paralelos por 
los que circulan corrientes el\'{e}ctricas.
Se introduce la ley de Amp\`{e}re y 
junto con la ley de Biot y Savart se utilizan
 para determinar el campo creado por diferentes 
distribuciones de corrientes. En particular, se calcula el campo magn\'{e}tico
 creado por una espira circular a grandes distancias de su centro, 
y se compara su dependencia con  la del campo el\'{e}ctrico
 creado por un dipolo el\'{e}ctrico. 
Seguidamente se presenta la ley de Gauss del campo magn\'{e}tico y se
 discuten las diferencias con el caso el\'{e}ctrico.
%{\color{red} 
Para finalizar se hace una breve discusi\'{o}n del campo magnetost\'{a}tico 
en un medio material, comenzando por describir el comportamiento de la 
materia en presencia de un campo magn\'{e}tico externo. Se introducen los
 conceptos de imanaci\'{o}n y de densidad de corriente de imanaci\'{o}n.
 Seguidamente se generaliza la ley de Amp\`{e}re, definiendo el vector
 excitaci\'{o}n magn\'{e}tica, la susceptibilidad y la permeabilidad
 magn\'{e}ticas. A continuaci\'{o}n, mediante un sencillo modelo de electrones 
girando en \'{o}rbitas circulares alrededor de los n\'{u}cleos at\'{o}micos, 
se relaciona la imanaci\'{o}n de un material con sus caracter\'{\i}sticas
microsc\'{o}picas. A partir de este modelo se establecen las diferencias entre 
los materiales diamagn\'{e}ticos, paramagn\'{e}ticos y ferromagn\'{e}ticos.
%}


\item {\bf Campos electromagn\'{e}ticos dependientes del tiempo.}
En este cap\'{\i}tulo se analizan los campos electromagn\'{e}ticos dependientes 
del tiempo. Comienza con una descripci\'{o}n de los experimentos de Faraday y tras
introducir el concepto de flujo del campo magn\'{e}tico,
 se enuncian la ley de Faraday-Henry y la ley de Lenz. 
Se presentan algunas aplicaciones pr\'{a}cticas de dichas leyes como son el
 generador y el motor el\'{e}ctrico. A continuaci\'{o}n se estudian los 
fen\'{o}menos de inducci\'{o}n (autoinducci\'{o}n e inducci\'{o}n mutua) y, 
a modo de ejemplo, se calculan el coeficiente de autoinducci\'{o}n de un 
solenoide y el coeficiente de inducci\'{o}n mutua de dos solenoides. 
Despu\'{e}s se calcula la energ\'{\i}a asociada al campo magn\'{e}tico almacenada
 en una bobina para obtener a continuaci\'{o}n la densidad de energ\'{\i}a del
 campo magn\'{e}tico y la del campo electromagn\'{e}tico en general. 
Seguidamente se analizan los circuitos RLC, poniendo de manifiesto 
la equivalencia de su ecuaci\'{o}n fundamental y la del oscilador arm\'{o}nico 
forzado y con amortiguamiento. En el siguiente apartado se introduce la 
correcci\'{o}n llevada a cabo por Maxwell a la ley de Amp\`{e}re para 
hacerla compatible con la ley de conservaci\'{o}n de la carga en el caso no estacionario.
 Finalmente se presentan las cuatro ecuaciones de Maxwell, que constituyen las leyes fundamentales del electromagnetismo.


\item {\bf Ondas electromagn\'{e}ticas.}
Comienza el cap\'{\i}tulo obteniendo la ecuaci\'{o}n de ondas en el vac\'{\i}o,
  para el campo el\'{e}ctrico y para el campo magn\'{e}tico. En primer
 lugar se estudian las ondas planas, poniendo de manifiesto el car\'{a}cter 
transversal de las mismas, se establece la relaci\'{o}n entre los campos 
el\'{e}ctrico y magn\'{e}tico y se definen los estados de polarizaci\'{o}n.
 Se pone de manifiesto el hecho de que las ondas electromagn\'{e}ticas
 transportan momento lineal y energ\'{\i}a y se define el vector de Poynting.
 A continuaci\'{o}n se describen algunas fuentes de ondas electromagn\'{e}ticas:
 dipolo el\'{e}ctrico o magn\'{e}tico oscilante, carga acelerada, etc... 
Se realiza tambi\'{e}n un breve an\'{a}lisis de la propagaci\'{o}n de las ondas 
en la materia, relacionando la velocidad de propagaci\'{o}n con la permitividad y 
permeabilidad del medio, y se define el \'{\i}ndice de refracci\'{o}n.
 Finalmente se describe el espectro electromagn\'{e}tico.
%, y se presenta la ecuación del efecto Doppler para las ondas electromagnéticas. Esta expresión se justificará más adelante, en el capítulo dedicado a la relatividad especial



\item {\bf \'{O}ptica.}
Para terminar el bloque tem\'{a}tico dedicado al electromagnetismo se estudia la luz,
 haciendo hincapi\'{e} en que la \'{O}ptica es el estudio de los fen\'{o}menos 
asociados a una parte del espectro electromagn\'{e}tico descrito en el tema anterior.
 En la primera parte del cap\'{\i}tulo se enuncia el principio de Huygens y se
 obtienen las leyes de reflexi\'{o}n y refracci\'{o}n de la luz en una superficie
 plana que separa dos medios con \'{\i}ndice de refracci\'{o}n diferente.
 Seguidamente se pone de manifiesto la dependencia del \'{\i}ndice de refracci\'{o}n
 con la longitud de onda mediante el ejemplo de la dispersi\'{o}n en un prisma.
 La segunda parte del tema est\'{a} dedicada a la \'{O}ptica Geom\'{e}trica.
 Despues de establecer los l\'{\i}mites de validez de la aproximaci\'{o}n que 
supone  representar la propagaci\'{o}n de las ondas mediante rayos,
 se introduce 
el concepto de camino \'{o}ptico y se presenta el principio de Fermat.
 A continuaci\'{o}n comienza el estudio de los sistemas \'{o}pticos: espejos, lentes, 
combinaciones de lentes, y sus aplicaciones m\'{a}s inmediatas:  
Lupa, c\'{a}mara fotogr\'{a}fica, microscopio y el telescopio. Mediante algunos ejemplos 
sencillos se calcula la posici\'{o}n de la imagen a trav\'{e}s de los sistemas 
\'{o}pticos anteriores, su aumento lateral y sus características. 


\end{enumerate}


% Python
\subsubsection{\large Pr\'{a}cticas de ordenador con {\it python}}
\begin{enumerate}
\item Ejemplo. Movimiento bajo una fuerza constante
\item P\'{e}ndulo de Foucault
\item Oscilador arm\'{o}nico de dos dimensiones. Diagramas de Lissajous
\item Oscilador amortiguado
\item Oscilador forzado
\item Movimiento bajo una fuerza disipativa
\item Circuito RLC en serie
\item Circuito RC
\item Circuito RL
\item Circuito LC
\end{enumerate}


\newpage
% Practicas
\subsection{Programa de pr\'{a}cticas: T\'{e}cnicas Experimentales I}

\subsubsection{\large Introducci\'{o}n}


Como ya se ha se\~{n}alado anteriormente, 
el trabajo de laboratorio es fundamental en el proceso de 
formaci\'{o}n de un Graduado en F\'{\i}sica o Graduado en Ingenier\'{\i}a Electr\'{o}nica.
Es por ello que,
aunque la
asignatura   {\it  T\'{e}cnicas Experimentales I} no constituye
el objeto de este Proyecto Docente, incluyo aqu\'{\i} los contenidos
del programa de pr\'{a}cticas  como
parte complementaria de la asignatura {\it F\'{\i}sica General}.

 El trabajo de laboratorio implica una participaci\'{o}n directa del alumno 
en el proceso de aprendizaje y cubre dos aspectos esenciales:
 en primer lugar, la aplicaci\'{o}n de los diferentes conceptos
 aprendidos en el curso te\'{o}rico, facilitando una mejor comprensi\'{o}n
 de los mismos y, por otro lado, el desarrollo de habilidades necesarias 
para la labor experimental, adquiriendo experiencia en el manejo de
 instrumentos de medida.

El programa de {\it  T\'{e}cnicas Experimentales I} (6.0 ECTS) est\'{a}
formado por una introducci\'{o}n te\'{o}rico-pr\'{a}ctica  impartida
en el aula y por 10  pr\'{a}cticas de laboratorio que se realizar\'{a}n
en el laboratorio en sesiones de  cuatro horas de duraci\'{o}n.
 Adem\'{a}s, se pueden incluir algunas clases de laboratorio 
adicionales para la repetici\'{o}n de pr\'{a}cticas y correcci\'{o}n
 de los informes.

Asimismo, se hace entrega a los alumnos de un cuaderno de 
pr\'{a}cticas en el que se explica la forma de realizar cada una de 
ellas y se incluye adem\'{a}s una serie de normas para la entrega de los 
informes de las pr\'{a}cticas.


\subsubsection{\large Programa}

\noindent
\begin{enumerate} [{\bf I.}]\addtolength{\itemsep}{-0.25\baselineskip}
\item {\bf  Introducci\'{o}n} \hfill{(0.7 ECTS)}
 \begin{enumerate}[{1.}] \addtolength{\itemsep}{-0.25\baselineskip} 
 \item C\'{a}lculo  de errores y tratamiento de datos
 \item Manejo de programas de gr\'{a}ficos y tratamiento de datos
 \item Presentaci\'{o}n de informes
 \end{enumerate}
\item {\bf  Instrumentos de medida}\hfill{ (0.3 ECTS)}
\begin{enumerate}[{1.}] \addtolength{\itemsep}{-0.25\baselineskip} 
	\item Nonius y micr\'{o}metro
        \item Fuentes de alimentaci\'{o}n
        \item Osciloscopio
        \item Mult\'{\i}metro
	\item Componentes el\'{e}ctricos
\end{enumerate}
\item {\bf  Complementos te\'{o}ricos preparatorios} \hfill{ (1.0 ECTS)}
\begin{enumerate}[{1.}] \addtolength{\itemsep}{-0.25\baselineskip}
        \item Teor\'{\i}a de circuitos
\end{enumerate}
\item {\bf  Pr\'{a}cticas de Mec\'{a}nica, Electromagnetismo y \'{O}ptica}
\hfill{(4.0 ECTS)}
\noindent
\begin{enumerate} [{\bf 1. }]
%\item{\bf Ca\'{\i}da libre. Medida de g} \\
%{\color{red} IDATZI}

\item {\bf El p\'{e}ndulo f\'{\i}sico. Medida de g}.\\
Se investiga la relaci\'{o}n entre el per\'{\i}odo de 
oscilaci\'{o}n de una varilla delgada y la distancia del eje de 
oscilaci\'{o}n al centro de gravedad de la misma.
 Con estos datos se calcula el valor de la aceleraci\'{o}n de la gravedad. 
\item {\bf  Movimiento arm\'{o}nico simple. Ley de Hooke.}\\
Se analiza experimentalmente el movimiento
 peri\'{o}dico de una masa suspendida de un muelle.
 Se determina la dependencia entre el per\'{\i}odo de oscilaci\'{o}n
 y la masa suspendida, y se obtiene el valor de la constante
 el\'{a}stica del muelle. Este resultado  se compara con el 
valor deducido de la ley de Hooke, midiendo el alargamiento del 
muelle en funci\'{o}n de la masa colocada.
\item {\bf   Plano inclinado: oscilaciones. Muelles en serie y en paralelo.}\\
En primer lugar se determina la constante el\'{a}stica de un muelle.
 Posteriormente se estudia el acoplamiento de las constantes  
el\'{a}sticas de dos muelles seg\'{u}n est\'{e}n 
dispuestos en serie o en paralelo.
 Finalmente se mide la aceleraci\'{o}n de la gravedad utilizando 
un dispositivo experimental basado en el deslizamiento
 de un cuerpo en un plano inclinado.
\item {\bf Momento de inercia}\\
Se analiza el movimiento de un disco acoplado a un tambor que gira por 
el par  ejercido por una masa suspendida
de una cuerda enrollada en el mismo.
Midiendo el tiempo de ca\'{\i}da de la masa y su aceleraci\'{o}n
se determina el momento de inercia del disco.

%  Pendulo de torsion sustituida por momento de inercia del disco%%%%
%\item {\bf   P\'{e}ndulo de torsi\'{o}n. Momento de inercia.}\\
%Se construye un p\'{e}ndulo de torsi\'{o}n con un hilo met\'{a}lico
% y un tubo hueco en su parte inferior. Utilizando el m\'{e}todo de 
%Maxwell se determina la constante de torsi\'{o}n del hilo y 
%el momento de inercia de una barra cil\'{\i}ndrica.
%\item{\bf P\'{e}ndulo bal\'{\i}stico. Conservaci\'{o}n del momento lineal.} \\
%{\color{red} IDATZI}

\item {\bf   Medida de la velocidad del sonido. Tubo de resonancia.}\\
Utilizando el fen\'{o}meno de resonancia de las ondas sonoras en un 
tubo de longitud variable, se determina experimentalmente 
la velocidad del sonido en el aire a temperatura ambiente.

%\item {\bf Ley de Coulomb. Determinaci\'{o}n de la constante de coulomb}\\
%{\color{red} IDATZI}

\item {\bf   Corriente continua I. Resistencia interna de una fuente.}\\
Se aprende a manejar el mult\'{\i}metro y se verifican la ley de Ohm
y las leyes de Kirchoff para circuitos de corriente continua.
 Se miden las resistencias conectadas en distintas configuraciones
 y se estima
 la resistencia interna y la fuerza electromotriz de una fuente.
\item {\bf   Corriente continua II. Curva caracter\'{\i}stica de una 
l\'{a}mpara.}\\
Se discuten situaciones en las que la inserci\'{o}n de instrumentos
de medida en circuitos de corriente continua altera las caracter\'{\i}sticas
del circuito. A continuaci\'{o}n se mide la curva caracter\'{\i}stica de una 
l\'{a}mpara.
\item {\bf  Instrumentos de medida. Descarga de un condensador.}\\
Se profundiza en el manejo del mult\'{\i}metro y de las fuentes de 
alimentaci\'{o}n
realizando medidas de la descarga de un condensador.
Se estudian las caracter\'{\i}sticas de los circuitos con condensadores
y las soluciones de las ecuaciones diferenciales simples. 
 \item {\bf   Corriente alterna. Circuito RLC. Manejo del osciloscopio.}\\
Se estudia la corriente alterna en circuitos con resistencias, condensadores 
y bobinas. Se determina experimentalmente el desfase entre
 la intensidad de corriente y la fuerza electromotriz en un circuito RLC,
 as\'{\i} como la ca\'{\i}da de potencial en cada uno de sus componentes.
 Todas estas medidas se realizan utilizando el osciloscopio como
 instrumento de visualizaci\'{o}n y medida de se\~{n}ales.
 Asimismo, se estudia el fen\'{o}meno de la resonancia en circuitos RLC.
\item {\bf   Corriente inducida en  un solenoide. El transformador.}\\
Se observan los fen\'{o}menos de
 inducci\'{o}n electromagn\'{e}tica entre imanes y circuitos.
 En particular se estudia el transformador y se comprueba 
la ley de Lenz.

%\item {\bf \'{O}ptica geom\'{e}trica. Formaci\'{o}n de im\'{a}genes
%con espejos y lentes.}\\
%{\color{red} IDATZI}

\end{enumerate}



\end{enumerate}







%\newpage
\section {Bibliograf\'{\i}a}
\label{PD-biblio}

La bibliograf\'{\i}a de F\'{\i}sica General es muy extensa y el contenido de 
casi todos los libros es bastante similar. Por otra parte, es dif\'{\i}cil
 encontrar un \'{u}nico texto que se adapte perfectamente a los prop\'{o}sitos
 del profesor, aunque cualquier texto es mejor para el alumno que 
los apuntes tomados en las clases. Un aspecto importante  es
 la escasez de libros de texto en lengua vasca. Este hecho implica que 
los alumnos deben realizar un esfuerzo adicional para adaptarse a
 la terminolog\'{\i}a en otras lenguas lo que, por otra parte,
es enriquecedor para el alumno.

La bibliograf\'{\i}a presentada contiene un n\'{u}mero suficiente de 
libros para que los alumnos tengan la posibilidad de elegir uno o varios 
libros de acuerdo con sus gustos, pero el n\'{u}mero de libros propuestos
 tampoco es excesivo para no desanimarlos ni desorientarlos.
 La primera parte de esta secci\'{o}n  incluye libros 
 para desarrollar el contenido te\'{o}rico de la 
asignatura, junto con un breve comentario sobre cada uno de ellos. 
Seguidamente se da una breve bibliograf\'{\i}a de profundizaci\'{o}n
para aquellos alumnos que deseen ampliar su conocimiento a un nivel superior
del impartido en clase.
A continuaci\'{o}n se propone un peque\~{n}o conjunto de libros de problemas. 
Hay que tener en cuenta que 
los libros de contenido te\'{o}rico  poseen todos ellos colecciones 
 de problemas del nivel adecuado; sin embargo, los
 alumnos pueden echar de menos el comprobar c\'{o}mo se resuelve alguno de
 los ejercicios que intentan solucionar. 
Es importante recomendar al alumno que utilice de forma adecuada los libros
 de problemas y no se limite a estudiar la forma en que se resuelven
 los ejercicios, sino que solamente recurra a la soluci\'{o}n una vez que lo haya
 resuelto \'{e}l mismo o haya reflexionado un tiempo suficiente sobre
 el problema.

Finalmente se incluyen varias direcciones de Internet donde pueden encontrarse 
 art\'{\i}culos divulgativos y materiales de inter\'{e}s.


\subsection{Libros de teor\'{\i}a recomendados a los estudiantes}
\begin{itemize}
\item
 TIPLER P. A.,  MOSCA G.,
{\it F\'{\i}sica para la ciencia y la tecnolog\'{\i}a 6$^a$ Ed.},
Editorial Revert\'{e}, Barcelona, 2010.

Es muy apropiado para desarrollar el programa presentado. Su dise\~{n}o es
 muy agradable e incluye una gran variedad de esquemas, ejemplos resueltos, 
cuestiones y problemas.
 Los desarrollos matem\'{a}ticos son tratados 
de forma sencilla, pero es muy rico en conceptos y elegante en los
 razonamientos. Posee adem\'{a}s  material adicional muy \'{u}til para el profesor. Las dos \'{u}ltimas ediciones en castellano, en 6 tomos,
 son muy pr\'{a}cticas.

\item 
FISHBANE, P. M., GASIOROWICZ, S., THORNTON, S. T.,
{\it Physics for Scientists and Engineers, 3$^a$ Ed.},
Addison-Wesley, 2003.

Al igual que el anterior es muy apropiado para desarrollar el programa 
propuesto. Se recomienda
la edici\'{o}n en ingl\'{e}s ya que la traducci\'{o}n al espa\~{n}ol es bastante pobre y 
no est\'{a} muy cuidada.

\item   
 FISHBANE, P. M., GASIOROWICZ, S., THORNTON, S. T. ,
{\it Fisika zientzialari eta ingeniarientzat}. Servicio editorial de la 
UPV/EHU, 2008.

Es la traducci\'{o}n al Euskera de la 2$^a$ edici\'{o}n en ingl\'{e}s. 
Se public\'{o} en 2008 y es el 
segundo libro publicado de F\'{\i}sica General a nivel universitario en lengua vasca.

\item 
YOUNG H. D.,  FREEDMAN R. A., 
{\it Sears  Zemansky F\'{\i}sica Universitaria,  12$^a$ Ed.},
Addison-Wesley, 2009.

Se trata de un texto cl\'{a}sico de F\'{\i}sica General que se ha modernizado al estilo 
de las anteriores referencias y ha mejorado mucho desde el punto de vista pedag\'{o}gico.
Las explicaciones son claras, con numerosos ejemplos y  contiene una 
gran variedad de ejercicios muy interesantes y adecuados para el nivel del curso que se propone.

\item
AGIRREGABIRIA J.M., DUOANDIKOETXEA A., ENSUNZA M.,ETXEBARRIA J.R., EZENARRO O., PITARKE J.M., TRANCHO A. Y UGALDE P., 
{\it  Fisika Orokorra 2$^a$ Ed.} UEU, Bilbao, 2003.

Es un libro basado en  las clases de F\'{\i}sica impartidas en la Facultad de Ciencias y su contenido ha sido elaborado a partir de libros de texto cl\'{a}sicos. De entre
todos los libros propuestos es el \'{u}nico que trata con rigor el tema del 
movimiento relativo tal y como se presenta en este proyecto docente.
Est\'a disponible en internet para su uso libre en 
 {\tt  www.buruxkak.org}

\end{itemize}

\subsection{Bibliograf\'{\i}a de profundizaci\'{o}n}
\begin{itemize}

\item FEYNMAN  R. P., LEIGHTON R. B. y SANDS M. L.,
 {\it The Feynman Lectures on Physics},
 Pearson-Addison-Wesley Iberoamericana 2006. \\
Sus tres vol\'{u}menes est\'{a}n basados en las clases impartidas por
 R. P. Feynman en Caltech en el periodo 1961-1963.
Tienen un enfoque totalmente distinto a los libros actuales de F\'{\i}sica 
universitaria. Est\'{a} estructurado en cap\'{\i}tulos cortos que pueden resultar 
de lectura interesante para aquellos estudiantes m\'{a}s avanzados y motivados.
Muchos de los temas tratados no son objeto del programa de esta asignatura 
y se estudiar\'{a}n en asignaturas de cursos superiores.



\item ALONSO M. y FINN E. J.,
 {\it F\'{\i}sica},
 Addison-Wesley 1995

Se trata de un esfuerzo de s\'{\i}ntesis y de puesta al d\'{\i}a de la edici\'{o}n en 
3 vol\'{u}menes publicada en 1976. Aquella edici\'{o}n era m\'{a}s formal desde el punto de
 vista matem\'{a}tico pero pedag\'{o}gicamente no era tan \'{u}til.
La edici\'{o}n resumida de 1995 cuida m\'{a}s el aspecto pedag\'{o}gico.

\end{itemize}

\subsection{Libros de problemas}
\begin{itemize}


\item HERN\'{A}NDEZ J., TOVAR J.,
 {\it Problemas de F\'{\i}sica: mec\'{a}nica}, 
Universidad de Ja\'{e}n, 2009.

Libro elaborado  a partir de problemas propuestos
en ex\'{a}menes y resueltos en clase durante los \'{u}ltimos a\~{n}os en la 
Escuela Polit\'{e}cnica Superior de la Universidad de Ja\'{e}n por los responsables de la 
asignatura en dicha escuela. Proporciona a los
alumnos un texto de referencia para establecer estrategias adecuadas en la resoluci\'{o}n de 
problemas. Los problemas resueltos est\'{a}n comentados y contienen una gran cantidad 
de figuras y diagramas. Trata todos los temas de mec\'{a}nica salvo la din\'{a}mica de 
fluidos.


\item BURBANO DE ERCILLA S., BURBANO GARC\'{I}A E.,
 {\it Problemas de F\'{\i}sica, 32$^a$ Ed.},
 Editorial T\'ebar, Madrid 2006.

Se trata de un libro muy cl\'{a}sico en la bibliograf\'{\i}a de 
F\'{\i}sica General de primer curso universitario. Contiene ejercicios a todos 
los niveles, desde bachillerato hasta nivel universitario. Es recomendable que los 
estudiantes lo utilicen con precauci\'{o}n, y no
debe utilizarse como una lectura. Al contrario, se animar\'{a} a los estudiantes
a realizar los 
ejercicios con su propia estrategia y utilizar el resultado del libro solo para 
comprobar que han llegado a la soluci\'{o}n correcta.

\item ENSUNZA M., ETXEBARRIA J.R., EZENARRO O., PITARKE J.M., UGALDE P. Y ZABALA N., 
{\it Fisika Orokorra, Ariketak},
 UEU, Iru\~{n}ea 1989.

Es un libro m\'{a}s modesto que los anteriores. Sin embargo tiene un nivel adecuado
para el programa que presento. Adem\'{a}s de contener problemas resueltos
con explicaciones claras ayudadas de diagramas y figuras, propone una serie de 
problemas al final de cada tema e indica la soluci\'{o}n a la que debe llegar el estudiante.

\item HALPERN, A.,
{\it 3000 Solved Problems in Physics. Schaum's Solved Problems Series},
Mc Graw Hill.

Se trata de una edici\'{o}n moderna y actualizada de uno de los libros cl\'{a}sicos de problemas
resueltos de la serie Schaum. Es muy completo. Contiene problemas de todos los temas propuestos en
este proyecto docente adem\'{a}s de temas de F\'{\i}sica General que por limitaciones de tiempo no est\'{a}n 
incluidos en el programa y que se ver\'{a}n en cursos superiores. 
Los problemas est\'{a}n resueltos apoyados por explicaciones claras y muchos diagramas de gran utilidad para 
el alumno. As\'{\i} mismo, y pese a ser un libro estadounidense, todas las unidades utilizadas son del sistema 
internacional.


\end{itemize}

\subsection{Revistas y direcciones de internet}
\begin{itemize}
\item La revista American Journal of Physics, editada por ``American Association of 
Physics Teachers'' presenta a menudo art\'{\i}culos de diferente dificultad 
destinados a profesores y estudiantes de F\'{\i}sica:
 \href{http://scitation.aip.org/ajp/}{http://scitation.aip.org/ajp/} 

\item La Real Sociedad Espa\~{n}ola de F\'{\i}sica (RSEF) en su p\'{a}gina WEB, 
zona de ``links'' da acceso a su revista, en la cual a menudo aparecen
 art\'{\i}culos divulgativos: \href{http://rsef.org}{http://rsef.org}


\item ``Open Courseware'' del Massachusetts Institute of Technology alberga materiales \'{u}tiles de sus cursos de F\'{\i}sica. 
\href{http://ocw.mit.edu/courses/physics/}{http://ocw.mit.edu/courses/physics/} 


\item Curso interactivo de F\'{\i}sica en Internet de Angel Franco, 
del Departamento de F\'{\i}sica Aplicada I de la UPV/EHU.
\href{http://www.sc.ehu.es/sbweb/fisica/}{http://www.sc.ehu.es/sbweb/fisica/}

\item  Repositorio de material educativo del consorcio ``Conceptual Learning of
 Science'': \href{http://www.colos.org/}{http://www.colos.org/}

\item Repositorio de materiales de Open Source Physics.
\href{http://www.compadre.org/osp/}{http://www.compadre.org/osp/}


\item The Python tutorial: 
\href{http://docs.python.org/py3k/tutorial/index.html}{http://docs.python.org/py3k/tutorial/index.html}


\end{itemize}

%\end{chapter}


