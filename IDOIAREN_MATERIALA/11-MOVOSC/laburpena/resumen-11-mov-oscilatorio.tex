\documentclass[a4paper,spanish,12pt]{article}

\usepackage[width=16cm,height=24cm]{geometry}
\usepackage{graphicx}
\usepackage[spanish]{babel}
\usepackage{times}


\begin{document}
\title{Resumen del tema 11.  Movimiento oscilatorio}
\author{Idoia Garc\'{\i}a de Gurtubay}
\date{}
\maketitle{}

\thispagestyle{empty}

En este tema se estudia el movimiento oscilatorio. Las oscilaciones se producen cuando se perturba un sistema y 
\'{e}ste pierde su posici\'{o}n de equilibrio.

Ejemplos de oscilaciones  en la vida cotidiana son: el balanceo de los barcos,
 el p\'{e}ndulo de un reloj o las cuerdas de los instrumentos musicales 
al producir sonidos. Otros 
ejemplos menos conocidos son las oscilaciones de las mol\'{e}culas de aire en las ondas sonoras y las oscilaciones 
de las corrientes el\'{e}ctricas en los aparatos de radio y televisi\'{o}n.

Comenzamos el tema repasando la ecuaci\'{o}n de movimiento de un oscilador 
arm\'{o}nico simple, que es la forma m\'{a}s b\'{a}sica de movimiento
 oscilatorio, que ya se estudi\'{o} en el tema 3 de cinem\'{a}tica, y
 analizamos la composici\'{o}n de movimientos arm\'{o}nicos en direcciones perpendiculares.

Aplicando la segunda ley de Newton y teniendo en cuenta que en
 un oscilador arm\'{o}nico simple
 la aceleraci\'{o}n es proporcional al desplazamiento, 
se plantea la ecuaci\'{o}n diferencial del oscilador.
Esta ecuaci\'{o}n es de gran importancia, ya que el oscilador
 arm\'{o}nico es un buen modelo
para entender muchos problemas de F\'{\i}sica. 
A continuaci\'{o}n se presentan algunos ejemplos
como el de una part\'{\i}cula sujeta a un resorte lineal que cumple la
 ley de Hooke, y
 el de un p\'{e}ndulo simple y un p\'{e}ndulo f\'{\i}sico que cumplen la misma ecuaci\'{o}n 
diferencial en la aproximaci\'{o}n de \'{a}ngulos peque\~{n}os.

Terminamos el an\'{a}lisis del oscilador arm\'{o}nico simple estudi\'{a}ndolo desde el 
punto de vista de la energ\'{\i}a.

A continuaci\'{o}n pasamos a analizar las oscilaciones amortiguadas, estudiando su movimiento 
en funci\'{o}n de la intensidad de 
la amortiguaci\'{o}n, y para finalizar introducimos una fuerza impulsora sinusoidal
para estudiar las oscilaciones forzadas y el fen\'{o}meno de resonancia.

Se describen brevemente las pr\'{a}cticas de laboratorio relacionadas
 con este tema, y 
se proponen los ejercicios  que los estudiantes deben preparar para las pr\'{a}cticas de aula.

Para acabar se realiza la pr\'{a}ctica de ordenador ``El p\'{e}ndulo de Foucault'' cuyo 
objetivo es calcular la velocidad angular del plano de oscilaci\'{o}n de un p\'{e}ndulo simple que
oscila a cierta latitud.

Hay que destacar que debido a que \'{e}ste  es uno de los \'{u}ltimos temas del bloque de mec\'{a}nica,
resulta muy adecuado para repasar todos los conceptos estudiados durante el 
cuatrimestre.

\vspace*{1cm}
Se adjuntan a este resumen los enunciados de los ejercicios propuestos, as\'{\i} como el gui\'{o}n de la
pr\'{a}ctica de ordenador ``El p\'{e}ndulo de Foucault''.





\end{document}
